\chapter{Serialization}


%---------------------------------------------
%---------------------------------------------
%---------------------------------------------

\section{Introduction}

\subsection{What is serialization?}

Serialisation is a very general mecanism that allow an object to describe
itself as a succession of information.   This succession of information is
then used by a \emph{receiver} (called an archive in MMVII); what does
do the receiver with this information is, as much possible, not the business
of the object, just describe yourself !  By the way, generaly the receiver will write or read the information
in some place with some format. So in most frequent case, the serialization is used
to save the current state of an object to a file, an re-built it later, generally in
another process.  In a photogragrammetric suite as {\tt MMVII},  where the computation
is made from a complex combination of (many) process the serialization is a key point.

What is the interest of this abstraction ? It's a classical problem of "traduction",
if you have $N$  source language, and $M$ target language and that you want to translate
source to target, the trivial way is to write $N*M$ translator; but generally, we prefer
to write a universal language $U$ , and write $N$ translator to $U$, and $M$ translator 
from $U$ , and we only write $N+M$ translators.

Here this is basically the same idea, with $N$ object and $M$ kind of receiver/format,
the descritpion by serialization beging the universal language, we have only $M+N$
translation to do.  One may think that there is not many receiver, but in fact there is many,
just to cite the existing one:

\begin{itemize}
	\item  {\tt xml} and {\tt  json}  format; in \emph{read and write mode},  the good thing
		with serialization is that if you can write an object you can also read it;

	\item  binary format, and raw-text format,  in \emph{read and write mode};

	\item  hash code, receiver are not necessarily files, and hash code can but computed
		from the accumulation of object;

	\item  specification file (to be finished);
\end{itemize}

%---------------------------------------------

\subsection{Principle of implementation in {\tt MMVII}}

The main principle principle of serialization is the following :

\begin{itemize}
       \item  all the receiver inherit from the pure abstract class {\tt cAr2007},
               when an objet describe itself it only knows this base-class;

       \item for describing itself, an object of type {\tt Type} only needs to define
	       the method {\tt void AddData(const  cAuxAr2007 \& , Type  \& );}

       \item the type {\tt cAuxAr2007} is just an {\tt cAr2007} + the eventual tag that will
             be used in tagged format \footnote{{\tt json,xml} and internal format {\tt tagt}}and ignored
             by others;
\end{itemize}

     %  -  -  -  -  -  -  -  -  -  -  -  -  -  - -  -  - - -

\subsubsection{Overview of class {\tt cAr2007}}

The class {\tt cAr2007} is an abstraction of the concept of receiver, it's mainly an interface class.
The main function to be overlaoded are the function that describe the serialization of atomic-base  value,
this is where everyting will be really done , the name is {\tt RawAddDataTerm}.

The atomic object are standard native elementary object : {\tt int,double, std::string, size\_t} ( plus
{\tt cRawData4Serial} we will describe later). Let take the example of {\tt int} we have, the declaration
of the function in the class  {\tt cAr2007} :

\begin{verbatim}
    virtual void RawAddDataTerm(int &    anI) =  0;
\end{verbatim}

Let examine now the overlaoding in different classes. For the binary formats for, reading/writing an object to a file,
, it's pretty obvious we just read/write the byte consituting the {\tt int}. So here is the overloading
for class write (class {\tt cOBin\_Ar2007}) and read (class {\tt cIBin\_Ar2007}) :

\begin{verbatim}
void cOBin_Ar2007::RawAddDataTerm(int &    anI) { mMMOs.Write(anI); }
void cIBin_Ar2007::RawAddDataTerm(int &    anI) { mMMIs.Read(anI); }
\end{verbatim}

For raw text-format, it's not very different ,  we write an {\tt int} to a string format in {\tt cOBaseTxt\_Ar2007}.
The {\tt mXTerm=true;}, is only to handle some cosmeticall pretty print, we dont detail here.

\begin{verbatim}
void cOBaseTxt_Ar2007::RawAddDataTerm(int &    anI) {Ofs() <<anI; mXTerm=true;}
\end{verbatim}

For read, we just make a conversion string to int from next string.

\begin{verbatim}
// get a string and put in I its conversion to int
void cIBaseTxt_Ar2007::RawAddDataTerm(int &    anI)
{
    FromS(GetNextStdString(),anI);
}
\end{verbatim}


     %  -  -  -  -  -  -  -  -  -  -  -  -  -  - -  -  - - -

\subsubsection{The function {\tt AddData}}

The principe is that the function {\tt void AddData(const  cAuxAr2007 \& anAux,Type \&)} must be defined
for each type that must be serialized.  There is basically $4$ types of
construction for defining the method {\tt AddData} :

\begin{enumerate}
	\item for basic-atomic type  ({\tt int, double \dots}), the method
		simple calls {\tt RawAddDataTerm} of the archive;

	\item for atomic type that can be converted to basic-atomic type, 
		the serialization can use the template method {\tt TplAddDataTermByCast}
		defined in the class {\tt cAr2007};

	\item for a certain number of standar template type construction (like {\tt std::vector, std::map}),
		there is template function in file {\tt include/MMVII\_2Include\_Serial\_Tpl.h} that define
		{\tt AddData} for these template type;  

	\item for user defines-type the user must define the function, typicall it will be a 
		sucession (serial) call to method {\tt AddData} of its elements;
\end{enumerate}

\begin{verbatim}
// Case 1 atomic type
void AddData(const  cAuxAr2007 & anAux, int  &  aVal) {anAux.Ar().RawAddDataTerm(aVal); }

// Case 2 AddData by cast
void AddData(const  cAuxAr2007 & anAux, tU_INT1  &  aVal) { anAux.Ar().TplAddDataTermByCast(anAux,aVal,(int*)nullptr); }

// Case 3 AddData for template type
template <class Type> void AddData(const cAuxAr2007 & anAux,std::vector<Type> & aL) { StdContAddData(anAux,aL); }

// Case 4 AddData for user defined type , cTestSerial0 : is made of mP1+mI1+mR4+mP2

void AddData(const cAuxAr2007 & anAux, cTestSerial0 &    aTS0)
{
    AddData(cAuxAr2007("P1",anAux),aTS0.mP1);
    // AddComment(anAux.Ar(),"This is P1");
    anAux.Ar().AddComment("This is P1");

    AddData(cAuxAr2007("I1",anAux),aTS0.mI1);
    AddData(cAuxAr2007("R4",anAux),aTS0.mR4);
    AddData(cAuxAr2007("P2",anAux),aTS0.mP2);
}
\end{verbatim}

     %  -  -  -  -  -  -  -  -  -  -  -  -  -  - -  -  - - -

\subsubsection{Class {\tt cAuxAr2007}}

The object {\tt cAuxAr2007} contains an {\tt cAr2007} plus   a tag.
The tag is used to give a name to a part of the object that is serialized.
There is three different way for the receiver to use these tags :

\begin{itemize}
    \item in a tagged format, like {\tt xml-json}, in write mode the
	    tag are printed to describe the signification of a given field;

    \item in a tagged format in read mode, the tag is used to control the integrity
	    of the file (i.e at a given moment, we know that we must get such given tag);

    \item for non tagged format, the tag are just igonred.
\end{itemize}




\section{Files related to serialization}

The main file containing serialization.





%---------------------------------------------
%---------------------------------------------
%---------------------------------------------

\section{Serialization by streaming}

%---------------------------------------------
%---------------------------------------------
%---------------------------------------------


\section{Serialization by tree}
