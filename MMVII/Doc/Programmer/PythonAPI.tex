

\chapter{Python API}


%---------------------------------------------
%---------------------------------------------
%---------------------------------------------
\section{Using Python API}

Some of MicMac V2 internal features are usable in python via a python module generated
with \textbf{pybind11} (\url{https://github.com/pybind/pybind11}).

Only a selection of MMVII classes, adapted to python usage, is made available.

For compilation and usage, see \texttt{MMVII/apib11/README.md}.

Most of the classes in the module are illustrated in \texttt{MMVII/apib11/examples}. 

%---------------------------------------------
%---------------------------------------------
%---------------------------------------------
\section{Extending Python API}
\label{Prog:Py:Extending}
%---------------------------------------------
\subsection{Introduction}

The sources files describing the classes to export to the python module are \texttt{MMVII/apib11/py\_MMVII*.cpp}.

To start using \textbf{pybind11}, the minimal documentation is:
\begin{itemize}
  \item \url{https://pybind11.readthedocs.io/en/stable/basics.html}
  \item \url{https://pybind11.readthedocs.io/en/stable/classes.html}
\end{itemize}

%---------------------------------------------
\subsection{Example}


An illustration on how to add a new class to the module can be found in \texttt{MMVII/apib11/py\_MMVII\_Aime.cpp}.

Its main function
\begin{lstlisting}
    void pyb_init_Aime(py::module_ &m) {
\end{lstlisting}
will add classes, methods and attibutes to the module.

This function is declared in \texttt{MMVII/apib11/py\_MMVII.h} and called in \texttt{MMVII/apib11/py\_MMVII.cpp}.

A minimal class export is:
\begin{lstlisting}
py::class_<cAimeDescriptor>(m,"AimeDescriptor",DOC(MMVII_cAimeDescriptor,cAimeDescriptor))
        .def(py::init<>())
        .def("ILP", &cAimeDescriptor::ILP, DOC(MMVII_cAimeDescriptor, ILP))
        ;
\end{lstlisting}

It creates a python class \texttt{AimeDescriptor} from the c++ class \texttt{cAimeDescriptor}.
The doxygen comment for the class is inserted in the module with the \texttt{DOC(namespace, part)} syntax
(see \url{https://github.com/pybind/pybind11_mkdoc} for more information).

It then adds a constructor and a method \texttt{ILP}, corresponding to \texttt{cAimeDescriptor::ILP}, with its documentation.

%---------------------------------------------
\subsection{Getting further}

See \textbf{pybind11} documentation, as it shows many more features.

Some hints to get further:
\begin{itemize}
  \item \textbf{pybind11\_mkdoc} is very sensible to doxygen syntax. If two kinds of comments are used for the same thing,
 it may refuse to use any. Please check that the python documentation is equivalent its c++ counterpart.
  \item if class \texttt{Bar} extends \texttt{Foo}, use
\begin{lstlisting}
    py::class_<Bar, Foo>(m,"Bar",...
\end{lstlisting}
syntax to give access to \texttt{Foo} methods from \texttt{Bar} python objects (if \texttt{Foo} is accessible in python).
  \item to name functions parameters, use \texttt{"name"\_a} syntax :
\begin{lstlisting}
    .def("toFile", &tPCIC::ToFile,"filename"_a,DOC(MMVII_cPerspCamIntrCalib, ToFile))
\end{lstlisting}
  \item \textbf{return\_value\_policy} is a fundamental concept to avoid crashes and memory corruption (see \url{https://pybind11.readthedocs.io/en/stable/advanced/functions.html})
  \item try to make pythonic classes (e.g. initialization from \texttt{py::tuple}, see \texttt{MMVII/apib11/py\_MMVII\_Ptxd.cpp})
  \item add \texttt{\_\_repr\_\_} and other useful methods to python classes to make them more pythonic:
\begin{lstlisting}
    .def("__repr__",
         [](const tMPP2I &m) {
           std::ostringstream ss;
           ss.precision(15);
           ss << "MapPProj2Im(" << m.F() << ",(" << m.PP().x() << "," << m.PP().y() << "))";
           return ss.str();
     })
\end{lstlisting}
  \item add numpy direct compatibility if needed (see \url{https://pybind11.readthedocs.io/en/stable/advanced/pycpp/numpy.html})
  \item template methods can be used to export c++ templates classes (with a unique name for each type in python, see \texttt{MMVII/apib11/py\_MMVII\_Mappings.cpp})
\end{itemize}


