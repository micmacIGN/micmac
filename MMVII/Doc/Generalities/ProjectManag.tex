\chapter{Project management command}

%---------------------------------------------
%---------------------------------------------
%---------------------------------------------
\section{Readible file formats}

\subsection{Readible/Binary}

As photogrammetric pipeline is a complex process, made of several computation,
the result of each computation has to be writen in some files that
will be read at by next process. There is basically two familly of such files:

\begin{itemize}
   \item files that may have some interest to  be read or manipulated by human
         or other programm, in this case {\tt MMVII} standar tagged format
	 as {\tt xml} of {\tt json}; example of such file are calibration
	 or pose estimation;

   \item files that have probably low interest for human, and for efficiency 
         they are stored in binary format  (note that there exist also a
         text version of this binary format, note easy to read, but that facilitate
	   import export);
\end{itemize}

\subsection{Xml and Json files}

{\tt MMVII} offer the possibility to export data in two different tagged format : {\tt xml} and
{\tt json}. By default it's {\tt xml}, but this can be changed using mecanism descrined in \ref{UserParametrisation}.
Implicit conversion may appear in a near future for sharing files between user,
by the way it's recommanded that user make a choice once for all  to avoid any problem.

The folder {\tt MMVII-UseCaseDataSet/SampleFiles} contains examples
of files in {\tt xml} and {\tt json}. 

Note that {\tt xml} allows real coments and, for example, in the file {\tt Calib...xml}, they are used to
explicit  the meanin of distorsion parameters. As {\tt json} do not have
real comments, we use the special tags , for example in file
{\tt Calib...json} you can find {\tt "<!--comment6-->":"(X,0)"}
(corresponding to  {\tt <!--(X,0)-->} in {\tt xml}).

These file are made to be relatively easy to read. They can also be created or
modified easily by user or programm however there are some \emph{strict} rules to observe so that such
file remain valid . Starting from a valid file, here are example of things that can be done to maintain
validity :

\begin{itemize}
       \item change the value of atoms while respecting their type (float, int, string \dots);

       \item add or supress an element  in a sequence or a map,  see for example  the file
	       {\tt TestObj.*}, the element of sequence are tagged {\tt el} in {\tt xml}, while the element of
		pair key-value  in a map are tagged {\tt K/V};

       \item supress or add any comment;

       \item supress or add an optionnal value (rare case for now), the optionnal value can
	     be  detected as their tags begin by {\tt Opt:}, see file {\tt F\_T2.*};
\end{itemize}

And here a \emph{non-exhautive} list of thing you cant do :

\begin{itemize}
        \item obviously create a non-valide xml/json file;

	\item swap two elements of different tags  \emph{very bad, dont do it, strictly forbiden, naughty , Micmac is whatching you \dots}
          also it woul be theoretically possible to recover the information, this would involve unnecessary software devlopment
	  that wont be done;

        \item supress a non optional tag ;

	\item add/supress a value in fixed size tab (used to represent point for exeample);
\end{itemize}

The format of each file is :

\begin{itemize}
	\item a single root node , named {\tt root} in {\tt xml} and anonymous in Jason,
	\item root node has exacly $3$ sub-node that have  a fixed tag

	\item the first node is the type, it must be tagged {\tt Type} and its value must {\tt MMVII\_Serialization} 
              signing the fact that is was created by {\tt MMVII};

      \item the second node is a version number, it must be tagged {\tt Version}, its value will allow
	    compatibility policy with older files, unused at the time being;
             
    \item the third node must be tagged {\tt Data} and contains in fact the data itself !
          in most frequent case, it will contain a single node (see {\tt Calib*, Ori*} ) allowing
          some type-checking at the very begining   of the command (i.e we can check that {\tt Calib*}
          are most probably  MMVII-calibration files as their data contains the single tag {\tt InternalCalibration});
\end{itemize}


%---------------------------------------------
%---------------------------------------------
%---------------------------------------------

\section{User specific parametrization}

\label{UserParametrisation}

Sometimes user need to fix some stable default parametrization of {\tt MMVII}, 
the parameters acessible for now are :

\begin{itemize}
    \item maximal number of processor that will be allowed when  {\tt MMVII} execute
          parallel computation;

     \item default format for human readible export, it can be for now \emph{xml} or \emph{json};

      \item name of the user, this field is for now rather target for devloppers who want to
            include in this devlopped code   some message/test ...  specific to himself
           (for example, in some suspicipus case, I will make a breakpoint for myself, but will
		not bother others with that as it should work if we are a bit lucky);
\end{itemize}

This will probably evolve, for example we can imagine to have some category of user.
The way this is done is done by filling a {\tt xml} file located in the folder 
{\tt MMVII-LocalParameters}, for example we have the file {\tt Default/MMVII-UserOfProfile.xml} :

\begin{verbatim}
<?xml version="1.0" encoding="ISO8859-1" standalone="yes" ?>
<Root>
   <Type>"MMVII_Serialization"</Type>
   <Version>"0.00"</Version>
   <Data>
      <UserName>"Uknown"</UserName>
      <NbProcMax>1000</NbProcMax>
      <SerialMode>"xml"</SerialMode>
   </Data>
</Root>
\end{verbatim}

Now the question is how will {\tt MMVII} locate file to use. This name of the folder containing
this file will be contained in a file {\tt MMVII-CurentPofile.xml} if it exists,
or in {\tt Default-MMVII-CurentPofile.xml} in the other case. This file should always exist
at it is a git-shared file {\emph should not ne modified except by devlopers}.
If we take a look at  it : 

\begin{verbatim}
<?xml version="1.0" encoding="ISO8859-1" standalone="yes" ?>
<Root>
   <Type>"MMVII_Serialization"</Type>
   <Version>"0.00"</Version>
   <Data>
      <NameProfile>"Default"</NameProfile>
   </Data>
</Root>
\end{verbatim}

We see that the only meaningfull part is {\tt NameProfile} indicating the name
of the folder. For creating a profile, what user must do is :

\begin{itemize}
    \item creat a file  {\tt MMVII-CurentPofile.xml} if it was not already done;
    \item fill the field {\tt <NameProfile>};
    \item create in the corresponding folder, a file {\tt MMVII-UserOfProfile.xml}
\end{itemize}

The idea, is that a user can have several predefined profile in different folders,
and only modify  {\tt MMVII-CurentPofile.xml}.


%---------------------------------------------
%---------------------------------------------
%---------------------------------------------

\section{General data organization}

The data organization in {\tt MMVII} is the following :

\begin{itemize}
     \item for a given project, almost all the file created are located somewhere under
           the same folder {\tt MMVII-PhgrProj},  this is necessary because during a complex 
           photogrammetric process many files will be created and we dont want to encumber
           the main folder;

     \item an example of such folder (resulting from the coded-target usecase) can be
	     found under {\tt MMVII-UseCaseDataSet/SampleFiles}

     \item for each kind of processing, there exist a subfolder corresponding to the "nature"
            of the data stored;

    \item  in the example there is of folder for orientation {\tt Ori}, one for points 3d coordinates
    {\tt ObjCoordWorld}, one for point measurement
	   {\tt ObjMesInstr}, one fore handling meta data {\tt MetaData}, one for storing reports
           {\tt Reports}, all the name of this subfolder are defined by {\tt MMVII} and cannot be changed
            by the user;

    \item there is (will be) many other king of folder : homologous point, radiometric model, radiometric data,
           \dots

     \item in each of the predefined folder, there exist different subfolder corresponding to different step
           of the process; the name of this subfolder are specified by user, sometime as input to a command,
	   sometime as output to a command; typically the output a command being the input of the next command;
\end{itemize}

In this example, we have $4$ different folder for orientation in {\tt Ori} :

\begin{itemize}
      \item  {\tt 11P} which is the result of initial estimation using uncalibrated spaced resection
	      (with the "$11$ parameters method");

      \item  {\tt Resec} which is the result of pose estimation using calibrated camera, using as input
	      the calibration stored in {\tt 11P};

      \item  {\tt BA} which is the result of pose estimation using bundled adjusment, using as input
	      for initial value the pose stored in {\tt Resec};

      \item   the refined pose of  {\tt BA} are used as input to drive a research of uncoded target with
	      accurate initial position in imahe;
 
       \item using the additional target, {\tt BA} is used as input to a new bundle adjusment and the result is
             stored in {\tt BA2}.

\end{itemize}


%---------------------------------------------
%---------------------------------------------
%---------------------------------------------


\section{Help command}
\index{Help}

\label{HelpCmd}


\section{Bench command}


\begin{itemize}
    \item {\tt  MMVII Bench 2 }  : standard mode  for execuring all bench at level 2

    \item {\tt MMVII Bench 2 PatBench=.*Der.* Show=0}  : execute benches matchin {\tt ".*Der.*"},
          {\tt  Show} is explicite as, by default, it is set to {\tt  true} 
          when {\tt  PatBench} is set;
   
    \item {\tt MMVII Bench 1 PatBench=XXX} : pattern specified but no match, print all benche existing


    \item {\tt MMVII Bench 2 KeyBug=Debord\_M1 }  : force the generation of  a given error

    \item {\tt MMVII Bench 1 KeyBug=XXXX }  : will print all possible value for explicit error generation


    \item {\tt MMVII Bench 1 PatBench=InspectCube }  : as InspectCube is not a bench function, but 
         only print information, exact name must be set with {\tt PatBench}

\end{itemize}






{\tt MMVII Bench 2 KeyBug=XXX }  : standard mode  for execuring all bench at level 2

{\tt MMVII Bench 1 PatBench=MemoryOperation KeyBug} : pattern specified but no match, print all benche existing

