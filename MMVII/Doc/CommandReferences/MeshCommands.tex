\chapter{Mesh related commands}


CNR  CERN 


%-----------------------------------------------------------------------
%-----------------------------------------------------------------------
%-----------------------------------------------------------------------

\section{Process of 3D meshes}

 {\tt MMVII} is not a point-cloud/mesh processing tool. However, during some devlopment, some
functionnalities appeared to be necessary and not easily accesible on open source.
As they may be usable in other context I describe them here.

%-----------------------------------------------------------------------
\subsection{MeshCheck}

\begin{verbatim}
 == Mandatory unnamed args : ==
  * string [FDP,Cloud] :: Name of input cloud/mesh
 == Optional named args : ==
  * [Name=Bin] bool :: Generate out in binary format ,[Default=false]
  * [Name=Out] string :: Name of output file if correction are done
  * [Name=Do2DC] bool :: check also as a 2D-triangulation (orientation) ,[Default=false]
  * [Name=Correct] bool :: Do correction, Defaut: Do It Out specified
\end{verbatim}

{\tt MicMac-V1} can generate mesh from a 3d point cloud with normal. For this it
uses the open source tool devlopped by M Kazhdan. Also this tool is robust,
it has sometime topologicall problem that can block further processing.
The command {\tt MeshCheck} has been developped to  detect and potentially
correct these problems.

The first problem detected   is the fact that the same point that is duplicate in
a single triangle  like $ABC$  with $P_A=P_B$. This point are detected and
if the option {\tt Correct} is activated, a graph-quotient algorithm is applied 
on point that are detected as multiple.

The second check is $3d$ surface orientatbility . A message indicating the detected
pair of adjacent triangle badly oriented is printed.  No correction is
done, because untill now no such problem was encounterred.

If the option {\tt Do2DC} is activated, the triangulation is considered
as a $2$ triangle (setting $z$ to $0$) and  the $2d$ orientation correctness is
tested (have all the triangle the same orientation).


%-----------------------------------------------------------------------
\subsection{MeshCloudClip}

\begin{verbatim}
 == Mandatory unnamed args : ==
  * string [FDP,Cloud] :: Name of input cloud/mesh
  * string [3DReg] :: Name of 3D masq

 == Optional named args : ==
  * [Name=Out] string :: Name of output file
  * [Name=Bin] bool :: Generate out in binary format ,[Default=false]
\end{verbatim}


The mesh generated by Poisson method generate a smooth extension of the surface
that can be problematic, for example in the surface devlopment process. This
commande allow to clip the mesh using a $3d$ region.  For now the region
must come in the {\tt MicMac-V1} format as seized by {\tt mm3d SaisieMasqQT}
as for now there is no well established format for $3d$ volumes.
Figure~\ref{fig:MeshClip} illustrate the result of {\tt MeshCloudClip}

\begin{figure}
\centering
\includegraphics[width=6cm]{Programmer/ImagesProg/MSkirt00.jpg}
\includegraphics[width=6cm]{Programmer/ImagesProg/MClip00.jpg}
\caption{Letft mesh with undesirable extension, right mesh after clipping}
\label{fig:MeshClip}
\end{figure}



%-----------------------------------------------------------------------
%-----------------------------------------------------------------------
%-----------------------------------------------------------------------

\section{Mesh development}

This section describe the tools that were devloped for surface devlopment.

%-----------------------------------------------------------------------
\subsection{MeshDevGen}
This "tiny" tool was made to check the correctness of the surface devlopment algorithm. 
It generate a synthetic mesh, that is completely devlopable and for with we
know the 2 devlopment. This surface is a $3$ extrusion of a logarithmic spiral.


\COM{
MeshDev => Generate a planar devlopment minimizing deformations
MeshImageDevlp =>  Compute devlopped images from 3d-mesh, 2d-dev-mesh and ori
MeshProjImage => (internal) Project a mes on an image to prepare devlopment

}




