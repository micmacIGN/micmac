%-------------------------------------------------------------------------------
% IPOL LaTeX class manual
% by rafael grompone von gioi
% ver 0.5 - July 1, 2014
%-------------------------------------------------------------------------------
\documentclass{ipol}
\usepackage{mathrsfs} 
\usepackage{xfrac}
\usepackage{amsmath} 
\usepackage{amsfonts}
\usepackage{rotating} 
\usepackage{fancyvrb} 
%%for subfigure using ?
\usepackage{graphicx}
\usepackage{caption}
\usepackage{subcaption}
\usepackage{makeidx}
\usepackage{authblk}
\usepackage{algorithm}
\usepackage{algorithmic}  
\usepackage{amsthm}
\usepackage[normalem]{ulem}

\ipolSetTitle{Epipolar rectification of a generic camera}
\ipolSetAuthors{Marc Pierrot Deseilligny\ipolAuthorMark{1},
                Ewelina Rupnik\ipolAuthorMark{1}}
\ipolSetAffiliations{%
\ipolAuthorMark{1} LASTIG, Univ Gustave Eiffel, ENSG, IGN, F-94160 Saint-Mande, France\\
                   (\texttt{marc.pierrot-deseilligny@ensg.eu}, \texttt{ewelina.rupnik@ign.fr})
%\\\ipolAuthorMark{2} IIE, UdelaR, Uruguay (\texttt{jirafa@fing.edu.uy})
}

%---------------------------------------------
\newcommand{\CPP}{\mbox{\tt C\hspace{-0.05cm}\raisebox{0.2ex}{\small ++} }}
\newcommand{\SiftPP}{\mbox{\tt Sift\hspace{-0.05cm}\raisebox{0.2ex}{\small ++} }}



\newcommand{\RR}{\ensuremath{\mathbb{R}}}
\newcommand{\COM}[1]{}
\newcommand{\LA}{ LLLLLLLLLLLLLLLLLLLLLLLLLLLLLLLLLLLLLLLLLLLLLLLLLLLLLL \\ \\ }
\newcommand{\LALA}{ \LA \LA \LA \LA \LA \LA \LA \LA \LA \LA \LA \LA \LA \LA \LA  }


%// \newcommand{\HComp}{{}^{\longleftrightarrow}_{\pi_1,\pi_2}}
\newcommand{\HComp}{\overset{\Longleftrightarrow}{\scriptscriptstyle \pi_1,\pi_2}}
\newcommand{\PiOT}[1]{\pi_1(\pi_2^{-1}(#1))}
\newcommand{\PiTO}[1]{\pi_2(\pi_1^{-1}(#1))}
\newcommand{\CurvO}{{\mathcal{T}_1}}

%  Bundles
\newcommand{\Bund}[1]{\ensuremath{\mathcal{B}_{#1}}}
\newcommand{\BundO}{\Bund{1}}
\newcommand{\BundT}{\Bund{2}}
\newcommand{\BundK}{\Bund{k}}


% Epipolar lines
\newcommand{\LineE}[1]{\ensuremath{\mathcal{L}_{#1}}}
\newcommand{\LineO}{\LineE{1}}
\newcommand{\LineT}{\LineE{2}}
\newcommand{\LineK}{\LineE{k}}

\newcommand{\CurveE}[1]{\ensuremath{\mathcal{C}_{#1}}}
\newcommand{\CurveO}{\CurveE{1}}
\newcommand{\CurveT}{\CurveE{2}}
\newcommand{\CurveK}{\CurveE{k}}

% "Epipolar Surface"
\newcommand{\Sv}{\ensuremath{\mathcal{S}_{v}}}

\newcommand{\BigV}[1]{\ensuremath{\overrightarrow{#1}}}
\newcommand{\TanO}[1]{\BigV{t_1#1}}
\newcommand{\TanT}[1]{\BigV{t_2#1}}

\newcommand{\Negl}[1]{\ensuremath{\mathcal{O}(#1)}}



\newcommand{\PiVert}{\widetilde{\pi}}
\newcommand{\PiZVert}{\widetilde{\pi}_1^{\mathcal{Z}} }
%\newcommand{\PiOT}{\Pi^{12}}
%\newcommand{\PiTO}{\Pi_{2\rightarrow 1}}

\newcommand{\DerPart}[2]{\frac{\partial #1}{\partial #2}}

\newtheorem{theorem}{Theorem}
\newtheorem{notation}{Notation}
\newtheorem{definition}{Definition}
\newtheorem{remark}{Remark}

\definecolor{orange}{rgb}{1,0.5,0} 
\newcommand{\er}[1]{\textcolor{orange}{#1}} 
\definecolor{magenta}{rgb}{1,0,1} 
\newcommand{\mpd}[1]{\textcolor{magenta}{#1}} 
\definecolor{blue}{rgb}{0,0,1} 
\newcommand{\ermpdok}[1]{{\bf \textcolor{blue}{#1}} }

\definecolor{black}{rgb}{0,0,0} 
\newcommand{\BEGINCHANGE} {\textcolor{magenta}{ ---------------------------------------BEGIN-BEGIN-BEGIN}}
\newcommand{\ENDCHANGE}{\textcolor{magenta} {END-END-END------------------------------------------------}}


%-------------------------------------------------------------------------------
\begin{document}

%-------------------------------------------------------------------------------
\begin{ipolAbstract}
ble
\end{ipolAbstract}

%-------------------------------------------------------------------------------
\ipolKeywords{epipolar rectification, generic camera, pushbroom sensor, central perspective}

%-------------------------------------------------------------------------------
TODO :
\begin{enumerate}
   \item  décrire méthode de calcul automatique des directions dans le cas sans modèle;

   \item Allégation :  marche aussi sans modèle, minimise globalement le critère d'erreur, 
         formulation théorique de l'ambiguité et des condition d'existence d'un crière épipolaire;
         mesure de "l'épipolarabilité" d'un couple;
 
\end{enumerate}

\section{Introduction}
The epipolar geometry of images plays a central role in many applications in the field of photogrammetry and computer vision. In the stereo-reconstruction pipeline, it is used twice:

\begin{enumerate}
   \item In the camera pose orientation step, when computing the
      relative orientation of a pair of images from their corresponding points. Assuming 
      the projection follows the central perspective and the internal calibration is known,
      one can compute the epipolar geometry of the images using the essential matrix. Finally, the relative orientation is recovered \cite{fusiello2000epi}.   
     
   \item In the image dense matching step, where the epipolar rectification simplifies the 
       correspondence search because for any point $(x_1,y)$ in image $I_1$, its correspondence is some point $(x_2,y)$ in image $I_2$. Therefore, finding correspondences across images reduces to 
        a $1$-dimensional problem ($1$D).
\end{enumerate}

\noindent In this paper, we only study  the epipolar rectification problem and more specifically
its application to a generic camera model.%, focusing on the pushbroom sensors.

\subsection{Related works}
Rectifying a central perspective camera stereo pair involves transforming their original epipolar geometry to a canonical form where: (a) their focal planes are coplanar, and (b) their conjugate epipolar lines are colinear, and parallel to the camera's x-axis. 
%It is equivalent of sending the stereo pair's epipoles at infinity. 
From the algebraic standpoint, this is equivalent to applying two $2$-dimensional ($2$D) projective transformations to both images of the stereo pair. Several approaches to computing such transformations have been proposed over the course of the last 30 years.

For a calibrated stereo pair (i.e. with known camera projection matrices), there exists a unique rectifying transformation, up to a rotation along the baseline~\cite{fusiello2000epi}. In an uncalibrated case, the solution is obtained by factoring out two 2D homographies from the fundamental matrix. Because there are no two unique homographies, the common practice is to parametrize these transformations such that the distortions caused by the rectification process are minimized. For instance, Loop and Zhang~\cite{loop1999epi} decompose the rectifying homographies to a combination of the projective, similarity and shearing transforms, with the condition that the prrojective transform remains (close to) affine. Hartley~\cite{hartley1999epi} satisfies the condition that for a neighborhood of a point (e.g. the center of an image), the computed homography is a rigid transformation. Building on this work, Isgro and Trucco's~\cite{isgro1999epi} approach obtains a unique solution by minimizing the x-disparity without having to explicitly calculate the fundamental matrix. Instead of minimizing the disparity in the first coordinate, Wu and Yu~\cite{wu2005epi} recycle an idea first introduced by Hartley~\cite{hartley1999epi} which requires that the aspect ratio of the images before and after rectification is constant. More recently, Fusiello and Irsara~\cite{fusiello2008epi} introduced the camera matrices back into the equation and proposed a \textit{quasi}-Euclidean approach for uncalibrated cameras, similar to that of the calibrated cameras case. Their projective transformations are parametrized by five angles and a focal length. Monasse \textit{et al.}~\cite{monasse2010epi} break down the one-time rotation of~\cite{fusiello2008epi} to a three step procedure, and prove increased robustness by using a geometric error measure (i.e., camera rotation angle) to reduce the rectifying error distortions.

%force the plane at infinity to be close to the plane at infinity of a calibrated stereo pair. This is possible by making an educated guess and estimating only the focal length.


 
%Epipolar recitification algorithms for central projection cameras can be classified into calibrated and uncalibrated camera approches.


%For central projection cameras, the epipolar recitification algorithms are grouped into the calibrated and uncalibrated camera case. Rectifying calibrated cameras comes down to fixing their perspective center and rotating the cameras until their focal planes are coplanar (Fusiello compact algo). Uncalibrated camera rectification involve additional parameters 

  
%
Unlike the central projection camera model, pushbroom-like sensors acquire each image row from a different perspective center. As a consequence, the epipolar lines are neither straight lines, nor are they conjugate across the image~\cite{Gupta1997}.  One way to overcome this particularity is to simplify the projection function with a 2D affine~\cite{ono1999epipolar,wang2011epipolar} or a parallel projection model~\cite{morgan2006epipolar}. Such approximations usually come at the price of precision, especially with the increasing camera field-of-view or in mountainous scenes. By extending the 2D affine model with two quaratic terms, Okamonoto \textit{et al.}~\cite{okamoto:99:DLT} demonstrates improved performance on SPOT images. In the context of dense image matching, de~Francis \textit{et al.}~\cite{deFrancis2014stereo} improves the precision by partitioning the images into small patches, for which indepedent affine rectifictions are computed. Alternatively, and with equally good precision, Oh~\cite{Oh2011} uses the \textit{Rational Polynomial Coefficients} (RPCs) to map the epipolar curves across the full size images with lines, in a piecewise approach, followed by a global rectification transformation using a polynomial function of 3$^{rd}$ order.
%
%ù while being locally approximated to straight lines. Oh~\cite{Oh2011} approximates the epipolar curves of an image pair by piecewise linear curves with a set of virtual correspondences. An image pair is then resampled to an equivalent geometry, imposing that the conjugate curves become straight and aligned on the y-axis. 


%uses the \textit{Rational polynomial coefficients} (RPCs) to trace the epipolar curves across image pairs with a dense set of virtual correspondences, and by doing so approximates the curves by a piecewise linear curve. Given a set of correspondences, an image pair is then resampled to an equivalent geometry, imposing that the conjugate curves become straight and aligned on the y-axis. 

%This work is similar to the approch proposed by Oh~\cite{Oh2011} in that it tries to model the epipolar curves across the entire image at once.
 
%  \cite{wang2011epipolar}

%The x-parallax remains linearly proportional to the depth of corresponding points - yes because we don(t move the x-coordinates

 
%\cite{orfeo2008}

%Carlo de Franchis , split the images in small patch, in each patch compute a coniq camera that approximate the model for each pair, make an epipolar resampling of this approximate coniq camera.  Advantage : simple an reuse existing method. Drawback : need small patch to be accurate enough.

\subsection{Contributions}
Our research work proposes an epipolar geometry rectification method that is not tied to any camera physical model. %Thus, our method can be applied to images taken with the central projection camera model and the pushbroom sensor. 
We demonstrate that on a range of models, including Pleiades images (pushbroom), the Corona images (panoramic pushbroom), a consumer grade camera images (central perspective) as well as Sentinel-1 radar images (pushbroom).
The method resembles Oh's~\cite{Oh2011} approach in that it exploits the point correspondences to find the polynomial mapping to epipolar geometry. However, unlike the work of Oh~\cite{Oh2011}, we do not require that the camera geometric model is known. We demonstrate that in some circumstances, point correspondences obtained from an image processing routine, e.g. SIFT~\cite{lowe2004distinctive}, can serve to find the epipolar resampling. 

In the remainder of this publication, we first outlay the mathematical background of the epipolar geometry and identify cases when epipolar geometry does not exist or is ambiguous  (Section~\ref{sec:math}). Then, we introduce our method (Section~\ref{sec:method}), and finally present the experiments on different datasets with and without the geometric model (Section~\ref{sec:experiments}). The results are compared against the Oh's~\cite{Oh2011} method with respect to the y-parallax remaining after the resampling to the epipolar geometry.

 %\subsection{Topic of the paper}



\section{The mathematics of epipolar geometry in the generic case}\label{sec:math}

\subsection{Formalisation and notation of projections}

\begin{figure}[htb]
\centering
\begin{tabular}{c}
% \hline 
 \\
\includegraphics[width=6cm]{FIGS/NotaProj.png} 
 \\ 
  \\%\hline 
\end{tabular}
\caption{A projection and a bundle.}
\label{FigNotaProj}
\end{figure}



%\includegraphics[width=12cm]{FIGS/NotaSets.jpg}

We define the geometric sensor model of an image by a projection function $\pi$, that computes, for a given 3D
point, its 2D projection in the image:

\begin{definition}[Generic geometric sensor model]  

\emph{Illustrated in Figure~\ref{FigNotaProj}.}

A geometric sensor model $\pi$ is a $\mathcal{C}^{\infty}$ mapping from ground space $(\RR^3)$ to image space $(\RR^2)$:

\begin{equation}
  \pi :  \RR^3  \rightarrow \RR^2  ,  (X,Y,Z)  \rightarrow (i,j) = \pi(X,Y,Z). \label{Eq:Proj}
\end{equation}
\end{definition}

\noindent Next, we define the bundles of a projection:

\begin{definition}[Bundle]
For $p_k \in I_k$ we note $\BundK(p_k)$  the bundle corresponding 
to $\pi_k^{-1}(p_k)$. When there is no ambiguity,
we note identically  $\BundK(P)$, where $P\in \RR^3$,
 the bundle corresponding to $\pi_k^{-1}(\pi_k(P)) = \BundK(\pi_k(P))$.
\end{definition}


\noindent Later, for simplicity, we will use the  {quasi-vertical} hypothesis, which allows us to extend $\pi$ to a
bijective mapping of $\RR^3$ and compute its inverse.

\begin{definition}[Quasi-vertical camera model]  
We say that the projection is quasi-vertical if the following mapping $\PiVert$ is a diffeomophism of $\RR^3$:
\begin{equation}
  \PiVert :  \RR^3  \rightarrow \RR^3  ,  (X,Y,Z)  \rightarrow (i,j,Z) = \PiVert(X,Y,Z)  , with (i,j) = \pi(X,Y,Z). \label{PiInvert}
\end{equation}
\end{definition}
%
Given $2$ images $I_1$ and $I_2$, the knowledge  of their geometric models $\pi_1$ and  $\pi_2$
reduces the matching between $2$ images to a 1D problem. In fact, given a point $p_1$ in $I_1$,
we can compute the 3D curve $\BundO(p_1)$ of  ground points that project to $p_1$ in $I_1$, and compute its
homologous curve in $I_2$ with $\pi_2(\BundO(p_1))$. We now define the H-compatible relation between two points
by the following definition:

\begin{definition}[H-Compatible, $\HComp$] 
\emph{Illustrated in Figure~\ref{FigNotaComp}.}

We say that $p_1$ in  $I_1$ and $p_2$ in $I_2$ are  $\pi_1-\pi_2$ H-compatible, and write $p_1 \HComp p_2$, if   the following condition is satisfied:

\begin{equation}
   ( \BundO(p_1) \cap  \BundT(p_2) \neq \emptyset    )
    \Leftrightarrow
  (\exists P \in  \RR^3 : \pi_1(P) =p_1 ,  \pi_2(P) = p_2).
\end{equation}
\end{definition}

\begin{figure}[htb!]
\centering
\begin{tabular}{c}
 %\hline \\
\includegraphics[width=9cm]{FIGS/NotaBundle.png} 
 %\\ \hline
\end{tabular}
\caption{Illustration of $\HComp$}
\label{FigNotaComp}
\end{figure}

\noindent In image matching, the relationship $p_1 \HComp p_2$  means that $p_1$ and $p_2$ are potentially homologous.

%---------------------------------------------
%---------------------------------------------
%---------------------------------------------



% - - - - - - - - - - - - - - -
\subsection{Definition of the epipolar geometry}

In fact, the previous relationships are sufficient to implement all the matching techniques and
$(\pi_1,\pi_2)$ can be used to define a matching process, taking advantage
of the~\emph{a priori} knowledge of the scene geometry. That is, given a point in one image, we can easily
follow its curve of potentially homologous points in the other image.
This technique, which does not exploit the epipolar geometry, has the advantage of also being adaptable to multi-image matching. Epipolar geometry is therefore not strictly required for the image matching process.

The drawback of this approach is that it combines two different problems in the same procedure: the handling of the geometry and resampling and
the matching process. When one is interested
in the matching of a single image pair, the epipolar geometry
can provide an elegant solution by separating the problem in two independent ones. 


\begin{definition}[Epipolar Geometry]
\emph{Illustrated in Figures~\ref{FigDefEpip} and~\ref{FigAmbigEpip}.}

Let $\pi_1,\pi_2$ be two cameras and let $\phi_1,\phi_2$  be two diffeomorphisms
of $\RR^2$. We say that $\phi_1,\phi_2$ are epipolar resamplings iff:

\begin{equation}
  \forall e_1=(u_1,v_1) , e_2=(u_2,v_2) : (v_1=v_2)   \Leftrightarrow  (\phi_1^{-1}(e_1) \HComp \phi_2^{-1}(e_2)).
\end{equation}
   \label{EqEpiEgalY}
\end{definition}

\begin{figure}
\centering
\begin{tabular}{c}
 %\hline \\
\includegraphics[width=9cm]{FIGS/Epip.png} 
 %\\ \hline 
\end{tabular}
\caption{Illustration of epipolar geometry.}
\label{FigDefEpip}
\end{figure}


\noindent The matching of epipolar images is simplified because we know that the lines in two images are globally homologous. 
%\er{The space of the image rectified to epipolar geometry is denoted by $E$ (see Figure~\ref{FigDefEpip})}


\begin{notation}[Epipolar line and curve.]
We denote $\LineK(v)$  as the epipolar  line of $E_k$ defined by $v_k=v$. We also denote $\CurveK(v)$ as the epipolar
curve of $I_k$ defined by $\CurveK(v) = \phi_k^{-1}(\LineK(v))$
\end{notation}
%
\noindent We can see that when epipolar geometry exists, the two curves $\CurveO(v)$ and $\CurveT(v)$ are globally homologous:

\begin{equation}
     \CurveO(v) = \PiOT{C_2(v)}   ;  \CurveT(v) = \PiTO{C_1(v)}.\label{Eq:CurvHom}
\end{equation}


%---------------------------------------------

%<<<<<<< HEAD
\subsection{Existence of the epipolar geometry}\label{ExistEpip} 

We now discuss the existence of the {epipolar geometry}. As you will see, the epipolar geometry generally does not exist, and when it does, it is not
unique. It is well known that:

\begin{itemize}
    \item for any image pair following the {central projection}, there exists  an epipolar geometry;

    \item  not all image pairs can be resampled to epipolar geometry. For example, a cylindrical projection, applicable to many push-broom satellites, generally does not allow for epipolar resampling.

\end{itemize}
We explain now why the epipolar geometry does not exist for any $\pi_1,\pi_2$ and is instead an exception. Let's define the surface $\Sv^k$ of $\RR^3$ by:

\begin{equation}
   \Sv^k = \pi_k^{-1}(\CurveK(v)).  \label{Svk}
\end{equation}
By the definition of {epipolar geometry} above, it can be seen that $\Sv^1$ and $\Sv^2$ are the same surface $\Sv$:

\begin{equation}
   \Sv^1 = \Sv^2 = \Sv.
\end{equation}
For any $P\in\Sv^1$, set  $e_1=\phi_1(\pi_1(P))=(u_1,v)$
and $e_2=\phi_2(\pi_2(P))=(u_2,{v_2})$. We then have $\pi_1(P) \HComp \pi_2(P)$ because they are projections
of the same point. Then, $v_2=v$ according to Definition~\ref{EqEpiEgalY} and $P \in \Sv^2$. Furthermore, the $\Sv$ defines a foliation of $\RR^3$, and it can be seen that:

\begin{equation}
   \forall v \forall P \in \Sv :  \BundO(P) \subset \Sv , \BundT \subset \Sv, \label{ClothureBundle}
\end{equation}
which also leads directly from the definitions above. If $P \in \Sv$ then $\pi_k^{-1}(P) \in \CurveK(v)$
(see equation~\eqref{Svk}), then $\pi_k^{-1}(\pi_k(P)) = \BundK(P) \subset \Sv$.
%
However, in general, the existence of a stable foliation for the two bundle sets, as expressed in Equation~\eqref{ClothureBundle},
cannot  be satisfied, as illustrated 
in Figure~\ref{FigClothPath}.  To explain further, let $\pi_1$ and $\pi_2$ be any two  projections and suppose
there exists  a foliation satisfying the Equation~\eqref{ClothureBundle}. Then, let: 

\begin{itemize}
   \item   $P$ be any point in $3$D space, and $\Sv$ be the surface such that $P \in \Sv$;
   \item   $P_1 \neq P$ be a point on $\BundO(P)$, $P_1 \in \Sv$, then $\BundT(P_1) \subset \Sv$;
   \item   $P_2 \neq P$ be a point on $\BundT(P)$, $P_2 \in S_v$, then $\BundO(P_2) \subset \Sv$.
\end{itemize}
As $\BundT(P_1)$ and $\BundO(P_1)$ are included in the same surface $\Sv$, they
            must intersect somewhere in a point $Q$.
In the general case, the above is a contradiction because  there is no reason that the condition $\BundT(P_1) \cap \BundO(P_2) \neq \emptyset $ is satisfied for any two sets
of bundles (see Figure~\ref{FigClothPath},right).

\begin{figure}[h!]
\centering
\begin{tabular}{cc}
%\hline
%& \\
\includegraphics[width=7cm]{FIGS/ClothPathEpip.png} &
\includegraphics[width=7cm]{FIGS/ClothPathNonEpip.png}\\
% & \\\hline
\end{tabular}
\caption{Path closure. Left: in the epipolar case, the bundles are on the same level of the foliation and the intersection.
Right: in the generic case, the paths don't intersect and no epipolar geometry exists.}
\label{FigClothPath}
\end{figure}

%---------------------------------------------


\subsection{A local characterization of the epipolar existence}

In this section, we compute a local formula (i.e. a differential equation) that provides conditions for
the existence of an epipolar geometry. This section is rather theoretical and
can be omitted by readers mainly interested in practical applications.

\begin{figure}
\centering
\begin{tabular}{c}
% \hline \\
\includegraphics[width=7cm]{FIGS/EquadifEpip.png}
% \\ \\ \hline
\end{tabular}
\caption{Notation for local characterization of the epipolar existence}
\label{EqDifEpip}
\end{figure}

Analogously to the proof in Section~\ref{ExistEpip}, we will make a  computation of two-way paths, $\BundO$ then $\BundT$,
as well as $\BundT$ then   $\BundO$. Then, we express the Taylor expansion
of the intersection distance between these two paths  
. Let's consider the following (see Figure~\ref{EqDifEpip}):
\begin{itemize}
   \item we make the quasi-vertical assumption (note that we could use
          curvilinear abscissa   when this assumption can not  be satisfied);
   \item let $P$ be any point in $\RR^3$;
   \item consider the first path $(P,P_1,Q_1)$ following  $\BundO$ then $\BundT$, making a
         progression $d_{z_1}$ on $\BundO$ and  $d_{z_2}$ on $\BundT$ ;
   \item consider a second path $(P,P_2,Q_2)$ following  $\BundT$ then $\BundO$, making a
         progression $d_{z'_2}$ on $\BundO$ and  $d_{z'_1}$ on $\BundT$ ;
   \item we denote $\overrightarrow{t_1(P)}=(x,y,1)$ as the tangent to the bundle  $\BundO$  in point $P$
         (and similarly $\overrightarrow{t_2}(P)$);
   \item when we write  $\DerPart{F}{z_1}$, we will refer to the coordinate system $(i_1,j_1,z) = \PiVert_1^{-1}(x,y,z)$,
         idem for  $\DerPart{F}{z_2}$, and obviously as they are two different coordinate systems, we have in general
        $\DerPart{F}{z_1}  \neq \DerPart{F}{z_2}$.
\end{itemize}

Now, for any pair of "small" values $(\delta_1,\delta_2)$,  we  compute 
$(\delta'_1,\delta'_2)$ which minimize the distance $|Q_1,Q_2|$ and express the canceling of the
second degree Taylor expansion of this distance (the first degree can always be canceled out as we will see).  Noting $\delta$ the max of all $\delta$, the second degree Taylor expansion gives :

\begin{equation}
    P_1 =  P +  \delta_1 \TanO{(P)} + \frac{{\delta_1}^2}{2} \DerPart { \TanO{}}{z_1}(P)  + \Negl{\delta^3} \label{P1} 
\end{equation}
\begin{equation}
    Q_1 =  P_1 +  \delta_2 \TanT{(P_1)} + \frac{{\delta_2}^2}{2} \DerPart { \TanT{}}{z_2}(P_1) + \Negl{\delta^3} \label{Q1}
\end{equation}
\begin{equation}
     \TanT{(P_1)} = \TanT{(P)} +  \delta_1   \DerPart { \TanT{}}{z_1}(P) + \Negl{\delta^2} \label{TP1}
\end{equation}
%
Putting together Equations~\eqref{P1},~\eqref{Q1},~\eqref{TP1} we can perform a Taylor expansion of the path
$P$ to $Q_1$:
\begin{equation}
    Q_1 =    P +  \delta_1 \TanO{(P)} 
               +  \delta_2 \TanT{(P)} 
               + \frac{{\delta_1}^2}{2} \DerPart { \TanO{}}{z_1}(P) 
               + \frac{{\delta_2}^2}{2} \DerPart { \TanT{}}{z_2}(P) 
               +  \delta_1  \delta_2  \DerPart { \TanT{}}{z_1}(P)  
               + \Negl{\delta^3}
       \label{Q1ofP}
\end{equation}
%
And similarly for $P$ to $Q_2$ :
%
\begin{equation}
    Q_2 =    P +  \delta'_2 \TanT{(P)} 
               +  \delta'_1 \TanO{(P)} 
               + \frac{{\delta'_2}^2}{2} \DerPart { \TanT{}}{z_2}(P) 
               + \frac{{\delta'_1}^2}{2} \DerPart { \TanO{}}{z_1}(P) 
               +  \delta'_1  \delta'_2  \DerPart { \TanO{}}{z_2}(P)  
               + \Negl{\delta^3}
       \label{Q2ofP}
\end{equation}
%
The first degree Taylor expansion of $Q_2-Q_1$ gives :
%
\begin{equation}
    Q_2 -Q_1 =   (\delta'_1 -\delta_1) \TanO{(P)} +  (\delta'_2 -\delta_2) \TanT{(P)}  + \Negl{\delta^2}
\end{equation}
%
To minimize $|Q_2 -Q_1|$, the first step is to cancel the first degree  terms of $ Q_2 -Q_1$. We assume
that $\TanT{(P)}$ and $\TanO{(P)}$  are independant vectors\footnote{Otherwise, it would be a 
degenerate case for stereovision} and we must then make $\delta'_2 -\delta_2$ and $\delta'_1 -\delta_1$ second degree terms:

\begin{equation}
   \Delta_1 =   \delta'_1 -\delta_1 = \Negl{\delta^2}  \; ; \; \Delta_2 =   \delta'_2 -\delta_2 = \Negl{\delta^2}
   \label{Delta}
\end{equation}
%
To develop $ Q_2 -Q_1$ we can use the following identities  that are direct consequences of~Equation~\eqref{Delta}:

\begin{equation}
   \delta_1  \delta_2 -  \delta'_1  \delta'_2  = \Negl{\delta^3} \;;\;
   {\delta_1}^2 - {\delta'_1}^2 =  \Negl{\delta^3} \;;\;
   {\delta_2}^2 - {\delta'_2}^2 =  \Negl{\delta^3} 
   \label{NeglDelta}
\end{equation}
%
Subtracting Equation~\eqref{Q1ofP} from Equation~\eqref{Q2ofP}, and using Equation~\eqref{NeglDelta}, we can write :

\begin{equation}
    Q_2 -Q_1 =   \Delta_1 \TanO{(P)} +   \Delta_2 \TanT{(P)}  
               + \delta_1  \delta_2(\DerPart { \TanT{}}{z_1}(P)  -\DerPart { \TanO{}}{z_2}(P) )
               + \Negl{\delta^3}
\end{equation}
%
We now translate the intersection of paths by canceling the second degree Taylor expansion in $Q_2 -Q_1$. We have three vectors, and their weighted sum can be null iff they are colinear.

\begin{theorem}[Existence of epipolar]
The epipolar geometry exists iff the following determinant is null:

\begin{equation}
\left[ \begin{array}{c|c|c}
\TanO{} & \TanT{}  & \DerPart { \TanT{}}{z_1}  -\DerPart { \TanO{}}{z_2}  
\end{array} \right]  
=0
\end{equation}

\end{theorem}

\begin{remark}[Epipolar equation with central perspective camera]
As an illustration in an easy case, we can see that this condition is trivially 
satisfied for a pair of central perspective cameras as we have the canceling of both terms as shown
in Equation~\eqref{EqEpipConik}. This is because for a given point $P$,
for any point $P_1$ on $\BundO(P)$, $\TanT{P_1}$ belongs to the epipolar
plane $\mathcal{P}$. We have $\TanT{P_1} \in \mathcal{P}$, so $\DerPart { \TanT{}}{z_1} \in \mathcal{P}$,
and as we have also $\TanO{(P)} \in  \mathcal{P}, \TanT{(P)} \in  \mathcal{P}$, the collinearity
between $\TanO{(P)}$ , $\TanT{(P)}$ and $\DerPart { \TanT{}}{z_1}(P)$ is thus proven.

\begin{equation}
\left[ \begin{array}{c|c|c}
\TanO{} & \TanT{}  & \DerPart { \TanT{}}{z_1}  
\end{array} \right]  
=\left[ \begin{array}{c|c|c}
\TanO{} & \TanT{}  & \DerPart { \TanO{}}{z_2}  
\end{array} \right]  
=0
\label{EqEpipConik}
\end{equation}
\end{remark}


%---------------------------------------------


\subsection{Ambiguity of the epipolar geometry}

When the epipolar geometry exists, the epipolar resampling is not unique. To demonstrate that our rectification method handles this ambiguity rigorously, we first describe it formally.


Let $\phi_1,\phi_2$ and  $\phi'_1,\phi'_2$ be two   epipolar resamplings, 
then, for any $v$ consider the pair of lines $\LineO(v),\LineT(v)$:

\begin{itemize}
   \item  $\phi_k^{-1}(\LineK(v))$ is the curve $\CurveK(v)$ by definition of epipolar resampling;
   \item  and $\phi'_k (\CurveK(v) = \phi'_k ( \phi_k^{-1}(\LineK(v)))$ is a line, also by definition of epipolar resampling;
   \item and still by definition $\phi'_1 ( \phi_1^{-1}(\LineO(v))) = \phi'_2 ( \phi_2^{-1}(\LineT(v)))$;
\end{itemize}

Consequently we have the following constraint between two pairs of epipolar ressampling:

\begin{itemize}
   \item  $\phi'_1 \phi_1^{-1}$  and $\phi'_2 \phi_2^{-1}$ are diffeomorphisms transforming lines into lines;
   \item $\phi'_1 \phi_1^{-1}$  and $\phi'_2 \phi_2^{-1}$ define the same global transformation on lines
        (i.e. if $\phi'_1 (\phi_1^{-1} (\LineO(v))) = \phi'_2 (i\phi_2^{-1}(\LineT(v)))$).
\end{itemize}


%For any pair of lines $\LineK(v)$ in $E_k$, there are corresponding 
%homologous curves $\CurveO(v),\CurveT(v)$  in images 
%$I_1,I_2$ (see Equation~\eqref{Eq:CurvHom}). Following the depiction in Figure~\ref{FigAmbigEpip}, if $\phi'_k(\CurveK(v)))$
%are  lines $\LineK(v')$ in $E_k$, one can observe that $\phi'_1 \phi_1^{-1}$  and $\phi'_2 \phi_2^{-1}$ are diffeomorphisms transforming lines into lines, and

% they globally aoperate the same transformation on lines
\emph{Vice versa}, let  $\phi_1,\phi_2$ be an epipolar resampling and let $\Lambda_1,\Lambda_2$ 
be diffeomorphisms  that are stable for lines and make globally the same transformation on lines. We can thus note that $\Lambda_1 \circ \phi_1$ and  $\Lambda_2 \circ \phi_2$ are also an epipolar resampling.


Having devised the exact ambiguity, we can now define two constraints to impose on a unique epipolar resampling: 
\begin{enumerate}
\item Constraint on the uniqueness of the deformation
inside each line . For instance, one can impose that the colums remain constant (i.e.
the deformation is only made  on  $y$), as given in Equations~\eqref{Ambig:PhiO}
and~\eqref{Ambig:PhiT};
\item Constraint on the global deformation of lines\footnote{i.e. where each line
is transformed globally to another line}. For instance, by fixing the transformation of one image, as given in Equation~\eqref{FigAmbigEpip}.
\end{enumerate}
 

\begin{figure}
\centering
\begin{tabular}{c}
% \hline \hline
\includegraphics[width=10cm]{FIGS/AmbigEpip.png}
% \\ \hline \hline
\end{tabular}
\caption{Ambiguity of the epipolar geometry: two possible epipolar resamplings for a single stereopair.}
\label{FigAmbigEpip}
\end{figure}




\begin{theorem}[Unique epipolar constraint]

If the epipolar geometry exists, there exists a unique epipolar resampling $\phi_1,\phi_2$ satisfying the  following three constraints:

\begin{equation}
    \phi_1(x,y) = (x,y') \label{Ambig:PhiO}
\end{equation}
\begin{equation}
    \phi_2(x,y) = (x,y') \label{Ambig:PhiT}
\end{equation}
\begin{equation}
    \phi_1(0,y) = (0,y) \label{Ambig:Line}
\end{equation}
\label{Theo:Fix:Ambig}

\end{theorem}



%---------------------------------------------
%---------------------------------------------
%---------------------------------------------

\section{Proposed method for epipolar geometry resampling}\label{sec:method}


\subsection{Hypothesis and layout}

\subsubsection{Principles}
The principle of the method is to use \emph{H-Compatible}  points $p_1,p_2$ to calculate a
pair of functions $\phi_1,\phi_2$ that comply with the epipolar constraint, i.e. 
\emph{"$\phi_1(p_1)$ and $\phi_2(p_2)$ are on the same line"}. As these epipolar functions
are not unique, we parameterize the $\phi_k$ in Theorem~\ref{Theo:Fix:Ambig} accordingly:
%
\begin{equation}
    \phi_k(i,j) = (i,V_k(i,j)); \; \;
    V_k : \RR^2 \rightarrow \RR  
  \label{EpipVParam}
\end{equation}
%
This parametrization implements the constraints of Equations~\eqref{Ambig:PhiO} and~\eqref{Ambig:PhiT}. We will account for the constraint of  Equation~\eqref{Ambig:Line} in Section~\ref{ChoicePolyn}\footnote{See Equations~\eqref{CstrV1:0},~\eqref{CstrV1:1}.}. 
To compute $V_1,V_2$, for any pair of \emph{H-Compatible} points, we add an observation that constrains $V_1$ and $V_2$:


\begin{equation}
    V_1(p_1) = V_2(p_2) \label{EqV1V2}.
\end{equation}

\subsubsection{Hypothesis}


The method takes two camera models $\pi_1$ and $\pi_2$ as inputs.
These models are considered black-boxes that satisfy Equation~\eqref{Eq:Proj}, and for which no specific assumption is made on the physical model of the camera. In our \CPP implementation,
the cameras are considered to be pure virtual classes offering the interface to Equation~\eqref{Eq:Proj}.
In this paper, the examples processed by our method are pushbroom satellite models known by their RPCs, the central perspective and radar models. However, the only restriction imposed on the generic nature of the model is that the projection function is "smooth", i.e.:

\begin{itemize}
    \item $\pi$ are $\mathcal{C}^{\infty}$ functions, and
    \item the directions of epipolar curves vary within a limited range (for example,
          less than $\frac{\pi}{2}$).
\end{itemize}

\begin{figure}
\centering
\begin{tabular}{c}
% \hline \hline
\includegraphics[width=10cm]{FIGS/BadGoodLines.png}
% \\ \hline \hline
\end{tabular}
\caption{Left: a set of epipolar lines not handled by our method. Right: a perfectly acceptable pair  of epipolar lines.}
\label{BadGoodEpip}
\end{figure}
%
Figure~\ref{BadGoodEpip} illustrates the latter constraint. The left image presents a set
of epipolar lines with too large direction variations. The right image represents a pair of epipolar lines whose directions change within a small range, therefore suitable for the proposed resampling method. 

\subsubsection{Estimation of the center and the global direction}

\label{EstCenterDir}

To begin with, the method estimates the centers  $C_1,C_2$ of a set of points $p_1$ and $p_2$. This is done by calculating the average of all points' coordinates.  Then, the computations continue in
the coordinate systems centered at $C_1,C_2$. After this "normalisation", the constraint in Equation~\ref{Ambig:Line} is applied at these centers.

\BEGINCHANGE{}

\begin{figure}
\centering
\begin{tabular}{c}
% \hline \hline
\includegraphics[width=10cm]{FIGS/EpipReqOrient.png}
% \\ \hline \hline
\end{tabular}
\caption{Left: quasi-vertical epipolar curve for which correction with Equation~\eqref{EpipVParam} is impossible.  Middle and right: oblique curves for which epipolar rectification  with Equation~\eqref{EpipVParam} is possible
         but generates significant distortion.}
\label{ReqOrient}
\end{figure}

Then we need to compute a coordinate system where epipolar lines are globaly horizontal. This requirement is
a consequence of equation~\eqref{EpipVParam}, and is illustrated by Figure~\ref{ReqOrient}:


\begin{itemize}
   \item left image of figure~\ref{ReqOrient} presents a case where epipolar curve are quasi vertical
         and for which an epipolar correction, without initial rotation,  
          according to equation~\eqref{EpipVParam} would be impossible;
   \item middle image of figure~\ref{ReqOrient} presents a case where epipolar curve are oblique,
         in this case epipolar correction according to equation~\eqref{EpipVParam} would be possible
         but would lead to important distorsion in the image, as can be seen on left image.
\end{itemize}


So for each image we estimate the average direction $\vec{D}_k$
of its epipolar lines, and using Equation~\eqref{EqRot}, a rotation $R_k$ is applied to the  \mpd{input point}. 

\begin{equation}
    R_k(p) =  \frac{p-C_k}{\vec{D}_k}  \label{EqRot}
\end{equation}

The epipolar lines are now globally horizontal and the subsequent epipolar deformation can be computed on the rotated data points. 

\ENDCHANGE{}

\subsubsection{Layout}

The layout of the method follows three steps: (1) estimate the global 
direction of epipolar lines; (2) estimate $F_1,F_2$ as  the local epipolar rectification
 in the coordinate system linked to the global direction; (3)  estimate the final
epipolar rectification as a composition of $F_1,F_2$ and the rotation.
A more formalized description of the algorithm is given in Algorithm~\ref{AlgoGlob}.


\begin{algorithm}[H]
\caption{Epipolar($\pi_1$,$\pi_2$). \emph{Layout of the algorithm for computing the epipolar rectification from camera models}}
\begin{algorithmic}
   % \STATE {\emph{Layout of the algorithm for computing the epipolar rectification from camera models}}
    \STATE Use $\pi_1,\pi_2$ to estimate a set of \emph{H-Compatible} points $\mathcal{H} =\{(p_1,p_2)\}$ : 
    \STATE Estimate centers $C_1$ and $C_2$;
    \STATE Estimate global direction of epipolars $\vec{D}_1$ and $\vec{D}_2$,
    \STATE Estimate rotations $R_1,R_2$ according to Equation~\eqref{EqRot}
    \FORALL{$p_1,p_2 \in \mathcal{H}$}
              \STATE set: $q_1 = R_1(p_1)$,  $q_2 = R_2(p_2)$
              \STATE add equation: $V_1(q_1) = V_2(q_2)$
    \ENDFOR
    \STATE estimate with the least squares method $V_1$ and $V_2$
    \STATE set $F_k(x,y)=(x,V_k(x,y))$  %, $F_2(x,y)=(x,V_2(x,y))$
    \STATE set $\phi_k =  F_k \circ  R_k $ % and $\phi_2=F_2 \circ R_2$
    \RETURN $(\phi_1,\phi_2)$
\end{algorithmic}
\label{AlgoGlob}
\end{algorithm}



\subsubsection{Why does our method work?}
\label{WhyWork}

Intuitively, it may not be  obvious that the system of equations in Equation~\eqref{EqV1V2} is well posed.
In fact, if there was a functional relationship between
$p_1$ and $p_2$, as $p_1=F(p_2)$,  an infinity of solutions
for $(V_1,V_2)$ would exist, because   for any function $V: \RR^2 \rightarrow \RR $ we can generate a solution $(V,V\circ F)$.

\begin{figure}
\centering
\begin{tabular}{c}
% \hline \hline
\includegraphics[width=10cm]{FIGS/NonFuncCorresp.png}
% \\ \hline \hline
\end{tabular}
\caption{Left: for each $p_1$, we generate several $3$D points on $B_1(p_1)$. Middle:
         the multiple correspondences in $I_2$. Right: a dense network of curves in $I_2$.}
 
\label{NonFuncCorresp}
\end{figure}

However, note that due to {the $3$D aspect} of $p_1$ and $p_2$, there is
\emph{no} functional relationship between them and, consequently,
there are more constraints on $(V_1,V_2)$. Instead of a functional relationship,
we can generate  "one to many" (and  "many to one") correspondences as illustrated in Figure~\ref{NonFuncCorresp}.
For example, for a given  point $p_1$, following the curve $\pi_2(\BundO(p_1))$, we can generate
several points on the bundle (potentially an infinity) which results in many correspondences.
%by  $\pi_2$ for a single point. 
To illustrate, if we take $p^k_2$ to be multiple homologous points of $p_1$,
we then have the equation:
%
\begin{equation}
    V_1(p_1) = V_2(p^1_2)   \;;\; V_1(p_1) = V_2(p^2_2)   \;;\; V_1(p_1) = V_2(p^3_2)  \dots , \label{MultiTieP}
\end{equation}
%
which in fact enforces this constraint: 
%
\begin{equation}
V_2(p^1_2) = V_2(p^2_2)  =  V_2(p^3_2) \dots \label{RedrCurv}
\end{equation}
%
If we now look at left image of Figure~\ref{NonFuncCorresp}, we see that Equation~\eqref{RedrCurv}
imposes the constraint that a "piece of curve" is horizontal.
In Section~\ref{EpipTieP}, we will see a more detailed analysis explaining how the 
method can work even with configurations different than those depicted in Figure~\ref{NonFuncCorresp}.




%---------------------------------------------
%---------------------------------------------
%---------------------------------------------


\subsection{Detailed implementation}


\subsubsection{Choice of a parametric functional space}
\label{ChoicePolyn}

We need to select a space of parametric functions to represent $V_1,V_2$. The only constraint
is that $V_1,V_2$ are $\mathcal{C}^{\infty}$ functions, and that the additional constraint in 
Equation~\eqref{Ambig:Line} is valid. 

Classically, when parameterizing a set of functions  $\mathcal{C}^{\infty}$,
a "natural" candidate is the set of polynomials of a given degree. We know that the function will be
$\mathcal{C}^{\infty}$ and, according to the Stone-Weierstrass theorem \cite{Weierstrass1885,Stone1937} (which says that the space of polynomials is dense in the space of continuous functions),  with a sufficiently high
degree we will be able to accurately approximate any continuous function. A~possible
limitation of selecting high degree polynomial is over-fitting, which may lead to
unwanted high frequency behavior. In our case, this problem should never arise as the measurements  are synthesized from the projection functions $\pi_1,\pi_2$, which provides sufficient redundancy (for instance,
 hundreds of times more measurements than constraints). \mpd{Note, this will be a different issue,
and we will have to take care of degree, when we use
the method with tie points and without model in section \ref{EpipTieP}}

If $d$ is the selected degree, we have two vectors of unknowns $C^1_{a,b},C^2_{a,b}$, 
corresponding to coefficients of the polynomials:


\begin{equation}
   V_k(p) = V_k(i,j) =  \sum\limits_{\substack{a=0}}^d  \sum\limits_{\substack{b=0}}^{d-a}  C^k_{a,b}  i^a j^b. \label{EqPol}
\end{equation}
   
\subsubsection{Imposing constraints on global lines deformation}

%In this paramatrisation, we must take into account Equation~\eqref{Ambig:Line}.
\mpd{When applying the constraint of equation~\eqref{Ambig:Line} to equation~\eqref{EqPol}, 
we have $i=0$, thence we can suppress all terms  $i^a$ for $a\neq 0$}. The constraint equation then reads:


\begin{equation}
    V_1(0,j) =  j =   \sum\limits_{\substack{b=0}}^{N}  C^1_{0,b}  j^b  \label{CstrV1:0}
\end{equation}

In Equation~\eqref{CstrV1:0}, $j$ on the left and the  sum on the right are both polynomials, so if their functions are equal on a segment, they
must be equal term by term. The constraint then comes to force a number the
unknowns $C^1_{0,k}$ which have known values: $1$ for $C^1_{0,1}$ and $0$ otherwise.
Using the Kronecker delta, we can write:

\begin{equation}
         C^1_{0,k} = \delta_{1,k} \label{CstrV1:1}
\end{equation}

\subsubsection{Generation of points, computation of the direction and centers}

The  points from $\pi_1$ and $\pi_2$ are generated twice, using each image as the master. The bundles are always generated from the  master images. The Algorithm~\ref{AlgoGenData} presents the 
generation of the points with $I_1$  as the master, as well as the computation of the global direction and the points' centers.

\begin{algorithm}[H]
%\emph{Compute a list $L_{1,2}$  of   $\pi_1-\pi_2$ H-compatible pair with $I_1$ as master image,  compute also center $c_1$ of points of $I_1$  and global direction $\vec{D}_2$ for epipolar curves of $I_2$}
\caption{GenerateData(). \emph{Compute a list $L_{1,2}$  of   $\pi_1-\pi_2$ H-compatible pairs with $I_1$ as the master image.
 Compute also the center $C_1$ of points in $I_1$  and the global direction $\vec{D}_2$ for epipolar curves of $I_2$.}}
\begin{algorithmic}
    \STATE $L_{1,2}\gets () $ ;  $C_1 \gets (0,0)$  ;   $\vec{D}_2 \gets  \overrightarrow{(0,0)}$ ; $N \gets 0 $
    \FOR{$p_1.x=0$  \TO $X_1$  {\bf Step} $\delta_{x,y}$   } 
        \FOR{$p_1.y=0$  \TO $Y_1$  {\bf Step} $\delta_{x,y}$    } 
             \FOR{$z=Z_0$  \TO $Z_1$  {\bf Step} $\delta_{z}$   } 
                  \STATE {$p_2 = \pi_2(\PiVert^{-1}_1(p_1,Z))$}
                  \STATE {$p'_2 = \pi_2(\PiVert^{-1}_1(p_1,Z+\delta_{z}))$}
                  \IF {$p_2 \in I_2$ \AND $p'_2 \in I_2$}
                       \STATE $L_{1,2}.append((p_1,p_2))$  
                       \STATE $C_1 \gets C_1 + p_1$
                       \STATE $\vec{D}_2 \gets  \vec{D}_2 + \frac{\overrightarrow{p_2 p'_2}}{|p_2 p'_2|}$
                       \STATE $N \gets  N +1 $
                  \ENDIF
             \ENDFOR
        \ENDFOR
    \ENDFOR
    $C_1 \gets \frac{C_1}{N}$  ; $\vec{D}_2 \gets \frac{\vec{D}_2}{N} $
\end{algorithmic}
\label{AlgoGenData}
\end{algorithm}


%  \er{what is it? \ref{PiInvert}}
%\paragraph{Using center and direction}

Once the centers $C_1,C_2$, directions   $\vec{D}_1,\vec{D}_2$  and the list $L_{1,2}$ are computed,
they are used to normalize the measurements and make the direction globally
horizontal by applying Equation~\eqref{EqRot} to all elements of the list.


\subsubsection{Estimating the rectification}

As the measurements are synthetic and without outliers, we can directly solve
the equations with the linear least squares  method, thus merging all previous steps:

\begin{itemize}
    \item let $d$ be the degree of the polynomials;
    \item the unknowns are the coefficient of the polynomials $V_1$ and $V_2$. There are
          $\frac{(d+1)(d+2)}{2}$ unknowns for $V_2$ and $\frac{(d+1)(d+2)}{2}-(d+1) $  for $V_1$,
          taking into account the  constraint in Equation~\eqref{CstrV1:1};
     \item for each pair of normalized points $q_1,q_2$ we add 
          the Equation~\eqref{EqPol} to the least squares equation system.
\end{itemize}
%
We then estimate the $V_1,V_2$ and obtain:
%
\begin{equation}
  \varphi_k(p) = \varphi_k(i,j) = (i,V_k(i,j))  \;;\;    \phi_k =  \varphi_k  \circ R_k 
\end{equation}

\paragraph{Estimating the inverse function}

The natural way to resample  $I_k$ in $E_k$ is to write:

\begin{equation}
  E_k(p) = I_k(\phi^{-1}_k(p)).
\end{equation}
Therefore, to rectify an image, we also need to calculate the inverse function. The inverse of $R_k$ is obvious. For computing the inverse of $\varphi_k$,  we
exploit the fact that if $\varphi$ is invariant for the column, then $\varphi^{-1}$ is invariant too. Consequently, we can parametrize it with a function $W: \RR^2 \rightarrow \RR$ as:

\begin{equation}
  \varphi^{-1}_k(p) = \varphi^{-1}_k(u,v) = (u,W_k(u,v))  
\end{equation}
%
To estimate  $W$, we follow the same rationale as in Section~\ref{ChoicePolyn}, and
use the base of a polynomial function. Once the $V_k$ are known, we generate
for each point $p_k=(i,j)$ in $L_{1,2}$ an observation:

\begin{equation}
   W_k(i,V_k(i,j))  = j \label{InverseEpip}
\end{equation}

If we want to ensure that the computed inverse is sufficiently close
to the "real" inverse, we can increase the polynomial's degree (in our implementation, we typically use the degree of $d+4$). It has no side effects as long as we maintain high redundancy.

%---------------------------------------------
%---------------------------------------------
%---------------------------------------------

\BEGINCHANGE{}

\subsection{Epipolar resampling without the geometric model -- \er{not too happy with this section} \mpd{me neither (;-)}}

\subsubsection{Possibility with only tie-points}

\label{EpipTieP}
\er{Ideally, there is no bullet-ing but full sentences.}


\emph{"Is it possible to use
the proposed method to compute the epipolar geometry
if we have point correspondences between the image pairs but we don't know the geometric model?"} There is \mpd{NO \sout{a}}
 straightforward answer to whether it is possible or not. The answer is that is not possible in general,
but it becomes possible when the relief is not smooth and we add the contraint that the ressampling is smooth enough.
 %As elaborated in the following pros and cons arguments, the
 % answer depends on the point correspondences, 
%  and is a trade-off between the nature of the 3D scene and the complexity of the model. 
%In fact there is \er{easy yes/no answer as there pro and cons arguments to solve this question} and, as we will see, the precise answer depends of the tie points (it is in fact a trade-off between the nature of the relief and complexity of the model).
%

{\underline {\bf Pro:}}  Algorithm~\ref{AlgoGlob} uses only point correspondences and a
direction to compute the epipolar geometry. It does not matter if the point correspondences are extracted from the geometric model
(as with Algorithm~\ref{AlgoGenData}) or from an image 
processing method, e.g. SIFT \cite{lowe2004distinctive}, which does not require any
\emph{a priori} information on image geometry. Hence, as long as 
we know the directions, our method can be used.


{\underline {\bf Cons:}} When the point correspondences are computed with an image processing routine,
 there exists some functional relationship between them. Suppose that:

\begin{itemize}
   \item the 3D scene can be described by a function $Z=\mathcal{Z}(X,Y)$,
         and denote $S^\mathcal{Z}$ as the corresponding surface;
   \item for a point $p_1$ of $I_1$, denote $ \PiZVert (p_1)$
         as the intersection of  $\BundO(p_1)$ and the surface  $S^\mathcal{Z}$;
         $\PiZVert$ is the inverse of the \er{restriction}  $\pi_1$ 
         as function from $I_1$ to $S^\mathcal{Z}$.
\end{itemize}
We now see that there exists a functional relationship between all the point correspondences $(p_1,p_2)$ and it follows the equation:
\begin{equation}
   p_2 = (\pi_2 \circ  \PiZVert) (p_1) = F^\mathcal{Z}(p_1).
\end{equation}
%
Therefore, as discussed in Section~\ref{WhyWork}, in the most general
case, it is impossible to recover the epipolar geometry from a set of correspondences.% when there is a functional relation $p_2=F(p_1)$ between them. 

{\underline {\bf Pro \& Cons:}} If we go back  to Section~\ref{WhyWork},
we can see that we missed the fact that in the proposed method,
the functions $V_1$ and $V_2$ have to be "smooth". 
Let's reason again:

\begin{itemize}
   \item suppose we have computed an epipolar geometry: $e_k=(u_k,v_k)=\phi_k(p_k) = (x_k,V_k(y_k)$,
         with $v_1=v_2$ and $V_k$ being a "smooth" function and we try to analyze if the
         geometry was ambiguous;

   \item  we have $e_2 = (\phi_2 \circ  \pi_2 \circ  \PiZVert \circ  \phi_1^{-1}) (e_1) = P(e_1)$

    \item we write the previous equation as $(u_2,v_2) = P(u_1,v_1) = (u_1 + p_x(u_1,v_1),v_1)$,  where
          $p_x$ is what is usually called the "parallax function"; 

   \item as in Section~\ref{WhyWork},  for any function $W_2 : \RR^2 \rightarrow \RR  $, let $W_1$ be the function
         defined by $W_1 = W_2 \circ P$;

   \item then, for any $e_1,e_2$, $(u_1,W_1(u_1,v_1))$ and $(u_2,W_2(u_2,v_2))$, we also satisfy the epipolar constraint
         as $W_1(u_1,v_1)) = W_2 (P(u_1,v_1)) = W_2(u_2,v_2)$.

\end{itemize}

Is it possible that $W_2$ and $W_2 \circ P$ are both smooth functions? This depends on the smoothness of $P$.
%
If $P$ is itself a smooth function, then obviously for any smooth $W_2$,  $W_2 \circ P$  will
also be smooth and the epipolar resampling will be ambiguous. If we take the canonical example of a flat 3D scene, then $p_x=0$,  $P=Id$ and as $W_1=W_2$, $W_1$ is smooth if $W_2$ is smooth. 
In this case the epipolar geometry is also ambiguous.
%
However, if the scene has high frequency depth changes, then $P$ also has high frequencies,
and  $W_2$ and $W_2 \circ P$ cannot both be smooth. This can be seen more formally
by the following equation:

\begin{equation}
   \DerPart{W_2 \circ P}{u} = \DerPart{P}{u} \DerPart{W_2} {u} \circ P  \label{DerParW2P}
\end{equation}

{As an archetype case, for any point where the scene is not differentiable,
we have $ \DerPart{P}{u}= \infty$ and Equation~\eqref{DerParW2P} which lead to $ \DerPart{W_2} {u}=0$
because $W_2 \circ P$ is supposed to be smooth.
If $W_2$ is a polynom of limited degree, and we have sufficient number of points with, $ \DerPart{W_2} {u}=0$,
then it will conduct to the fact that $W_2$ depends only of $v$, adding finally equation~\eqref{Ambig:Line},
we have the $W_2=Id$, which show the uniqueness of  the solution.
}


\subsubsection{Estimating the directions without model}

For using our methods, with tie-points, we need also a way to estimate the global directions, 
as it is done in  section~\ref{EstCenterDir} when we have the geometric models.

In the implemantation we provide, we give two options to the user : provide it or let the
programm compute it automatically.  Obviously the providing option can be safer when the user has 
sufficient knowledge on the acquisition; it can be used for examle with like satellite  along the track ,
like in the example we provide with Corona acquisition. The automatic option is based
on a  discretization of possible directions and combinatorial exploration of all the pair 
of possible direction ; for a given pair, a quality  criterion is computed  on residual
of epipolar ressampling with degree $0$ \mpd{TO CHECK !!!} , using a $L1$ norm (to be more robust to outliers).


\mpd{Faut il donner un algorithme ? Ou est-ce suffisement simple pour être potentiellement clair ?}

\ENDCHANGE{}





%---------------------------------------------
%---------------------------------------------
%---------------------------------------------

\section{Experiments}\label{sec:experiments}

We demonstrate the performance of the algorithm in two scenarios: with and without the geometric model of the camera. In the first scenario, we use high resolution satellite images acquired with the Pl\'eiades-1A satellite. We form pairs of images of varying base-to-height ratios ($\sfrac{B}{H}$), and differentiate between \textit{single orbit} and \textit{multiple orbit} acquisitions. The results are evaluated in terms of the remaining \textit{y-parallaxes}, and compared to a competetive method by Oh~\cite{Oh2011}.
 
In the second scenario, we do not dispose of the camera geometry, and the epipolar resampling is based on image correspondences (i.e., SIFT~\cite{lowe2004distinctive}). \er{ bla bla how we evaluate it and on which datasets}
 

Exemple de résultats avec résidus. + qq crop d'images rectifiees.

\paragraph{Pl\'eiade-1A images with known geometry}
% 
The geometries of 5 images with their corresponding RPC coefficients are first refined in a RPC-bundle adjustment~\cite{rupnika2016refined}. We then combine them in pairs of $\sfrac{B}{H}$ range $\in \left< 0.1,0.45 \right> $. Table~\ref{tab:PHR1mono} shows the remaining y-parallaxes 


about $Z_buff$, what is that we want from it

hyperparms:
- grid XY for estimation, 100x100
- NbZRand equal to 1
- NbZ equal to 1 (three layers)
 
%%%%%%%%%%%%%%%%%%%%%%%%%%%%%%%%%ù
\begin{table}[h!]

\begin{center}
\begin{tabular}{|c|c|c|c|c||c|c|c|c|}

%\cline{2-5} 
\multicolumn{1}{c}{}  & \multicolumn{4}{c}{Oh~\cite{Oh2011}} & \multicolumn{4}{c}{Ours}   \\
 \cline{2-9} 
 \multicolumn{1}{c}{}  & \multicolumn{4}{|c||}{ $\sfrac{B}{H}$ } & \multicolumn{4}{|c|}{$\sfrac{B}{H}$ }  \\
 %\cline{2-9}
 \multicolumn{1}{c|}{$Z_{buff}$ [m]}  &  0.1 &  0.15   & 0.25  &  0.45 &  0.1 &  0.15   & 0.25  &  0.45   \\
 \cline{1-9}
  
 50 & 0.0070  & 0.0142    & 0.0156   &  0.0084   &  0.0017   &  0.0017  & 0.0017  & 0.0012\\
100 & 0.0065  & 0.0144    & 0.0157   &  0.0092   &  0.0019   &  0.0018  & 0.0017  & 0.0012\\
200 & 0.0057  & 0.0151    & 0.0164   &  0.0098   &  0.0027   &  0.0022  & 0.0022  & 0.0014\\
270 & 0.0055  & 0.0156    & 0.0144   &  0.0010   &  0.0036   &  0.0026  & 0.0026  & 0.0020\\
\hline 

\end{tabular}
\end{center}
\caption{Maximum value of the remaining y-parallax [pix] for acquisitions from a \textbf{single orbit}. $Z_{buff}$ corresponds to half the depth of the volumne used in the resampling calculation.}\label{tab:PHR1mono}
\end{table}
%%%%%%%%%%%%%%%%%%%%%%%%%%%%%%%%%%%%%%%
\begin{table}[h!]

\begin{center}
\begin{tabular}{|c|c|c|c|c||c|c|c|c|}

%\cline{2-5} 
\multicolumn{1}{c}{}  & \multicolumn{4}{c}{Oh~\cite{Oh2011}} & \multicolumn{4}{c}{Ours}   \\
 \cline{2-9} 
 \multicolumn{1}{c}{}  & \multicolumn{4}{|c||}{ $\sfrac{B}{H}$ } & \multicolumn{4}{|c|}{$\sfrac{B}{H}$ }  \\
 %\cline{2-9}
 \multicolumn{1}{c|}{$Z_{buff}$ [m]}  &  0.13 &  0.2   & 0.3  &  0.4 &  0.13 &  0.2   & 0.3  &  0.4   \\
 \cline{1-9}
  
 50  &  0.0297    &  0.0059   & 0.0178   & 0.0112 & 0.0021   & 0.0021     & 0.0021    &  0.0021   \\
100  &  0.0284    &  0.0088   & 0.0183   & 0.0134 & 0.0029   & 0.0037     & 0.0034    &  0.0034   \\
200  &  0.0325    &  0.0196   & 0.0252   & 0.0232 & 0.0064   & 0.0010     & 0.0086    &  0.0088   \\
270  &  0.0381    &  0.0311   & 0.0343   & 0.0340 & 0.0101   & 0.0171     & 0.0143    &  0.0154   \\
\hline 

\end{tabular}
\end{center}
\caption{Maximum value of the remaining y-parallax [pix] for acquisitions from \textbf{multiple orbits}. $Z_{buff}$ corresponds to half the depth of the volumne used in the resampling calculation.}\label{tab:PHR1multi}
\end{table}
%%%%%%%%%%%%%%%%%%%%%%%%%%%%%%%%%%%%%%%
\begin{table}[h!]

\begin{center}
\begin{tabular}{|c|c|c|c|c||c|c|c|c|}

%\cline{2-5} 
\multicolumn{1}{c}{}  & \multicolumn{4}{c}{Single orbit} & \multicolumn{4}{c}{Multiple orbits}   \\
 \cline{2-9} 
 \multicolumn{1}{c}{}  & \multicolumn{4}{|c||}{ $\sfrac{B}{H}$ } & \multicolumn{4}{|c|}{$\sfrac{B}{H}$ }  \\
 %\cline{2-9}
 \multicolumn{1}{c|}{$Z_{buff}$ [m]}  &  0.1 &  0.15   & 0.25  &  0.45 &  0.13 &  0.2   & 0.3  &  0.4   \\
 \cline{1-9}
  
 50-270  &   1.6$e^{-5}$    &  0.8$e^{-5}$  &  0.3$e^{-5}$  &  0.1$e^{-5}$   &   2.0$e^{-5}$  &     0.7$e^{-5}$  &  0.5$e^{-5}$   &    0.2$e^{-5}$   \\
%100  &       &      &       &     &     &      &     &       \\
%200  &       &      &       &     &     &      &     &       \\
%270  &       &      &       &     &     &      &     &       \\
\hline 

\end{tabular}
\end{center}
\caption{Epipolarability indices for acquisitions from a single and {multiple orbits}.}\label{tab:PHR1indices}
\end{table}

%\subsection{Radar images}

 

%\subsection{Use with pinhole camera}
%Permet de faire des test supplémentaire.

 
\paragraph{Corona KH-4B images without the geometric model}
Exposé precis avec modele analytique:

    * calcul des points homologues, en 3D => direction moyenne
    * resolution par moindres carress avec degres élevé, degré sup pour l'inverse
    * Utilité du 3D, precision en fonction de la nappes, possibilité d'utiliser un modèle 3D grossier .


\begin{figure}[h!]
\centering
\begin{tabular}{c}
% \hline \hline
\includegraphics[width=14cm]{FIGS/ExpCorona_lowres.png}
% \\ \hline \hline
\end{tabular}
\caption{Corona KH-4B steropair and the point correspondences (top); the epipolar curves in the original geometry of the stereopair (middle); and the stereopair resampled to the epipolar geometry, rotated by 90$^\circ$ (bottom).}
 
\label{ExpCorona}
\end{figure}
    
    
%\paragraph{Pl\'eiades images}
%---------------------------------------------
%---------------------------------------------

\paragraph{Consumer grade camera}
%---------------------------------------------
\begin{figure}[h!]
\centering
\begin{tabular}{c}
% \hline \hline
\includegraphics[width=14cm]{FIGS/escalier_sol.jpeg}
% \\ \hline \hline
\end{tabular}
\caption{dddd consumer grade cam}
 
\label{ExpConsumerCam}
\end{figure}

\section{Discussion and perspective}

Avantage de la methode : modeles analyique, minimise deformation et residu, peut être appliquee 
si on a que les points homologues (photo escalier ?).

Inconvenient ? Cas comme \ref{BadGoodEpip} pas gere, peut être le sont il par Oh ?

Future work => use MNT ? Test sur configuration plus compliquées : orbites differentes ? Radar ? Radar/visible ?






%---------------------------------------------
%---------------------------------------------
%---------------------------------------------

\bibliographystyle{siam}
\bibliography{epip}
%\begin{thebibliography}{References}

%   \bibitem[Oh,     Jaehong. 2011]{Oh2011}
%            Oh,     Jaehong. 2011,     Novel     Approach     to     Epipolar Resampling  of  
%            HRSI  and  Satellite  Stereo  Imagery-based Georeferencing   of   Aerial   Images.Diss.
%            The   Ohio   State University.
%
%   \bibitem[De Franchis,Carlo 2015]{Franchis2015}   Earth Observation and Stereo Vision.
%           Ph Dissertation, Paris-Saclay 2015.
%
%    \bibitem[Weierstrass,Karl 1885 ]{Weierstrass1885} Über die analytische Darstellbarkeit sogenannter willkürlicher Functionen 
%           einer reellen Veränderlichen. Sitzungsberichte der Königlich Preußischen Akademie der Wissenschaften zu Berlin, 
%           1885 (II).  Erste Mitteilung (part 1) pp. 633–639, Zweite Mitteilung (part 2) pp. 789–805.
%
%    \bibitem[Stone, M. H. 1937]{Stone1937} "Applications of the Theory of Boolean Rings to General Topology", 
%          Transactions of the American Mathematical Society, Transactions of the American Mathematical Society, 
%          Vol. 41, No. 3, 41 (3): 375–481
%\end{thebibliography}

\end{document}
%-------------------------------------------------------------------------------
