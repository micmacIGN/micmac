\chapter{Modelization of Bloc Rigid of instrument}


%-----------------------------------------------------
%-----------------------------------------------------
%-----------------------------------------------------

\section{Introduction}

In this chapter we describe how rigid bloc of instrument are handled in \PPP. In this approach
we try to have a unified description of all type of instrument that can be embeded
together in a rigid block. The type of instrument under consideration are :

\begin{itemize}
   \item   camera ;
   \item   clinometers ;
   \item   targets ;
   \item   GNSS ;
   \item   IMU;
   \item   others ?
\end{itemize}

At the time being, the implementation if far from complete and we have 
the following restriction :

\begin{itemize}
   \item   only camera and clinometers are handleds
   \item   a block must contain at least one camera  (btw, what would be the interest of a clinometers only block);
   \item   the system can handle a single block (not a hard constraint, just need to give access to mutiple block, an to have data
           to check the implementation...)
\end{itemize}



%-----------------------------------------------------
%-----------------------------------------------------
%-----------------------------------------------------

\section{Formalisation}

\subsection{Equation between instrument}


Let $k_1$ and $k_2$ be $2$ instrument,





\subsection{Repair of the block}

For each bloc $k$ we have an unknown pose  $(R_k,C_k)$,



%-----------------------------------------------------
\subsection{Targets}

Targets can be a block of rigid target  alone , for example
if we have one (or several) pannel of target as used in CERN data-set, and simultaneously 
some target in word coordinates, we can no longer  consider that the panel is "the" system
of coordinates and we have to introduce multiple coordinate system.

Another case of use, is when we "stick" targets on a camera (the so called 
"arroseur arros\'e" or "bitter bit").



%-----------------------------------------------------
%-----------------------------------------------------
%-----------------------------------------------------

\section{User's point of view}

%-----------------------------------------------------
%-----------------------------------------------------
%-----------------------------------------------------

\section{Programmer's point of view}




