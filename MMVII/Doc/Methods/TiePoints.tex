

\chapter{The "Aime" methods for Tie Points computation}


% Conclusion, on peut sans doute limiter le nombre de point avec ScaleStab
% pour filtrage a priori => genre les 500 les plus stable

%---------------------------------------------
%---------------------------------------------
%---------------------------------------------

\section{Indexing problem}

%---------------------------------------------

\subsection{Motivation}
For each image, we have computed tie points. A tie points is made
of a vector $V \in \RR^n)$ . Typically $V$ is invariant
to the main geometric deformation .  We want to compute 
a function $\Psi $ that indicate if two vector $V_1$ and $V_2$
correspond to the same tie points.

\begin{equation}
   \delta R = \frac{1}{\sigma} \frac{L}{S} = \frac{1}{\sigma}  \frac{2 \pi a }{ e \; dz} 
\end{equation}

\begin{equation}
   \phi  = \iint \overrightarrow{B} \overrightarrow{dS} 
         = B_M \cos(\omega t) \pi a^2
\end{equation}

\begin{equation}
   e = -\frac{d\phi}{dt}= B_M \omega  \sin(\omega t)  \pi a^2
\end{equation}


\begin{equation}
   d P_J = \frac{e^2}{\delta R}  
         = \frac{ (B_M \omega \pi a^2  \sin(\omega t))^2 \sigma_e dz }{2 \pi a}
\end{equation}


\begin{equation}
   <d P_J> = \frac{B_M^2 \; \omega ^2 \; \pi \; a^3 \; e \; \sigma \; dz}{4}
\end{equation}

\begin{equation}
  <P_J> =  \int_{z=0}^H <d P_J> = \frac{B_M^2 \; \omega ^2  \; \pi \;  a^3  \; e  \; \sigma  \; H}{4}
\end{equation}

\begin{equation}
  dU = \delta ^2 Q_{creee} +  \delta ^2 Q_{recue de l'air}
\end{equation}

\begin{equation}
  C dT = <P_J> dt - g 2\pi aH(T-T_a)
\end{equation}

\begin{equation}
  C = \mu c 2 \pi a e H
\end{equation}


\begin{equation}
  \mu c 2 \pi a e H \frac{dT}{dt} = \frac{B_M \omega^2 \pi a^3 e \sigma H}{4} - g 2 \pi a H (T-T_a)
\end{equation}


\begin{equation}
  \tau  \frac{dT}{dt} + T = T_{\infty}
\end{equation}


\begin{equation}
  \tau  = \frac{\mu  c e}{g}
\end{equation}

\begin{equation}
  T_{\infty} - T_a = \frac{(B_M \omega a)^2 e \sigma}{8g}
\end{equation}



