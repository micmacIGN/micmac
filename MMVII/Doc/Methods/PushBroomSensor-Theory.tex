\chapter{Pushbroom sensors}
\label{Chap:PushBroom}


%-----------------------------------------------------------------------
%-----------------------------------------------------------------------
%-----------------------------------------------------------------------


%  9 0157  0002 2184 5
%  90157000221845
\section{Introduction }

%-----------------------------------------------------------------------
\subsection{Notice to the reader}

This chapter present how pushbroom sensors are handled in {\tt MMVII}.
It is written before a programmation session scheduled in mars $2024$
as a support to this session;
it will be probably significantly modified after the session and is different
from the other chapter in two points :

\begin{itemize}
	\item  the  chapter mixes theoreticall aspects with practicall aspects
		of implementation in micmac (some reorganization will probably occur later);
	\item  the theory and organisation are written (at least the first drafts)
               before the actual implementation;
\end{itemize}

%-----------------------------------------------------------------------

\subsection{Physicall and mathematical modelisation}

  %  -  -  -  -  -  -  -  -  -  -  -  -  -  -  -  -  -  -  -  -  -  -  -  -  -  -  -  -  -  -  -  -
\subsubsection{Physicall model}

A pushbroom sensor is made a monodimensional line sensor which has deplacement during the time.
This is the deplacement that create the second dimension. We suppose in the following  that in the
formed  image the coordinte $y$ correspond to a line og the monodimensionnal sensor while the coordinate
$x$ correspond to the "time". We note :

\begin{itemize}
	\item  mmm
\end{itemize}


\begin{itemize}
    \item aaaaa
\end{itemize}



%-----------------------------------------------------------------------
\section{Vrac}

\begin{itemize}
    \item begin by example with polynomial image
    \item method with some reverse engenering,  $\pi$ and $\pi^{-1}$ being "black box"
\end{itemize}


