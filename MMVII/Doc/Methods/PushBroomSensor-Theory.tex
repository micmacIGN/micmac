\chapter{Pushbroom sensors}
\label{Chap:PushBroom}


%-----------------------------------------------------------------------
%-----------------------------------------------------------------------
%-----------------------------------------------------------------------


%  9 0157  0002 2184 5
%  90157000221845
\section{Introduction }

%-----------------------------------------------------------------------
\subsection{Notice to the reader}

This chapter present how pushbroom sensors are handled in {\tt MMVII}.
It is written before a programmation session scheduled in mars $2024$
as a support to this session;
it will be probably significantly modified after the session and is different
from the other chapter in two points :

\begin{itemize}
	\item  the  chapter mixes theoreticall aspects with practicall aspects
		of implementation in micmac (some reorganization will probably occur later);
	\item  the theory and organisation are written (at least the first drafts)
               before the actual implementation;
\end{itemize}

%-----------------------------------------------------------------------

\subsection{Physicall and mathematical modelisation}

  %  -  -  -  -  -  -  -  -  -  -  -  -  -  -  -  -  -  -  -  -  -  -  -  -  -  -  -  -  -  -  -  -
\subsubsection{Physicall model}

A pushbroom sensor is made a monodimensional line sensor which has deplacement during the time.
This is the deplacement that create the second dimension. We suppose in the following  that in the
formed  image the coordinte $j$ correspond to a line og the monodimensionnal sensor while the coordinate
$i$ correspond to the "time". We note :

\begin{itemize}
	\item  $C(t)=C(i)$ the position of the sensor at time $t$;
	\item  $R(t)=R(i)=[\vec{I},\vec{J},\vec{K}]$ the orientation of the sensor at time $t$;
	\item  $\vec{u}(j)$ the direction of bundle in the repair of the sensor;
	\item  $\vec{U}(i,j) = R(i)\vec{u}(j)$ the direction of bundle in the global repair;
	\item  $B(i,j)= (C(i),\vec{U}(i,j))$ the bundle issued of pixel $i,j$;
\end{itemize}

The localisation function is completly caracterized by the mapping $(i,j) \rightarrow B(i,j)$.
By the way, practically we prefer to use the projection $\pi$ that  compute the 
image $i,j$ of given ground point, $\pi : (x,y,z) \rightarrow (i,j)$.  To make
$\pi$ and invertible function we change slightly to $\pi : (x,y,z) \rightarrow (i,j,z)$.
The function "easy" to compute from the previous notation is $\pi^{-1}$,
we have the parametric equation of $B(i,j)(\lambda) = C(i) + \lambda \vec{U}(i,j)$,
and computing $\lambda$  to have the right $z$ we get :

\begin{equation}
	\pi^{-1}(i,j,z) = C(i) + \frac{z-C_z(i)}{U_z(i)} \vec{U}(i,j)
\end{equation}



%-----------------------------------------------------------------------
\section{Vrac}

\begin{itemize}
    \item begin by example with polynomial image
    \item method with some reverse engenering,  $\pi$ and $\pi^{-1}$ being "black box"
\end{itemize}


