
\chapter{Description de la m\'ethode Aim\'e}

\section{Introduction}

Aim\'e vise à \^etre une m\'ethode de calcul de points homologues qui sera  int\'egr\'ee dans MMVII. 
Son architecture est fortement inspir\'ee par SIFT qui a fait ses preuves depuis de longues
ann\'ees en tant que m\'ethode \emph{analytique}\footnote{"analytique" par opposition à m\'ethode d'apprentissage,
les ayatolah du deep dirait handcrafted} de r\'ef\'erence. Elle vise aussi à tirer parti de $10$ ann\'ees
d'exp\'erience d'utilisation de SIFT en photogramm\'etrie pour corriger les principaux point faibles
de SIFT dans ce contexte.  Notemment les principaux points  sur lesquels on souhaite am\'eliorer sont :


\begin{itemize}
   \item SIFT passe difficilement à l'\'echelle lorsque  l'on l'utilise sur des tr\`es grands jeux de donn\'ees,
         notamment il ne permet pas de d\'etecter rapidement les paires potentiellement homologues;

   \item lorsque l'on en dispose, SIFT n'utilise pas d'information  de spatialisation approch\'ee qui permetrait
         de faciliter l'appariement (plus robuste et plus rapide), voir de se passer dans ce contexte d'invariance
         inutile (i.e ne pas normaliser à l'\'echelle ou à la rotation si on a des information permettant de
         connaitres ces infos);

   \item tirer parti des m\'ethodes d'apprentissage pour avoir un appariement meilleurs (quoique veuille dire
         ceci); une contrainte pour tirer parti des m\'ethodes d'apprentissage est de disposer de  jeux de
         donn\'es de v\'erit\'e suffisemment denses; or cela est possible, au moins pour la phase d'appariement 
         en utilisant des jeux de donn\'ees trait\'es par des cha\^ines photogramm\'etriques automatiques.

   \item prendre en compte que les probl\`emes d'appariement rencontr\'es en photogramm\'etrie sont
         suffisement vari\'es pour qu'il soit n\'ecessaire de viser \'a une m\'ethode fortement param\'etrable
         pour pouvoir s'adapter \`a cette vari\'et\'e de probl\`emes.


\end{itemize}

    % ==================================================================================
    % ==================================================================================
    % ==================================================================================

\section{Architecture g\'en\'erale d'Aim\'e}

L'architecture est la suivante :

\begin{itemize}
   \item calcul de points caract\'eristiques, cette partie est celle qui resssemble le plus \'a SIFT;

   \item calcul de descripteurs, le descripteur est fait d'un (ou plusieurs) descripteurs binaires
         permettant de faire une pr\'e-s\'election rapide des couples potentiellement appariables et
         d'un descripteurs plus complets , ces descripteurs sont bas\'es sur un r\'e\'echantillonage
         log-polaire de du voisinage de chaque point;

   \item appariement en plusieurs \'etapes, on commence par des filtres relativement peu s\'electifs
        mais rapides et l'on compl\`ete par des calcul plus s\'electif sur les descripteurs complets;

   \item filtrage spatial imposant une coh\'erence spatiale sur le principe "les homologues de mes voisins
         sont les voisins de mes homologues"; \'eventuellement la phase d'appariement a pu \^etre volontairement
         ambigu\"e (chaque point a un ensemble d'homologues potentiels) , parce que  certaines ambigu\"it\'es ne
         peuvent  pas \^etres r\'esolues au niveau individuel, et la phase de filtrage cherchera \`a r\'esoudre
         ces ambigu\"it\'es par une approche de type relaxtion.

   \item \'eventuellement, am\'elioration de la pr\'ecision de localisation par des m\'ethodes basiques 
         de corr\'elation.

\end{itemize}

Le degr\'e d'impl\'ementation actuel est tr\`es in\'egal et diminue lorsque l'on avance dans le processus.
Sch\'ematiquement ;

\begin{itemize}
   \item pour les points caract\'eristiques, il existe un prototype complet que l'on peut consid\'erer comme 
         pr\'e-op\'erationnel; il est surement perfectible \'a terme, mais ce n'est pas une priorit\'e; 

   \item le descripteur en log-polaire est implant\'es, des descripteurs binaires ont \'et\'e prototyp\'es de
         mani\`ers handcrafted (par une ACP),  mais gagneraient problement \'a \^etre con\c{c}u avec une 
         approche bas\'ee sur de l'apprentissage;

   \item l'appariement a \'et\'e impl\'emnt\'e de mani\`ere basique, essentiellement pour valider les deux
         \'etapes pr\'c\'dentes;  une voie d'am\'elioration majeure (en tout cas esp\'er\'ee comme telle)
         serait de faire de l'apprentissage sur les paires de descripteur complets;

   \item rien n'a \'et\'e fait sur le filtrage spatial, il aussi esp\'er\'e  que l'approche 
         "appariement ambigue/filtrage spatial avec relaxation" soit une source d'am\'elioration importante
         dans certain cas , notamment le cas de sc\`enes avec des structures partiellement r\'ep\'etitives;

   \item enfin le raffinement par corr\'elation, least square matching n'a pas \'et\'e test\'e, il s'agit
         d'une option a priori facile, qui donnerait une plus value op\'erationnelle sans avoir de valorisation
         en recherche (c'est assez classique).

\end{itemize}


    % ==================================================================================
    % ==================================================================================
    % ==================================================================================

\section{Calcul des points caract\'eristiques}


\subsection{Multi-\'echelle}

\subsubsection{Pr\'esentation g\'en\'erale}

Comme indiqu\'e pr\'ec\'edemment, cette partie est celle qui s'inspire le plus directement de SIFT.
Notamment, comme dans SIFT :

\begin{itemize}
    \item on calcule une pyramide d'image d'echelle d\'ecroissante suivant une loi exponentielle, 
          on note $I_0$ l'image initiale, et $I_k$ la $k^{ieme}$ image, entre $I_k$ et $I_{k+1}$
          il y a un rapport d'\`echelle $\sigma$, donc par it\'erations successive entre $I_0$
          et $I_k$ il y a un rapport d'\'echelle $\sigma^k$;

    \item comme dans SIFT, $\sigma$ est choisi tel que $\sigma^n=2$ ou $n$ est un entier, et
          tous les $n$ l'image est d\'ecim\'e d'un facteur $2$ pour gagner un facteur important de temps
          de calcul, sachant que quand l'image est devenu suffisement floue, l'auto-corr\'élation entre un
          pixel et son voisin fait que l'on perd peu d'information avec cette d\'ecimation;

    \item  la pyramide d'\'echelle permet d'avoir plus de point quand cela est n\'ecessaire, et surtout
           permet d'obtenir l'invariance \`a l'\'echelle (pour deux images $I$ et $J$ prises \`a deux
           \'echelles diff\'erentes de rapport $R$, si un point $p$ est d\'etect\'e sur $I_k$, il pourra aussi
           \^etre d\'etect\'e sur $J_{k'}$ avec $R = \sigma^{k-k'}$ )

    \item  la convolution par une gaussienne est isotrope ce qui  est n\'ecessaire pour l'invariance par rotation;


\end{itemize}


\subsubsection{Imp\'ementation dans MMVII de la pyramide}

\emph{Il n'est pas prouv\'e que les choix d'impl\'ementation d\'ecrit ici soit judicieux, cette description
vise à comprendre le code. Des tests de performance \'a une impl\'ementation basique reste \`a effectuer.}

L'impl\'ementation repose sur deux remarques :

\begin{itemize}
    \item en vertu du th\'eor\`eme central limite , si on convolue avec suffisement de fois avec des fonctions r\'eguli\`eres,
          on converge vers la convolution avec une gaussienne;
    \item convoluer avec deux gaussiennes sucessive d'\'ecart type $a$ et $b$, a exactement le même effet que
          de convoluer avec une seule gaussienne d'\'ecart type $\sqrt{a^2+b^2}$;
  
\end{itemize}

Le calcul se fait en iterant une convolution par $e^{-a|x|}$ qui est une fonction 
"smooth" et dont le produit de convolution peut \^etre calcul\'e rapidement (algorithme r\'ecursif classique).
La fonction est impl\'ement\'ee dans {\tt ExpFilterOfStdDev}.  Cette fonction prend
en param\`etre un $\sigma$ et un nombre d'it\'eration $N$, elle calcule la valeur
$a$ qui donnera un \'ecart type de $\frac{\sigma}{\sqrt{N}}$,

A chaque octave  :

\begin{itemize}
   \item  pour $I_0$ le calcule se fait avec un nombre relativement important d'it\'eration pour
          bien approximer la gaussienne

   \item  ensuite le calcul de $I_{k+1}$ se fait \`a partir $I_k$ (en tenant compte du flou d\'ej\'a pr\'esent
          en $I_k$, on convolue par $\sigma^k \sqrt{\sigma^2-1}$), comme $I_k$ est d\'ej\'a le r\'esultat de plusieurs
          convolution, on peut faire moins d'it\'ération (MMVII en fait $4$ pour $I_0$, $3$ pour $I_1$, et $2$ ensuite);
\end{itemize}

Il y a un cas particulier pour la premi\`ere it\'eration, si on veut que la pyramide soit r\'eguli\`ere en \'echelle.
Typiquement si l'image initiale est tr\`e floue, il faudrait la d\'econvoluer, inversement si elle est tr\`es piqu\'ee,
il faudra la flouter.  Je ne vois pas d'autre solution que de faire une hypothèse sur la largeur de la t\^ache image.





\begin{equation}
     I_k = I_0  \circledast G(\sigma^k)
\end{equation}


\section{VRAC}

\begin{itemize}
   \item indexe binaire
    \item crit\`ere rapide de d\'etection de paires
\end{itemize}

