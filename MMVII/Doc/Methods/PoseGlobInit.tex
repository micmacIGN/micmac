\chapter{Global initialisation of pose}
\label{Chap:PoseGlobInit}


%-----------------------------------------------------------------------
%-----------------------------------------------------------------------
%-----------------------------------------------------------------------

\section{Introduction}

%-----------------------------------------------------------------------

\subsection{Context and scope}

This draft describe some ideas on a possible method for pose initialisation in \PPP.

Many of the material ideas presented are not (or probablby not) original, but maybe thir carefull use can guide
to an efficient method, because evil is in detail.

We suppose that in entry have a set of pair, triplet \dots with their relative orientation, and
optionally the covariaance.

%-----------------------------------------------------------------------

\subsection{Reminder on pose}

We note $\mathcal{P} = [C,R]$ a pose composed from the center and orientation of the camera.
$\mathcal{P}$ seen as the function that tranformate camera coordinate to \emph{word} coordinates,
 can be considered as a mapping from $\RR^3$ to $\RR^3$ :

\begin{equation}
   \mathcal{P}  :  \RR^3 \rightarrow \RR^3,  X \rightarrow \mathcal{P}(X) = C + R X
   \label{PGI:PoseAsFunc}
\end{equation}

Using composition of function, equation \ref{PGI:PoseAsFunc} inferate a group structure on pose, and we have :


\begin{equation}
    \mathcal{P}_a  \mathcal{P}_b = [C_a,R_a] [C_b,R_b] = [C_a + R_a C_b , R_a R_b]
   \label{PGI:PoseCompos}
\end{equation}

\begin{equation}
    \mathcal{P}^{-1}  = [C,R] ^{-1} = [- R^{-1} C,R^{-1}]
   \label{PGI:PoseInv}
\end{equation}

Let $\mathcal{P}_a,\mathcal{P}_b$ be the pose of camera $a$ and $b$, the pose $\mathcal{P}_{b/a}$,
a.k.a the pose of $b$ relatively to $a$,
transforming coorinates $X_a$ to $X_b$ is given by  :

\begin{equation}
    X_b =  \mathcal{P}^{-1}_b(X) =   \mathcal{P}^{-1}_b(\mathcal{P}_a(X_a))
   \label{PGI:PoseRel}
\end{equation}

Then :

\begin{equation}
    \mathcal{P}_{b/a}  =  \mathcal{P}^{-1}_b \mathcal{P}_a 
   \label{PGI:PoseRel}
\end{equation}

If often happend that two pose are equal to up to scale $\lambda$, we write :

\begin{equation}
    \mathcal{P}_a  \equiv   \mathcal{P}_b  \Leftrightarrow \exists \lambda / [C_a,R_a] = [\lambda  C_b,R_b]
   \label{PGI:PoseRel}
\end{equation}


Also if $\mathcal{P}_a,\mathcal{P}_b$ are poses relative to a word coordinate $W$,
and $\mathcal{P}'_a,\mathcal{P}'_b$  are pose ralative to another word coordinate $W'$,
as they both are relative poses and equal up to a scale factor, we have :

\begin{equation}
     \mathcal{P}_b ^{-1}  \mathcal{P}_a  \equiv \mathcal{P}'^{-1}_b  \mathcal{P}'_a
   \label{PGI:InvarPoseRel}
\end{equation}


%-----------------------------------------------------------------------

\subsection{Set of relative poses}

We have a set of $M$ poses to estimate.
We suppose that we have set of subset (pair, triplet \dots) for which we have computed
the relative poses. Let note :

%   p_ki ->  Wk   -> W ->PI  :    PI-1 = Wk p_ki 
%

\begin{itemize}
    \item  $N$  the number of  subset  and $n_k$ be cardinality of $k_{th}$ subset $S_k$;
    \item  we write $S_k =\{I^k_1,\dots I^k_{n_k}\} $  where $k \in [1,N]$ , $ i \in [1,n_k]$ , $ I^k_i \in [1 , M] $ ;
    \item   we suppose that each $S_k$ has been oriented in its own word coordinate $W_k$ ;
    \item   we note $\mathfrak{p}^k_i=[c^k_i, r^k_i]$ the pose (relative) of $I^k_i$ in $W_k$;
\end{itemize}

We try to compute the poses in some common word coordinates $W$, all the poses $\mathcal{P}(I)=[C(I),R(I)]$
taking into account all the  $\mathfrak{p}^k_i$  .  Using equation \ref{PGI:InvarPoseRel}
we have :

\begin{equation}
     \forall k \in [1,N]   \forall i,j \in [1,n_k]  :
     \mathcal{P}^{-1}(I^k_j)  \mathcal{P}(I^k_i)  \equiv  \mathfrak{p}{^k_j}^{-1} \mathfrak{p}^k_i 
   \label{PGI:PoseRelAb}
\end{equation}

As all poses in each word are defined up to a global pose we set :
\begin{equation}
     \forall k \in [1,N]   
      \mathfrak{p}^k_1  = Id
   \label{PGI:FirstPoseId}
\end{equation}

And we  simplify ~\ref{PGI:PoseRelAb}  by :

\begin{equation}
     \forall k \in [1,N]   \forall i \in [2,n_k]  :
         \mathcal{P}^{-1}(I^k_1)  \mathcal{P}(I^k_i)  \equiv   \mathfrak{p}{^k_1}^{-1} \mathfrak{p}^k_i   =\mathfrak{p}^k_i 
   \label{PGI:PoseRelAb2}
\end{equation}

Or :

\begin{equation}
     \forall k \in [1,N]   \forall i \in [2,n_k]  :
         \mathcal{P}^{-1}(I^k_1)  \mathcal{P}(I^k_i) = [\lambda_k c^k_i, r^k_i]
   \label{PGI:PoseRelAb2}
\end{equation}

%-----------------------------------------------------------------------

\subsection{Setting  a linear problem}

In equation ~\ref{PGI:PoseRelAb2}  the $\mathcal{P}(I^k_i)$ and $\lambda_k$ are the unknowns and the $\mathfrak{p}(I^k_1) $
are observations , to have  a linear formula we write :


\begin{equation}
     \forall k \in [1,N]   \forall i \in [2,n_k]  :
           \mathcal{P}^{-1}(I^k_1) = [\lambda_k c^k_i, r^k_i]  \mathcal{P}^{-1} (I^k_i)
   \label{PGI:PoseRelAb3}
\end{equation}

Now we considers that unknown are not the  $\mathcal{P}(I)$  but rather the $\mathcal{P}^{-1}(I)$ and we write :

\begin{equation}
      \mathcal{P}^{-1}(I) = [C'(I),R'(I)]
\end{equation}

We then have :

\begin{equation}
     \forall k \in [1,N]   \forall i \in [2,n_k]  :
           [C'(I^k_1),R'(I^k_1)]  = [\lambda_k c^k_i + r^k_i C'(I^k_i) , r^k_i R'(I^k_i)]  
   \label{PGI:PoseRelAb3}
\end{equation}

Here the unkwons are the $C'(I),R'(I),\lambda_k$.

We have then two independant set of equations. Equations for translations  with  $3M+N$ unknwons :

\begin{equation}
     \forall k \in [1,N]   \forall i \in [2,n_k]  :
     C'(I^k_1) = \lambda_k c^k_i + r^k_i C'(I^k_i)  
     \label{PGI:EqTrans}
\end{equation}

If $M \gg N$ and we use least square method with normal matrix, the $\lambda_k $ can be eliminated
using schurr complement.


Equation for rotations :

\begin{equation}
     \forall k \in [1,N]   \forall i \in [2,n_k]  :
           R'(I^k_1)  =  r^k_i R'(I^k_i)
   \label{PGI:PoseRelAb3}
\end{equation}


Going to quaternions, and noting $Q',q$ the quaternion associated to $R',r$ , we can write :

\begin{equation}
     \forall k \in [1,N]   \forall i \in [2,n_k]  :
           Q'(I^k_1)  =  q^k_i Q'(I^k_i)
   \label{PGI:EqQuat1}
\end{equation}

To the price of having $4M$ unknowns, rather than $3N$, we have a linear problem.

%-----------------------------------------------------------------------

\subsection{Gauge}

$R(I)$, $C(I)$ and $\lambda_k$ are determined up to some arbitary repair (
a global similitude). So are $R'(I)$, $C'(I)$ and $\lambda_k$.
We then have to use some gauge in equation \ref{PGI:EqQuat1} and \ref{PGI:EqTrans}.

As a reminder we have $R'(I) = R^{-1}(I)$ and $C'(I) = - R^{-1}(I) C(I) $.

For equation~\ref{PGI:EqQuat1}, $R$ and $R'$ play similar role  in equations,
and we can simply set arbitrarily the firt pose as identity:

\begin{equation}
       Q'(1) = 1
   \label{PGI:GaugeQuat}
\end{equation}

For equation~\ref{PGI:EqTrans},  for $C$ we usally set $ \sum C(I)=$;
but due the more complicated relation we cannot set for example 
$\sum C'(i)=0 $.  By the way an alternative is to set $C(1)=0$,
and in this case the relarion between $C$ and $C'$ lead to $C'(1)=0$.

And for $\lambda_k$ we can set for example $\lambda_1=1$.



%-----------------------------------------------------------------------

\subsection{Can it work ?}

The previous method is relatively classical . It's not reputed to  be very stable.
Why could it work better with our data ?  We hope that adding pairs and triplet
can make it more robust.

Open question, can we add covariance information on $ r^k_i , c^k_i $ ?  Probably not rigourously,
as when we already have an approximate solution, but maybe just the uncertaincy can be used in weigthing the equation ?




