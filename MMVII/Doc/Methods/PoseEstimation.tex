
\chapter{Pose estimation, elementary method}


%-----------------------------------------------------
%-----------------------------------------------------
%-----------------------------------------------------

\section{Introduction}

This chapter present the "elementary" method, that compute the pose of an image
from observations that can, typically, be tie points (relative pose) or 
ground control point (relative or absolute pose).

By elementary we means algorithms that compute directly a solution for a limited
(typically $2$ or $3$, a few with tomasi-kanade) number of image. These algorithm
are the "tacticle" part, the result of the elementary algorithm will be used as
elementary part of the puzzle by more global method that will try to have a "strategic"  view.


%-----------------------------------------------------
%-----------------------------------------------------
%-----------------------------------------------------

\section{Space resection, calibrated case}


    %-----------------------------------------------------
\subsection{Introduction}

We deal here will the following problem  we have :

\begin{itemize}
   \item a calibrated camera ;
   \item a set of point for which we know the  $3d$ word coordinates $W_k$ and their 
        $2d$ coordinate $p_k$ in a image acquired with this camera;
\end{itemize}

We want to extract the pose $R,C$ of the camera (Rotation,center) such that for every point
we have the usual projection equation:

\begin{equation}
       \mathcal I(\pi (R*(W_k-C))) = p_k \label{EQ:PROJ}
\end{equation}


Each equation~\ref{EQ:PROJ}  creates $2$ constraints : one on line and one on column.
Also the pose $R,C$ has $6$ degrees of freedom, so we can expect that the problem has
a finite number of solution when we have $3$ points (in general less than $3$ will create
infinite number and more than $3$ no solution).

So here we will deal specifically  with the computation of the $R,C$ satisfying
exactly the equation~\ref{EQ:PROJ}  from a set of $3$ correspondance. Other
chapter will discuss of how we use more (possibly many) correspondance for beign 
robust to outlier (for
example with Ransac) or beign more accurate in presence of "gaussian" noise (for 
example with leas-square like) .


    %-----------------------------------------------------

\subsection{Putting things in equation}

           %  -  -  -  -  -  -  -  -  -  -  -  -  -  -  -
\subsubsection{Equivalence to local coordinates}

First, we remark that if we know the local coordinates $L_k$ of the points in the repair
of the camera, the problem becomes easy . Let show it, we will have to find a translation-rotation such that  :

\begin{equation}
       R*(W_k-C) = L_k  ; k\in{1,2,3} \label{SpResecEQ:WL}
\end{equation}


For each pair $k,k'$, we have (noting $\Vec{X}_{kk'} = \overrightarrow{X_{k}X_{k'}} $):


\begin{equation}
	R* \Vec{W}_{kk'} =  \Vec{L}_{kk'} \label{SpResecEQ:WLVECT}
\end{equation}

As  $R$ is a rotation we have $R(A \wedge  B) = R(A) \wedge   R(B) $ and then:

\begin{equation}
	R* \begin{bmatrix} \Vec{W}_{12} & \Vec{W}_{23} &  \Vec{W}_{12} \wedge \Vec{W}_{23} \end{bmatrix} 
        =  \begin{bmatrix} \Vec{L}_{12} & \Vec{L}_{23} &  \Vec{L}_{12} \wedge \Vec{L}_{23} \end{bmatrix} 
        \label{SpResecEQ:WLVECT}
\end{equation}

Equation~\ref{SpResecEQ:WLVECT} allows to compute $R$ by matrix inversion and multiplication, and 
once $R$ is known it can be injected in equation~\ref{SpResecEQ:WL} to compute $C$.

           %  -  -  -  -  -  -  -  -  -  -  -  -  -  -  -
\subsubsection{Equivalence to compute depth}

We need to compute the local coordinates . For this, we know for each point the bundle its belongs
to because we know internal calibration. Le $B_k$ be the bundle bearing the point $p_k$ We can write symbolicly :


\begin{equation}
	B_k =  \pi^{-1} (\mathcal I ^{-1} (p_k)) 
\end{equation}

Let write $B_k = (0,\Vec{u}_k)$,  we can compute $\Vec{u}_k$ with internal calibration and
as we know that $L_k \in B_k$ we can write  :


\begin{equation}
	L_k = \lambda_k \Vec{u}_k
\end{equation}

So to solve our problem it is sufficent to estimate $\lambda_k$.

           %  -  -  -  -  -  -  -  -  -  -  -  -  -  -  -
\subsubsection{Setting equations}

Now we remmber that rotation and translation are isometric, so we have conservation of
distances between  word an local coordinates. We can then write :

\begin{equation}
     D_{kk'}	||\Vec{W}_{kk'}||   = || \Vec{L}_{kk'} || = || \lambda_k \Vec{u}_k -  \lambda_{k'} \Vec{u}_{k'}|| \label{SpResecEQ:ConsDist}
\end{equation}


In equation~\ref{SpResecEQ:ConsDist}, the $D_{kk'}$ and  $\Vec{u}_k$ are know, so we have $3$ equation 
$3$ unkowns.



    %-----------------------------------------------------

\subsection{Solving $3$  equations with $3$ unknowns}

\subsubsection{Fix notation an supress 1 unknown}

We change notation to have lighter formula and to make them coherent with the {\tt C++} code of {\tt MMVII}
impplemating this resolution. We call $\Vec{A} , \Vec{B}, \Vec{C}$  the bundles, $D_{AB}$ the distances \dots
equations become :

\begin{equation}
	D_{AB} = || \lambda_a \Vec{A} -  \lambda_{b} \Vec{B} || \label{SpResecEQ:ABC}
\end{equation}

By doing quotient of equation~\ref{SpResecEQ:ABC} we have :

\begin{equation}
	\rho^A_{bc}  
	=	\frac{D^2_{AB}}{D^2_{AC}} 
	= \frac{|| \Vec{A} -  \frac{\lambda_{b}}{\lambda_{a}} \Vec{B} ||^2 }{||\Vec{A} -  \frac{\lambda_{c}}{\lambda_{a}} \Vec{C}||^2}
\end{equation}
\COM{
}


           %  -  -  -  -  -  -  -  -  -  -  -  -  -  -  -
\subsubsection{Supression another unknown}

In a firtst step we will write quotient betwenn equation


Supress

    %-----------------------------------------------------

\subsection{Notes for TD}

TD :

Begin with a calibrated camera w/o distortion,

From $3$ point compute 2 lamba with MMVII , then generate the rotations, solve ambiquity (with $4$ points)

From data with outlier (50\% on 50 point) use ransac like to make a robust init.




