\usepackage[english]{babel}
%\usepackage[english,french]{babel}
\usepackage{amsmath}
\usepackage{amsfonts}
\usepackage{rotating}
\usepackage{multicol}
\usepackage{color}
\usepackage{verbatim}
\usepackage{hyperref}
\usepackage{float}
\usepackage{listings}
%\usepackage{cprotect}
\usepackage{fancyvrb}
\usepackage{color}
%for subfigure using ?
\usepackage{graphicx}
\usepackage{caption}
\usepackage{subcaption}
\usepackage{makeidx}
%\usepackage{amsmath}
\usepackage{amssymb}


\makeindex


\DeclareMathOperator*{\argmax}{arg\,max}

%\usepackage{subfigure}

%\usepackage{mathabx}
%\usepackage{amstext}
%\usepackage{amssymb}
%\usepackage{ae}


\setcounter{tocdepth}{4}
\setcounter{secnumdepth}{4}


%allow to break on / in texttt
\usepackage{xparse}
\ExplSyntaxOn
\NewDocumentCommand{\replace}{mmm}
 {
  \marian_replace:nnn {#1} {#2} {#3}
 }
\tl_new:N \l_marian_input_text_tl
\cs_new_protected:Npn \marian_replace:nnn #1 #2 #3
 {
  \tl_set:Nn \l_marian_input_text_tl { #1 }
  \tl_replace_all:Nnn \l_marian_input_text_tl { #2 } { #3 }
  \tl_use:N \l_marian_input_text_tl
 }
\ExplSyntaxOff
\let\OldTexttt\texttt
\renewcommand{\texttt}[1]{\OldTexttt{\replace{#1}{/}{/\allowbreak}}}
\let\Oldtt\tt
\renewcommand{\tt}[1]{\Oldtt{\replace{#1}{/}{/\allowbreak}}}

%---------------------------------------------
\newcommand{\CPP}{\mbox{\tt C\hspace{-0.05cm}\raisebox{0.2ex}{\small ++} }}
\newcommand{\SiftPP}{\mbox{\tt Sift\hspace{-0.05cm}\raisebox{0.2ex}{\small ++} }}


\newcommand{\tran}{\ensuremath {^{t} }}
\newcommand{\trans}{\ensuremath {^{t} \!}}
\newcommand{\transs}{\ensuremath {^{t} \!\!}}
\newcommand{\transss}{\ensuremath {^{t} \!\!\!}}


\newcommand{\transL}{{\transs L}}
\newcommand{\transK}{{\transs K}}
\newcommand{\transX}{{\transs X}}
\newcommand{\transY}{{\transs Y}}

\newcommand{\transLm}{\ensuremath {\transs L^m}}
\newcommand{\transKm}{\ensuremath {\transs K^m}}
\newcommand{\KmtKm}{\ensuremath { K^m \, \transKm}}
\newcommand{\LmtLm}{\ensuremath { L^m \, \transLm}}

\newcommand{\KTH}{\ensuremath {^{th}}}
\newcommand{\EME}{\ensuremath {^{i\grave eme}}}
\newcommand{\ETer}{\ensuremath {\mathcal T}}
\newcommand{\EIm}{\ensuremath {{\mathcal I}_k}}
\newcommand{\EPx}{\ensuremath{{\mathcal E}_{px}}}

\newcommand{\FPx}{\ensuremath{{\mathcal F}_{px}}}

\newcommand{\Ok}{\ensuremath{{\mathcal O}_{k}}}

\newcommand{\Ess}{\ensuremath{{\mathcal E}}}

\newcommand{\DimPx}{\ensuremath{D_{px}}}

\newcommand{\PiI}{\ensuremath{\dot{\pi}}}
\newcommand{\PxMoy}{\ensuremath{\tilde{P_x}}}
\newcommand{\PxZone}{\ensuremath{P_x^Z}}

\newcommand{\LP}{\ensuremath{{\mathcal L_P}}}
\newcommand{\ChBit}{\ensuremath{{\mathcal B}}}

\newcommand{\RR}{\ensuremath{\mathbb{R}}}
\newcommand{\ZZ}{\ensuremath{\mathbb{Z}}}
\newcommand{\NN}{\ensuremath{\mathbb{N}}}
\newcommand{\Ind}{\ensuremath{\mathbb{I}^{nd}}}

\newcommand{\Ress}{\ensuremath{{\mathcal A}}}
\newcommand{\Reg}{\ensuremath{{\mathcal R}^{eg}}}
\newcommand{\Energ}{\ensuremath{{\mathcal E}}}
\newcommand{\Echant}{\ensuremath{{\mathcal E}}}
\newcommand{\PZero}{\ensuremath{{\mathcal P}^0}}
\newcommand{\SUn}{\ensuremath{{\mathcal S}^1}}

\newcommand{\DeltaI}{\ensuremath{\Delta^{\imath}}}

\newcommand{\DdSt}{\ensuremath{d^2}_{/\mathcal{S}^3}}
\newcommand{\DeuxExtre}{\ensuremath{\unrhd}}
%\newcommand{\DeuxExtre}{\ensuremath{\nabla}}
\newcommand{\RefFantome}{{\bf ?2Def?}}
\newcommand{\PourLecteurAverti}{{\Large \bf \emph{Ce paragraphe peut
facilement \^etre omis
en premi\`ere lecture.}}}
\newcommand{\COM}[1]

%  \verb|\|


\newcommand{\ELISE}
{\mbox{{\bf $\mathcal{E}$}\hspace{-0.15em}\raisebox{-0.4ex}{L}\hspace{-0.3em}\raisebox{0.3ex}{i}\raisebox{-0.4ex}{S}\raisebox{0.0ex}{e}}}

%\newcommand{\UNCLEAR}[1]{\textcolor{red}{\textbf{#1}}}
\newcommand{\UNCLEAR}{}
\newcommand{\ISITCLEAR}{}
\newcommand{\PPP}{MMVII}
\newcommand{\CdPPP}{{\tt MMVII}}
\newcommand{\MMVIDIR}{{\tt MMVII-MainFolder/}}
\newcommand{\doxy}{\emph{doxygen}}
\newcommand{\MMNONE}{NONE}

%---------------------------------------------
\begin{document}
\selectlanguage{english}

%\title{MicMac, Apero, Pastis and Other Beverages. The documentation!}
\title{Project 2007}
%\author{MPD}

\maketitle

\tableofcontents


%###########################################################################################################
%###########################################################################################################
%---------------------------------------------- PART I -----------------------------------------------------
%###########################################################################################################
%###########################################################################################################

\part{Generalities}

\section{Foreword}
In $2007$ I began to write some of MicMac documentation in French. Then,
for different reasons (laziness, lack of courage, idleness, \dots) I stopped.

In March $2011$, as preparing a course on my photogrammetric tools, I decided
to start again this documentation. I thought it would be useful to
do it in a language (hopefully) close to English.  This is the new
version you are reading. However, I doubt that it may be complete before a long time and,
during this transitional step, I will conserve the existing French chapter at the end
of this documentation and there may be some cross references between English and French chapters.



\chapter{Introduction}


%=======================================================================================================
\section{History, Status and Contributors}

This is the documentation of a set of software
photogrammetric tools that, under certain
conditions, allow to compute a 3D modelization from a
set of images.

MicMac is a tool for image matching. I began to write it in $2005$, while working at the
French National Geographic Institute (IGN), as
a tool integrating several recent results of the scientific community.
It is a general purpose tool, probably in many (if not all) specific
contexts, one will be able to find a more accurate tool. However, one
of its expected advantages is its generality. It has been used
in a lot of different contexts, for example:

\begin{itemize}
   \item digital terrain model in rural context from pairs of satellite
         images, with exact or approximate orientation;

   \item digital elevation model in urban context with high resolution
         multi-stereoscopic images;

   \item detection of terrain movements;

   \item 3D modelization of objects (sculptures) or interior and exterior scenes;

   \item multi-spectral matching of images registration.

\end{itemize}

Of course this generality comes with a price \dots : it requires a lot of parameterization
which sometimes turns to be quite complex.  For 3D computation, MicMac works only
with oriented images like the ones resulted from classical aero-triangulation process. Early
in $2007$, there were several opportunities that encouraged me to create a
tool that could orientate a set of overlapping images, so that they can be matched in
MicMac:

\begin{itemize}
   \item  I bought my first reflex digital camera, and thought it would be fun
          to be able to make 3D models from my holidays pictures, which turned to be right;

   \item  I discovered the existence of the magical SIFT algorithm from David Lowe, and thought this
          would make this idea feasible by solving the tie point problem, which turned to be right;

   \item  I had already written several pieces of software, including some calibration tools, which
          could be reused and made me think it could be done easily and quickly, which turned to be wrong \dots
\end{itemize}

Since $2008$, several tools were added to solve specific requirements: tools for ortho-photo, tools
for demosaicing\dots \  Since $2007$,  MicMac is an open source software, under the CeCILL-B license
(an adaptation to the French law of the L-GPL license); as far as I understand law (not very much) all
the other tools described in this document are extensions and evolutions of MicMac and obey to
the same license.

Different people have helped me in writing these tools:

\begin{itemize}
   \item Gregoire Maillet for supporting satellite orientation models (grid of rpc),
   \item Arnaud Le Bris for adaptation of \SiftPP supporting large images,
   \item Didier Boldo for the first Windows adaptation,
   \item Aymeric Godet and Livio de Luca for  developing two different user friendly interfaces and also making many tests,
   \item Christophe Meynard for solving some tricky Linux problems,
   \item Christian Thom for the first idea of multi-correlation,
   \item Jean-Micha\"el Muller for improvements about installation,
   \item Ana-Maria Rosu for many typo corrections (alas, I can create them faster than she can correct them).
   \item since september $2012$, the \emph{culture 3D} team : J\'er\'emie Belvaux, G\'erald Choqueux,
         Matthieu Deveau.
\end{itemize}

Of course, there are also many people who helped without knowing by creating
free softwares which I integrated:


\begin{itemize}
   \item    AMD  (\url{http://www.cise.ufl.edu/research/sparse/amd/}) for approximate minimum degree ordering;
   \item    SIFT
   \item    DCRAW by Dave Coffin ( \url{http://www.cybercom.net/~dcoffin/dcraw/}) for using raw image;
   \item    image magick for convert, for using jpg images (\url{http://www.imagemagick.org/});
   \item    proj4 for handling cartographic projection (\url{http://trac.osgeo.org/proj/}) ;
\end{itemize}


%===========================================================================
\section{Prerequisites}

These tools are low-level tools. Although I will try to make this
documentation as clear and self-contained as possible, there are some
prerequisites:

\begin{itemize}
   \item the reader must be comfortable with the Linux PC on which the software
         will be installed; at least, if you are not familiar with installation
         of software from source code, you should have the support of
         an administrator;

   \item some very basic notions of photogrammetry are necessary, not a lot
         (for example to have a notion of what cross-correlation,
         epipolar geometry, rotation matrix are).
\end{itemize}


%=======================================================================================================
\section{Installation and Distribution}

\subsection{Documentation}

This document is rather a "reference" documentation, not specially tuned for an
easy begining; for reference to photogrammetry with MicMac see :

\begin{itemize}
   \item \url{http://jmfriedt.free.fr/lm\_sfm\_eng.pdf} by JM Friedt;
   \item \url{http://forum-micmac.forumprod.com/bibliography-f38.html} a section of the forum dedicated to bibliography;
   \item several documents in the {\tt Documentation} folder and sub folder of MicMac distribution (see in
         particular {\tt pdf} file
         and {\tt Paper-Algo/ , Paper-MPD/ , Papers-Internship/ , Paper-UseCase/} sub folder)
   \item \href{http://combiencaporte.blogspot.fr/2013/10/micmac-tutoriel-de-photogrammetrie-sous.html}{micmac-tutoriel-de-photogrammetrie}\\*
\end{itemize}



There is also a forum where you will find answers to the most frequent question
you may have : \url{http://forum-micmac.forumprod.com/}



\subsection{Install}

\label{Install}
These tools are written in \CPP. They are distributed mainly in source code
format that you have to compile. As described below, they are relatively
low level tools, and the installation, computation and running of these
tools require some basic background in practical computer science.

See now the section~\ref{MERCURIAL} for the new website since $2012$, install mercurial and follow the instruction.
Several links that may be useful :

\begin{itemize}
   \item \url{http://logiciels.ign.fr/?Telechargement,20}   for the sources
   \item \url{http://forum-micmac.forumprod.com/new-site-of-micmac-since-januray-2013-t340.html}
   \item \url{http://forum-micmac.forumprod.com/tuto-install-t518.html}
\end{itemize}


In this documentation, you will find examples that require some
data.The server is now a {\tt ftp} server :

\begin{verbatim}

ftp2.ign.fr
user = micmac_user
pwd = scAEf9MR

\end{verbatim}




%=======================================================================================================

\section{Libraries, Programs and Dependencies}

These softwares have few dependencies to other libraries or programs. Basically,
if you use tiff or raw files as input, and if you do not use any of the graphical
tools provided \footnote{for handling mask, visualize tie points ...} ,
there might not be any dependency.

By the way, on Linux, as the graphical interface are by default required, the compiler
will require the header file {\tt X.h, Xlib.h, Xutil.h, cursorfont.h, keysym.h}.
If they are not installed, you can easily get them  with something like :


\begin{verbatim}
sudo apt-get install x11proto-core-dev libx11-dev
\end{verbatim}

Most users will want sooner or later to use jpeg files. In this case, it will be necessary
to have installed the command {\tt convert}, this command is a part of the excellent
{\tt ImageMagick} package.


The {\tt dcraw} source code I use to handle xif files info is sufficient in most cases.
However, when it fails, I try to use the {\tt exiv2} tool. I also recommend that
you install this excellent and free package.

I also recommend that you install the excellent package {\tt exiftool}, it is a free
open source package and has the ability to read many {\tt xif} information (including
GPS tags that will be soon usable in {\tt Apero}).




%=======================================================================================================
\section{Interface for the Tools}

\subsection{Kinds of Interfaces}
There are roughly three kinds of interfaces for softwares:

\begin{itemize}
   \item user friendly graphical interface, with intuitive menu and window etc.
         Its advantage is that it may be usable by all final users,
         the drawback of this solution being the cost for the developer;

   \item API or application programming interface. Using this level of interface requires you
         to use one of the programming language the API is functioning with. One of
         the drawbacks of these API is that they require a lot of documentation;

   \item a set of programs that you can call on a command line, with parameters being
         added on a command line or included in a file.

\end{itemize}

The tools described here use mainly the third kind of interface. This seemed to be the
optimal solution as these tools have been primarily developed for my own usage and
usage of colleagues from the same building. Since this is not optimal for end users, some user friendly graphical interfaces have been added, to help to set parameters.

\subsection{Simple Tools}

The tools described here are all command line tools. Their parameters can be added
directly on the command line or, for more complex tools (like Apero and MicMac) the
parameters are provided in an XML file.

Here is an example of calling the command {\tt GrShade} for computing the shading
of a depth image:

\begin{verbatim}
bin/GrShade ../micmac_data/Boudha/F050_IMG_5571_MpDcraw8B_GB.tif Visu=1 FZ=0.1
\end{verbatim}

The simple tools described here, that have all their parameters on command lines,
include:

\begin{itemize}
    \item {\tt bin/GrShade} for computing shading;
    \item {\tt bin/Nuage2Ply} for transforming depth map in cloud point in ply format;
    \item {\tt bin/ScaleIm} for rescaling an image (\UNCLEAR{with some care on} aliasing); %dealing with
    \item {\tt bin/ScaleNuage} for scaling a depth map.
\end{itemize}

Generally, these tools understand the syntax {\tt bin/Tool -help} that prints the syntactic
description of the command. For example{ \tt bin/Nuage2Ply -help} will print on the terminal:

\begin{verbatim}
*****************************
*  Help for Elise Arg main  *
*****************************
Unamed args :
  * string
Named args :
  * [Name=Sz] Pt2dr
  * [Name=P0] Pt2dr
  * [Name=Out] string
  * [Name=Scale] REAL
  * [Name=Attr] string
  * [Name=Comments] vector<std::string>
  * [Name=Bin] INT
  * [Name=Mask] string
  * [Name=Dyn] REAL
\end{verbatim}


This indicates that {\tt bin/Nuage2Ply} has one mandatory argument, of type string;
mandatory arguments come first and the order matters. { \tt bin/Nuage2Ply}  also admits
several optional arguments. For example, there is one optional argument named  { \tt Scale},
of type  { \tt REAL}. If this argument is to be specified with the value $2.5$,  the command
line will contain {\tt Scale=2.5}.  Of course the command  {\tt bin/Tool -help} gives information
essentially on the syntactic aspect, the semantic has to be found in this documentation
(when the chapter exists \dots).

\subsubsection{GUI for command line tools}

For each command line tool, a graphical interface can be launched to help setting parameters. To run this interface, one should replace command name in command line by {\tt mm3d + "v" + command}. For example, to set parameters for command  {\tt GrShade}, one should call:

\begin{verbatim}
bin/mm3d vGrShade
\end{verbatim}

This will raise an interface where parameters can be set, and where all available options are shown:

\begin{figure}[!h]
\centering
\includegraphics[width=87.31mm]{FIGS/Saisie/Visual.jpg}
\caption{Visual interface for command line tools}
\end{figure}

\subsection{Complex Tools}

For a more complex command, that requires arbitrary numbers of arguments, the command line
would not be manageable. For this command, it has been decided to use an {\tt XML} file
for specifying the parametrization. {\tt XML} has the following advantages:


\begin{itemize}
    \item it is a standard, with \UNCLEAR{current} specialized editor;
    \item the name tagging convention, although heavy for writing, make it easier to read;
    \item it allows textual description of attributed tree structures, which is
          exactly what is required for complex parametrization.
\end{itemize}

Here is an example for calling {\tt Apero}:

\begin{verbatim}
bin/Apero ../micmac_data/Ref-Apero/Test-Lion/AperoQuick.xml
\end{verbatim}

If you downloaded the data example as described in~\ref{Install}, you could have
a look at the {\tt AperoQuick.xml} file to see what it looks like. For all these
complex tools, that exit in an  XML file, there is a formal description of the
XML file that is correct from the syntactic point of view. These XML formal description
files are all  located in the {\tt include/XML\_GEN/} directory.
For example, the file {\tt include/XML\_GEN/ParamApero.xml} contains a formal
description of the XML files which are syntactically valid XML files for the
{\tt Apero} program.

How the formal files are used to specify the valid files is too complex
to describe it here. The mechanism is described in chapter~\ref{Mic:Tree:Match}.
Basically, the idea is that the parameter file
must be a sub-tree of the specification file satisfying some arity
constraints.


Generally, the {\tt XML} file can be modified using optional command line
arguments. For example, you can run one of the example data set with no
argument:

\begin{verbatim}
bin/MICMAC /home/mpd/micmac_data/Jeux1-Spot-Epi/Param-0-Epi.xml
\end{verbatim}

But if, for some reason, you want to start the computation directly
from the second step you shall add an optional argument and type:

\begin{verbatim}
bin/MICMAC /home/mpd/micmac_data/Jeux1-Spot-Epi/Param-0-Epi.xml  FirstEtapeMEC=2
\end{verbatim}


\subsection{Where Calling the Tools From (The Mandatory Working Directory)}

At the beginning of {\tt MicMac}, it was mandatory to run the file from the
{\tt micmac} directory. This is why in all the examples you will see
commands like {\tt bin/MICMAC \dots}. As I had a lot of complaints
about this not being very convenient, I have corrected this fact for most
of the tools. However, I do not guarantee that this has been corrected
everywhere. So if you encounter problems, you should try to run the file
from the {\tt micmac} directory.



%=======================================================================================================

\section{Data Organization and Communication}

When you want to use photogrammetric tools for complex tasks, there are a lot
of things about data organization that has to be specified to programs.
For example:

\begin{itemize}
   \item  at a given step, you want to orientate a certain subset of images of a
          project; so you need  to have the possibility to specify sets
          and subsets of files;

   \item  sometimes you will want to specify that if an image name is
          {\tt toto\_123.tif} or  {\tt toto\_0123.tif}  then the associated orientation is
          {\tt 123\_tata.xml}; so you need to have the possibility to specify the probably complex
          rules of computation that transform strings to strings;

   \item  sometimes you will want to specify that a matching process (for example
          tie points computation) must be executed between all pairs of images satisfying
          certain conditions; so  you need to have the possibility to specify relations
          (in the mathematical way).
\end{itemize}

All these tasks may be performed by a database management system. Although there
are some very efficient systems such as open source systems, this is not what I chose
for supporting this functionality (because I wanted my tools to stay relatively autonomous).
Maybe it was not a good choice, however it has to be assumed now.


The precise mechanism is quite complex and it is described in the chapter~\ref{Chap:NFS}.
The main ideas are:

\begin{itemize}
   \item there is a huge use of modern regular expressions to specify string sets and
         string manipulation. For example, the pattern {\tt Img([0-9]\{4\}).tif} will
         describe the name set beginning with {\tt Img}, followed by four digits and ending
         with {\tt .tif}. If one wants to specify that the file associated  to {\tt Img1234.tif}
         is {\tt Ori/1234-HH.xml}, there will be something like {\tt  Ori/\$1-HH.xml} associated
        to {\tt Img([0-9]\{4\}).tif}  (the meaning is that {\tt \$1}  is to be replaced by
          the first sub-expression between parenthesis);



   \item to facilitate the sharing of sets, transformations, relations \dots between programs,
         generally they are not manipulated directly. They are created in a common file and
         are given a name (or Key). The program refers to these objects by their key which
         facilitates name convention sharing. For example, if the transformation
         {\tt Img([0-9]\{4\}).tif $\rightarrow$  Ori/\$1-HH.xml} is to be used
         to describe the association between image and orientation, it  may be declared in
         the file {\tt MicMac-LocalChantierDescripteur.xml} under the key {\tt Key-Im2Ori},
         this key will then be used in {\tt Apero} for the creation of orientation file and in
         {\tt MicMac} for using the result of {\tt Apero};


     \item a lot of pre-existing conventions are automatically loaded by the tools, and
           for most of the cases these standard conventions should be sufficient.
\end{itemize}

\UNCLEAR{For example} %il n'y a pas d'exemple


%=======================================================================================================

\section{Existing Tools}


\label{Gen:ExTools}

The pipeline for transforming a set of images in a 3D model, and optionally
generating ortho-photo, is made essentially of four
"complex" tools:


\begin{itemize}
   \item {\tt \bf Pastis}. In fact, this tool is no more than an interface to the
         well known \SiftPP, distribution of {\tt Sift}, there is no algorithmic
         added value. Its advantage is to integrate the tie point generation
         in a way compatible with the global pipeline;

   \item {\tt \bf Apero} starts from tie points generated by {\tt \bf Pastis},
         and optional complementary measurements, and generates external and internal
         orientations compatible with these measurements;

   \item {\tt \bf MicMac} starts from orientation generated by {\tt \bf Apero} and
         computes image matching;


   \item {\tt \bf Porto}  starts from individual rectified images, that have been
         optionally generated by  {\tt \bf MicMac}, and generates a global ortho-photo;
         this tool is still in a very early stage.
\end{itemize}

There are several auxiliary tools that may be helpful for importing or
exporting data at different steps of this pipeline:


\begin{itemize}
   \item {\tt \bf BatchFDC} for batching a set of commands;

   \item {\tt \bf Casa} for computing analytic surface (cylinder \dots),
                        from points cloud, very early stage;

   \item {\tt \bf ClipIm} for clipping image;

   \item {\tt \bf ConvertIm} for some image conversion;

   \item {\tt \bf Dequant} for quantifying an image;

   \item {\tt \bf GrShade} for compute shading from depth image;

   \item {\tt \bf MapCmd} transforms a command working on a single file in a command
         working on a set of files;

   \item {\tt \bf CpFileVide}   to complete

   \item {\tt \bf MpDcraw}, an interface to the great {\tt dcraw} offering some low-level
         service useful for image matching;

   \item {\tt \bf MyRename} for image renaming, using modern regular expression and giving the possibility
         to integrate xif data in the new name, tricky but necessary in the existing pipeline;

    \item {\tt \bf Nuage2Ply}, a tool to convert depth map in point cloud;

   \item {\tt \bf SaisieMasq}, a user friendly (compared to others \dots) tool to create
         mask upon an image;

   \item {\tt \bf ScaleIm}, tool for scaling image;

   \item {\tt \bf ScaleNuage}, tool for scaling internal representation of point cloud;

   \item {\tt \bf tiff\_info}, tool for giving information about a tiff file;

   \item {\tt \bf to8Bits}, tool for converting $16$ or $32$ bit image in a $8$ bit image.

   \item {\tt \bf SupMntIm}, tool for generating a superposition of image and MNT in hypsometry
         and level curves.

   \item {\tt \bf PanelIm}. Gather images in a panel.
\end{itemize}










\chapter{Project management command}

%---------------------------------------------
%---------------------------------------------
%---------------------------------------------
\section{Readible file formats}

\subsection{Readible/Binary}

As photogrammetric pipeline is a complex process, made of several computation,
the result of each computation has to be writen in some files that
will be read at by next process. There is basically two familly of such files:

\begin{itemize}
   \item files that may have some interest to  be read or manipulated by human
         or other programm, in this case {\tt MMVII} standar tagged format
	 as {\tt xml} of {\tt json}; example of such file are calibration
	 or pose estimation;

   \item files that have probably low interest for human, and for efficiency 
         they are stored in binary format  (note that there exist also a
         text version of this binary format, note easy to read, but that facilitate
	   import export);
\end{itemize}

\subsection{Xml and Json files}

{\tt MMVII} offer the possibility to export data in two different tagged format : {\tt xml} and
{\tt json}. By default it's {\tt xml}, but this can be changed using mecanism descrined in \ref{UserParametrisation}.
Implicit conversion may appear in a near future for sharing files between user,
by the way it's recommanded that user make a choice once for all  to avoid any problem.

The folder {\tt MMVII-UseCaseDataSet/SampleFiles} contains examples
of files in {\tt xml} and {\tt json}. 

Note that {\tt xml} allows real coments and, for example, in the file {\tt Calib...xml}, they are used to
explicit  the meanin of distorsion parameters. As {\tt json} do not have
real comments, we use the special tags , for example in file
{\tt Calib...json} you can find {\tt "<!--comment6-->":"(X,0)"}
(corresponding to  {\tt <!--(X,0)-->} in {\tt xml}).

These file are made to be relatively easy to read. They can also be created or
modified easily by user or programm however there are some \emph{strict} rules to observe so that such
file remain valid . Starting from a valid file, here are example of things that can be done to maintain
validity :

\begin{itemize}
       \item change the value of atoms while respecting their type (float, int, string \dots);

       \item add or supress an element  in a sequence or a map,  see for example  the file
	       {\tt TestObj.*}, the element of sequence are tagged {\tt el} in {\tt xml}, while the element of
		pair key-value  in a map are tagged {\tt K/V};

       \item supress or add any comment;

       \item supress or add an optionnal value (rare case for now), the optionnal value can
	     be  detected as their tags begin by {\tt Opt:}, see file {\tt F\_T2.*};
\end{itemize}

And here a \emph{non-exhautive} list of thing you cant do :

\begin{itemize}
        \item obviously create a non-valide xml/json file;

	\item swap two elements of different tags  \emph{very bad, dont do it, strictly forbiden, naughty , Micmac is whatching you \dots}
          also it woul be theoretically possible to recover the information, this would involve unnecessary software devlopment
	  that wont be done;

        \item supress a non optional tag ;

	\item add/supress a value in fixed size tab (used to represent point for exeample);
\end{itemize}

The format of each file is :

\begin{itemize}
	\item a single root node , named {\tt root} in {\tt xml} and anonymous in Jason,
	\item root node has exacly $3$ sub-node that have  a fixed tag

	\item the first node is the type, it must be tagged {\tt Type} and its value must {\tt MMVII\_Serialization} 
              signing the fact that is was created by {\tt MMVII};

      \item the second node is a version number, it must be tagged {\tt Version}, its value will allow
	    compatibility policy with older files, unused at the time being;
             
    \item the third node must be tagged {\tt Data} and contains in fact the data itself !
          in most frequent case, it will contain a single node (see {\tt Calib*, Ori*} ) allowing
          some type-checking at the very begining   of the command (i.e we can check that {\tt Calib*}
          are most probably  MMVII-calibration files as their data contains the single tag {\tt InternalCalibration});
\end{itemize}


%---------------------------------------------
%---------------------------------------------
%---------------------------------------------

\section{User specific parametrization}

\label{UserParametrisation}

Sometimes user need to fix some stable default parametrization of {\tt MMVII}, 
the parameters acessible for now are :

\begin{itemize}
    \item maximal number of processor that will be allowed when  {\tt MMVII} execute
          parallel computation;

     \item default format for human readible export, it can be for now \emph{xml} or \emph{json};

      \item name of the user, this field is for now rather target for devloppers who want to
            include in this devlopped code   some message/test ...  specific to himself
           (for example, in some suspicipus case, I will make a breakpoint for myself, but will
		not bother others with that as it should work if we are a bit lucky);
\end{itemize}

This will probably evolve, for example we can imagine to have some category of user.
The way this is done is done by filling a {\tt xml} file located in the folder 
{\tt MMVII-LocalParameters}, for example we have the file {\tt Default/MMVII-UserOfProfile.xml} :

\begin{verbatim}
<?xml version="1.0" encoding="ISO8859-1" standalone="yes" ?>
<Root>
   <Type>"MMVII_Serialization"</Type>
   <Version>"0.00"</Version>
   <Data>
      <UserName>"Uknown"</UserName>
      <NbProcMax>1000</NbProcMax>
      <SerialMode>"xml"</SerialMode>
   </Data>
</Root>
\end{verbatim}

Now the question is how will {\tt MMVII} locate file to use. This name of the folder containing
this file will be contained in a file {\tt MMVII-CurentPofile.xml} if it exists,
or in {\tt Default-MMVII-CurentPofile.xml} in the other case. This file should always exist
at it is a git-shared file {\emph should not ne modified except by devlopers}.
If we take a look at  it : 

\begin{verbatim}
<?xml version="1.0" encoding="ISO8859-1" standalone="yes" ?>
<Root>
   <Type>"MMVII_Serialization"</Type>
   <Version>"0.00"</Version>
   <Data>
      <NameProfile>"Default"</NameProfile>
   </Data>
</Root>
\end{verbatim}

We see that the only meaningfull part is {\tt NameProfile} indicating the name
of the folder. For creating a profile, what user must do is :

\begin{itemize}
    \item creat a file  {\tt MMVII-CurentPofile.xml} if it was not already done;
    \item fill the field {\tt <NameProfile>};
    \item create in the corresponding folder, a file {\tt MMVII-UserOfProfile.xml}
\end{itemize}

The idea, is that a user can have several predefined profile in different folders,
and only modify  {\tt MMVII-CurentPofile.xml}.


%---------------------------------------------
%---------------------------------------------
%---------------------------------------------

\section{General data organization}

The data organization in {\tt MMVII} is the following :

\begin{itemize}
     \item for a given project, almost all the file created are located somewhere under
           the same folder {\tt MMVII-PhgrProj},  this is necessary because during a complex 
           photogrammetric process many files will be created and we dont want to encumber
           the main folder;

     \item an example of such folder (resulting from the coded-target usecase) can be
	     found under {\tt MMVII-UseCaseDataSet/SampleFiles}

     \item for each kind of processing, there exist a subfolder corresponding to the "nature"
            of the data stored;

    \item  in the example there is of folder for orientation {\tt Ori}, one for points 3d coordinates
    {\tt ObjCoordWorld}, one for point measurement
	   {\tt ObjMesInstr}, one fore handling meta data {\tt MetaData}, one for storing reports
           {\tt Reports}, all the name of this subfolder are defined by {\tt MMVII} and cannot be changed
            by the user;

    \item there is (will be) many other king of folder : homologous point, radiometric model, radiometric data,
           \dots

     \item in each of the predefined folder, there exist different subfolder corresponding to different step
           of the process; the name of this subfolder are specified by user, sometime as input to a command,
	   sometime as output to a command; typically the output a command being the input of the next command;
\end{itemize}

In this example, we have $4$ different folder for orientation in {\tt Ori} :

\begin{itemize}
      \item  {\tt 11P} which is the result of initial estimation using uncalibrated spaced resection
	      (with the "$11$ parameters method");

      \item  {\tt Resec} which is the result of pose estimation using calibrated camera, using as input
	      the calibration stored in {\tt 11P};

      \item  {\tt BA} which is the result of pose estimation using bundled adjusment, using as input
	      for initial value the pose stored in {\tt Resec};

      \item   the refined pose of  {\tt BA} are used as input to drive a research of uncoded target with
	      accurate initial position in imahe;
 
       \item using the additional target, {\tt BA} is used as input to a new bundle adjusment and the result is
             stored in {\tt BA2}.

\end{itemize}


%---------------------------------------------
%---------------------------------------------
%---------------------------------------------


\section{Help command}
\index{Help}

\label{HelpCmd}


\section{Bench command}


\begin{itemize}
    \item {\tt  MMVII Bench 2 }  : standard mode  for execuring all bench at level 2

    \item {\tt MMVII Bench 2 PatBench=.*Der.* Show=0}  : execute benches matchin {\tt ".*Der.*"},
          {\tt  Show} is explicite as, by default, it is set to {\tt  true} 
          when {\tt  PatBench} is set;
   
    \item {\tt MMVII Bench 1 PatBench=XXX} : pattern specified but no match, print all benche existing


    \item {\tt MMVII Bench 2 KeyBug=Debord\_M1 }  : force the generation of  a given error

    \item {\tt MMVII Bench 1 KeyBug=XXXX }  : will print all possible value for explicit error generation


    \item {\tt MMVII Bench 1 PatBench=InspectCube }  : as InspectCube is not a bench function, but 
         only print information, exact name must be set with {\tt PatBench}

\end{itemize}






{\tt MMVII Bench 2 KeyBug=XXX }  : standard mode  for execuring all bench at level 2

{\tt MMVII Bench 1 PatBench=MemoryOperation KeyBug} : pattern specified but no match, print all benche existing




\part{Methodologies}

\chapter{Pose estimation, elementary method}

%-----------------------------------------------------
%-----------------------------------------------------
%-----------------------------------------------------

\section{Introduction}

This chapter present the "elementary" methods, that compute the pose of an image
from observations that can, typically, be tie points (relative pose) or 
ground control points (relative or absolute pose).

By elementary we mean algorithms that compute directly a solution for a limited
(typically $2$ or $3$, a few with Tomasi-Kanade) number of images. These algorithms
are the "tactical" part. The result of these elementary algorithms will be used as
elementary part of the puzzle by more global method that will try to have a "strategic" view.


%-----------------------------------------------------
%-----------------------------------------------------
%-----------------------------------------------------

\section{Space resection, calibrated case}

\label{SR_Cal}

    %-----------------------------------------------------
\subsection{Introduction}

We deal here with the following problem, we have:

\begin{itemize}
   \item a calibrated camera;
   \item a set of points for which we know the  $3d$ word coordinates $G_k$ and their 
        $2d$ coordinates $p_k$ in a image acquired with this camera;
\end{itemize}

We want to extract the pose $R,C$ of the camera (Rotation, Center) such that for every point
we have the usual projection equation:

\begin{equation}
       \mathcal I(\pi (R*(G_k-C))) = p_k \label{EQ:PROJ}
\end{equation}


Each equation~\ref{EQ:PROJ}  creates $2$ constraints : one on line and one on column.
Also the pose $R,C$ has $6$ degrees of freedom, so we can expect that the problem has
a finite number of solution when we have $3$ points (in general less than $3$ will create
infinite number and more than $3$ no solution).

So here we will deal specifically  with the computation of the $R,C$ satisfying
exactly the equation~\ref{EQ:PROJ}  from a set of $3$ correspondance. Other
chapter will discuss of how we use more (possibly many) correspondance for beign 
robust to outlier (for
example with Ransac) or beign more accurate in presence of "gaussian" noise (for 
example with least-square like) .


This problem occurs in two case in photogrammetric pipeline :

\begin{itemize}
   \item the first case is in fact rather rare, it's the case where we have "real" GCP
         an approximate calibration, and want to compute the initial pose of a camera ;
         it's rare because the requirement of having at least $3$ GCP visible is rather
         high; by the way it can occurs in calibrating a camera on a calibration fields
         with many GCP;

 \item the second case is when we have a set of $N$ already oriented camera  ($N\geq 2 $),
       we want to orientate a new camera \emph{relatively} to this set, we have a
       set of $M \geq 3$ multiple tie point , each tie point being visible in the new
       camera and least two of already oriented camera;  when it is the case,
       we can compute for each tie point, by bundle intesection, an estimation of the
       "ground" coodinate of the point in the current system, and then estimate a pose
       for the new camera coherent with the existing one;
       this case is extremely frequent in automatic photogrammetric
       pipeline (at least will be in {\tt MMVII});

\end{itemize}


The {\tt MMVII} code corresponding to this section can be found in 
{\tt PoseEstim/CalibratedSpaceResection.cpp}.

    %-----------------------------------------------------

\subsection{Putting things in equation}

           %  -  -  -  -  -  -  -  -  -  -  -  -  -  -  -
\subsubsection{Equivalence to local coordinates}

\label{SpRes:EquivLocCoord}

First, we remark that if we know the local coordinates $L_k$ of the points in the
camera frame, the problem becomes easy. Let show it, we will have to find a translation-rotation such that:

\begin{equation}
       R*(G_k-C) = L_k  ; k\in{1,2,3} \label{SpResecEQ:WL}
\end{equation}


For each pair $k,k'$, we have (noting $\Vec{X}_{kk'} = \overrightarrow{X_{k}X_{k'}} $):


\begin{equation}
	R* \Vec{G}_{kk'} =  \Vec{L}_{kk'} \label{SpResecEQ:WLVECT}
\end{equation}

As  $R$ is a rotation we have $R(A \wedge  B) = R(A) \wedge   R(B) $ and then:

\begin{equation}
	R* \begin{bmatrix} \Vec{G}_{12} & \Vec{G}_{23} &  \Vec{G}_{12} \wedge \Vec{G}_{23} \end{bmatrix} 
        =  \begin{bmatrix} \Vec{L}_{12} & \Vec{L}_{23} &  \Vec{L}_{12} \wedge \Vec{L}_{23} \end{bmatrix} 
        \label{SpResecEQ:WLVECT}
\end{equation}

Equation~\ref{SpResecEQ:WLVECT} allows to compute $R$ by matrix inversion and multiplication, and 
once $R$ is known it can be injected in equation~\ref{SpResecEQ:WL} to compute $C$.

           %  -  -  -  -  -  -  -  -  -  -  -  -  -  -  -
\subsubsection{Equivalence to compute depth}

We need to compute the local coordinates. For this, we know for each point the bundle its belongs
to because we know internal calibration. Let $B_k$ be the bundle bearing the point $p_k$. We can write:


\begin{equation}
	B_k =  \pi^{-1} (\mathcal I ^{-1} (p_k)) \label{SpResecEQ:DefBundle}
\end{equation}



Let $B_k = (0,\Vec{u}_k)$,  we can compute $\Vec{u}_k$ with internal calibration.
As we know that $L_k \in B_k$ we can write:


\begin{equation}
	L_k = \lambda_k \Vec{u}_k \label{SpResecEQ:DefLambda}
\end{equation}

So to solve our problem it is sufficent to estimate $\lambda_k$.

           %  -  -  -  -  -  -  -  -  -  -  -  -  -  -  -
\subsubsection{Setting equations}

Now we remember that rotation and translation are isometric, so we have conservation of
distances between  word an local coordinates. We can then write :


\begin{equation}
  D_{kk'}=||\Vec{G}_{kk'}||   = || \Vec{L}_{kk'} || = || \lambda_k \Vec{u}_k -  \lambda_{k'} \Vec{u}_{k'}|| \label{SpResecEQ:ConsDist}
\end{equation}


In equation~\ref{SpResecEQ:ConsDist}, the $D_{kk'}$ and  $\Vec{u}_k$ are knowns, so we have $3$ equation  for
$3$ unkowns, so far so good \dots 


    %-----------------------------------------------------

\subsection{Solving $3$  equations with $3$ unknowns}

\subsubsection{Fix notation an supress 1 unknown}

We change notation to have lighter formula and to make them coherent with the {\tt C++} code of {\tt MMVII}
implemating this resolution. We call $\Vec{A} , \Vec{B}, \Vec{C}$  the bundles, $D_{AB}$ the distances, \dots
equations become:


\begin{equation}
	D_{AB} = || \lambda_a \Vec{A} -  \lambda_{b} \Vec{B} || \label{SpResecEQ:ABC}
\end{equation}

By doing quotient of equations ~\ref{SpResecEQ:ABC} we have:

\begin{equation}
	\rho^A_{bc}  
	=	\frac{D^2_{AB}}{D^2_{AC}} 
	= \frac{|| \Vec{A} -  \frac{\lambda_{b}}{\lambda_{a}} \Vec{B} ||^2 }{||\Vec{A} -  \frac{\lambda_{c}}{\lambda_{a}} \Vec{C}||^2}
	\;\;
	\rho^C_{ba}  
	=	\frac{D^2_{CB}}{D^2_{CA}} 
	= \frac{|| \frac{\lambda_{c}}{\lambda_{a}} \Vec{C} -  \frac{\lambda_{b}}{\lambda_{a}} \Vec{B} ||^2 }
	       {||\Vec{A} -  \frac{\lambda_{c}}{\lambda_{a}} \Vec{C}||^2}
	       \label{SpResecEQ:Lambda}
\end{equation}

Now we see that in equation~\ref{SpResecEQ:Lambda}, only the ratio $\frac{\lambda_{b}}{\lambda_{a}}$ and $\frac{\lambda_{c}}{\lambda_{a}}$
matters, so one can solve them a system of $2$ equation with $2$ unknowns. Once we will have computed these ratio the triangle
$(\Vec{A} , \frac{\lambda_{b}}{\lambda_{a}} \Vec{B},  \frac{\lambda_{c}}{\lambda_{a}} \Vec{C})$ will be propoprtionnal the
triangle $(G_A,G_B,G_C)$ and it will be sufficient to compute this proportion.


           %  -  -  -  -  -  -  -  -  -  -  -  -  -  -  -
\subsubsection{Supression another unknown}

We fix the notation for ratio :

\begin{equation}
	\frac{\lambda_{b}}{\lambda_{a}} = 1+ b  \;\;
	\frac{\lambda_{c}}{\lambda_{a}} = 1+ c
\end{equation}


The $1$ in $1+b$, is to be coherent with {\tt C++} implementation where this makes the system more stable numerically.
Also to be coherent with {\tt C++} implementation, we will make the hypothesis that bundles have been normalized :

\begin{equation}
	|| \Vec{A}|| = || \Vec{B}|| = || \Vec{C}|| = 1
\end{equation}

Equations ~\ref{SpResecEQ:Lambda} rewrites :

\begin{equation}
	\rho^A_{bc}  
	= \frac{|| \Vec{A} -  (1+b) \Vec{B} ||^2 }{||\Vec{A} - (1+c) \Vec{C}||^2}
	= \frac{|| \overrightarrow{AB} +  b \Vec{B} ||^2 }{||\overrightarrow{AC}  + c \Vec{C}||^2}  \label{SpResecEQ:EqFracABC}
\end{equation}
\begin{equation}
	\rho^C_{ba}  
	= \frac{|| (1+c) \Vec{C} -  (1+b) \Vec{B} ||^2 } {||\Vec{A} -  (1+c) \Vec{C}||^2}
	= \frac{|| \overrightarrow{BC} + c \Vec{C} -  b \Vec{B} ||^2 } {||\overrightarrow{AC}  + c \Vec{C}||^2}
	 \label{SpResecEQ:EqFracCBA}
\end{equation}

Now we can write equation~\ref{SpResecEQ:EqFracABC} :

\begin{equation}
	b^2 + 2 b \overrightarrow{AB}. \Vec{B} + (||AB||^2 -   \rho^A_{bc} ||\overrightarrow{AC}  + c \Vec{C}||^2)
	\label{SpResecEQ:Pol2B}
\end{equation}

Equation~\ref{SpResecEQ:Pol2B} can be see as $2d$ degree polynom in $b$, and we can express $b$ as function of $c$:

\begin{equation}
	b = -\overrightarrow{AB}. \Vec{B} \pm \sqrt{Q(c)}  = -\overrightarrow{AB}. \Vec{B} + \epsilon \sqrt{Q(c)} 
	 \;\;\;  \epsilon \in \{-1,1\}  \label{SpResecEQ:SolEqD2}
\end{equation}

With $Q(c)$ being a $2$ degree polynom in $c$:
\begin{equation}
	Q(c) = (\overrightarrow{AB}. \Vec{B})^2 -  (||AB||^2 -   \rho^A_{bc} ||\overrightarrow{AC}  + c \Vec{C}||^2)
\end{equation}

           %  -  -  -  -  -  -  -  -  -  -  -  -  -  -  -
\subsubsection{Resolving last unknown}

We can now use equation~\ref{SpResecEQ:SolEqD2} to substituate $b$  in equation~\ref{SpResecEQ:EqFracCBA}

\begin{equation}
	\rho^C_{ba}   ||\overrightarrow{AC}  + c \Vec{C}||^2
	= || \overrightarrow{BC} + c \Vec{C} -  (-\overrightarrow{AB}. \Vec{B} + \epsilon \sqrt{Q(c)}) \Vec{B} ||^2 
	\label{SpResecEQ:EqcInit}
\end{equation}

We note $R(c)$ the $2d$ degree polynom in $c$ defined by :

\begin{equation}
   R(c) =   \rho^C_{ba}   ||\overrightarrow{AC}  + c \Vec{C}||^2
        - Q(c)
        - ||\overrightarrow{BC} + c \Vec{C} + (\overrightarrow{AB}. \Vec{B}) \Vec{B} ||^2
\end{equation}

We note $L(c)$ the $1d$ degree polynom in $c$ defined by :

\begin{equation}
	L(c)=   ( \overrightarrow{BC}.\Vec{B} + \overrightarrow{AB}. \Vec{B} +   c \Vec{C}.\Vec{B}  ) 
\end{equation}

Then equation~\ref{SpResecEQ:EqcInit} writes :

\begin{equation}
	R(c) = 2 \epsilon L(c)  \sqrt{Q(c) }   \label{SpResecEQ:RL}
\end{equation}

Squaring equation~\ref{SpResecEQ:RL} we obtain a $4$ degree polynomial equation :

\begin{equation}
	R(c) ^2 - 4 L(c)^2  Q(c) = 0   \label{SpResecEQ:D4}
\end{equation}

We just need to solve equation~\ref{SpResecEQ:D4} to get possible values of $c$, 
then get values of $b$ using~\ref{SpResecEQ:SolEqD2}, 
then get depth $\lambda _k$, then get local coordinates $L_k$ using~\ref{SpResecEQ:DefLambda} 
and~\ref{SpResecEQ:DefBundle}, and finally finally compute the pose $R,C$  going back to~\ref{SpRes:EquivLocCoord}.

%-----------------------------------------------------
%-----------------------------------------------------
%-----------------------------------------------------

\section{Space resection, uncalibrated case}

    %-----------------------------------------------------

\subsection{Introduction}

We deal here with the following problem,  we have :

\begin{itemize}
   \item an \emph{un-calibrated} camera ;
   \item a set of point for which we know the  $3d$ word coordinates $G_k$ and their 
        $2d$ coordinate $p_k$ in a image acquired with this camera;
\end{itemize}


We want to extract \emph{simulatenously} the pose $R,C$ of the camera (Rotation,center)  and the calibration
$\mathcal I$ such that for every point we have the usual projection equation:

\begin{equation}
       \mathcal I(\pi (R*(G_k-C))) = p_k \label{EQ:PROJ}
\end{equation}

Obviously this problem require more informations ~\ref{SR_Cal}, because we want to estimate more parameters.
Also, due to correlation between internal and external parameters, the solution
are frequently not so stable. It usage is far less current than calibrated case,
to our knowledge, the two cases where it appears practically are :


\begin{itemize}
   \item approximate  orientation of old images for wich we have no idea of focal length and/or principal
        point; one advantage of the method, as it reestimates the principal point, is that it can work
        also with croped images (current case for historical images);

   \item for forcing conversion of  non central perspective sensor to a central perspective sensor
        using  artficial $2d-3d$ correspondances; also theoretically not recommanded~\footnote{it's
        always better to use rigourous modelization when we can}, it can unlock situation where one needs to
        use a software that accept only central perspective sensor; this is the case for example
        with satellite images, if one use "small" patches, then the projection  function can be
        approximate by a central perspetive on this "small" patch ~\footnote{In fact, the accuracy of
        the approximation is a difficult question, depend of the size of the patch, of relief, 
        of the exact sensor \dots}.
\end{itemize}

The {\tt MMVII} code corresponding to this section can be found in 
{\tt PoseEstim/UnCalibratedSpaceResection.cpp}.
    %-----------------------------------------------------

\subsection{Hypothesis}

As we want also to estimate the distorsion, we will have to make some hypothesis
and select a model.  We will make the hypothesis that the distorsion is purely linear, so using
our current modelization we will set :

\begin{equation}
	\mathcal{I}  \begin{pmatrix} u \\ v \end{pmatrix}  = P^p + F \begin{pmatrix} u + b_1 + b_2 v \\ v \end{pmatrix} 
\end{equation}

We remind (see \RefFantome) that we use  only $2$ parameters for linear distorsion 
\footnote {from the $4$ possible linear term in a $2 \times 2$ mapping}
 as $1$ is already include in the focal and another is include
because rotation in the plane is redundant with $3D$ rotation. 

This model is selected, not because it will be always appropriate (it will not !) but because
it is the model  for wich we can have direct solution. Sometime we will need less parameters,
for example we "know" that $b_1=b_2=0$ , or we can  know that $P^p$ in the middle of image. Sometime
we will need more parameters, for example adding a radial distorsion. And also sometime
"more and less" \dots When we require other calibration model, we still can use this method to compute
an initial solution and then make a bundle adjustement to refine the model.

    %-----------------------------------------------------

\subsection{Setting equations}

We write $A \sim B$ the relation indicating that $A$ and $B$ are colinear :

\begin{equation}
	  A \sim B   \Leftrightarrow  \exists \lambda : A =  \lambda B
\end{equation}

This projective relation, is related to image formula via :

\begin{equation}
	 \begin{pmatrix} u \\ v \end{pmatrix}  = \pi_0  \begin{pmatrix} x \\ y \\z  \end{pmatrix}
   \Leftrightarrow   \begin{pmatrix} u \\ v \\ 1 \end{pmatrix}  \sim  \begin{pmatrix} x \\ y \\z  \end{pmatrix}
\end{equation}

We have, noting $\mathcal{C}_{al}$ the calibration matrix :

\begin{equation}
	   \begin{pmatrix} i \\ j \\ 1\end{pmatrix}
      =   \begin{pmatrix} P^p_x + F(u+p_1u+p_2v) \\ P^p_y + F v \\ 1\end{pmatrix}
      =  \begin{pmatrix} F(1+p_1) & F p_2 & P^p_x ) \\  0 &   F & P^p_y \\  0 & 0 &1\end{pmatrix} * \begin{pmatrix} u \\  v \\ 1\end{pmatrix} 
	      =  \mathcal{C}_{al} \begin{pmatrix} u \\  v \\ 1\end{pmatrix} 
              \label{UCResecCalibM}
\end{equation}

We have also :

\begin{equation}
	 \begin{pmatrix} u \\  v \\ 1\end{pmatrix}
		 \sim P_c =  \trans R(P-C)
\end{equation}

So :
\begin{equation}
	   \begin{pmatrix} i \\ j \\ 1\end{pmatrix}
		   \sim \mathcal{C}_{al}  \trans R  (P-C)
\end{equation}

Noting $  \mathcal{M} =  \mathcal{C}_{al}  \trans R $, and $Tr = -\mathcal{M} C$, we have :

\begin{equation}
	\begin{pmatrix} i \\ j \end{pmatrix} = \pi_0 ( \mathcal{M} P + Tr )
\end{equation}

Noting :

\begin{equation}
	\mathcal{M}   =  \begin{pmatrix} a & b & c \\ d & e & f \\ g & h & i \end{pmatrix} 
   \;\;\; ; \;\;\;
	 Tr  =  \begin{pmatrix} t_x \\ t_y \\ t_z \end{pmatrix} 
\end{equation}

We finaly have :

\begin{equation}
	i = \frac{ax+by+cz+t_x}{gx+hy+iz+t_z}
   \;\;\; ; \;\;\;
	j = \frac{dx+ey+fz+t_y}{gx+hy+iz+t_z}
	\label{EqHomSRU}
\end{equation}

    %-----------------------------------------------------

\subsection{Solving equation}

    %-----------------------------------------------------
\subsubsection{General approach}

We want to compute $\mathcal{C}_{al},R,C$  from a set of correspondance $(i,j) (x,y,z)$.
We will proceed in $2$ steps :

\begin{itemize}
     \item estimate $a,b\dots ,t_x \dots$, i.e. $\mathcal{M}$ and $Tr$ using equation \ref{EqHomSRU};
     \item extract  $\mathcal{C}_{al},R,C$ from $\mathcal{M}$ and $Tr$;
\end{itemize}


    %-----------------------------------------------------
\subsubsection{Homography estimation}

Equation~\ref{EqHomSRU} is a classical equation for computing unknown homography.
Considering that $i,j,x,y,z$ are observations and $a,b\dots ,t_x \dots$ are unknowns, we can write it 
as :

\begin{equation}
	0 = {ax+by+cz+t_x} -i (gx+hy+iz+t_z)
   \;\;\; ; \;\;\;
	0 = dx+ey+fz+t_y - j (gx+hy+iz+t_z)
	\label{EqLineSRU}
\end{equation}

The nice point is that equation~\ref{EqLineSRU} is linear, easy to solve and with
many observations, we can use least square approach. The bad point is that,
as there is no constant occuring in it: 

\begin{itemize}
    \item the system is ambiguous , if $S$ is a solution $\lambda S$ is a solution
    \item worst, the null vector is a perfect solution that will always annulate the residual
	    of the least-square system.
\end{itemize}

There is different way to overcome this problem. We indicate the way used in {\tt MMVII}.
The trick, as equations ~\ref{EqLineSRU} are defined up to a scaling factor is to fix the
ambiguity by fixing arbitrary one of the variable to a non null value, for example $a=1$ or $t_z=1$. 
But which of the variable must we fix ?  If we make a bad choice, will it have an influence on the
quality.  The method we use is in fact to not really make a choice! We test all the
possible variable, each test lead to a solution and we use equation~\ref{EqHomSRU} to
select the best solution. Note, that it is fast, as the normal matrix is computed only once.
For more detail on the implementation please refer to the code which is densely
documented.


    %-----------------------------------------------------
\subsubsection{Un-mixing parameters}

Sometime, it will be sufficient to know $\mathcal{M}$ and $Tr$, because
with them we can project any point in the image using equation~\ref{EqHomSRU}.
But in other case, we will need to recover the "physical" parameters.

Considering the center, it is easy to compute it, from $Tr = -\mathcal{M} C$, we have :

\begin{equation}
     C= - \mathcal{M}^{-1} Tr
\end{equation}

Computing $\mathcal{C}_{al}$ and $R$ from $\mathcal{M}$, is a classical problem
from matrix algebra, we want the decomposition :

\begin{itemize}
    \item   $\mathcal{M} = \mathcal{C}_{al} \trans R $
    \item   $\mathcal{C}_{al} $  is a triangulas superior matrix;
    \item   $R $  is a rotation matrix;
\end{itemize}

The decomposition of $\mathcal{M}$ is almost what is known to
be the $QR$ decomposition, where $Q$ is orthogonal and $R$ triangular superior
\footnote{we use this  denomination, as it is universal, be aware that it is
error prone as $R$ is not the rotation !}
The eigen library, and most current matrix library offer efficient implementaion
of $QR$ decomposition as it is one of the central method for solving linear
equation \footnote{solving a linear system becomes trivial if matrix are
triangular or othogonal}
So $QR$ decomposition almost save our problem \dots except many details :

\begin{itemize}
    \item what we need is $RQ$ decomposition and not $QR$; by the way it's easy
          by transposition and some column/line symetry to use $QR$ method to make $RQ$,
          interested reader can see the {\tt MMVII} method {\tt RQ\_Decomposition};

    \item $QR$ decomposition is ambiguous up to sign-matrix \footnote{a diagonal
          matrix with only $\pm 1$ on the diagonal}; let $S$ be any sign matrix, we have $SS=Id$,
          then we have $QR=(QS)(SR)$, $QS$ is still orthogonal and $SR$ is still triangular;
          in the {\tt MMVII} library , the $RQ$ decomposition method make a post processing
          to have positive diagonal (it is easy as letf/right multiplying by a sign matrix is only 
          a matter of changing the sign of line/column);
          
    \item let write $\mathcal{M} = T O $ the result of $RQ$ decomposition \footnote{$T$ triangular,
          $O$ orthogonal},  $\mathcal{M} $ is defined up to a scaling factor, so are $T$ and $O$;  this has two
          consequences

          \begin{itemize}
                \item for $\mathcal{C}_{al} $, we see in equation~\ref{UCResecCalibM} that
                     $\mathcal{C}_{al}(2,2)=1$, so  we just have to do
                     $\mathcal{C}_{al}(2,2) = \frac{T}{T(2,2)}$

                \item for $R$, there can be a global sign ambiguity , so we test the determinant of
                      $O$ and do something like $R=\frac{\trans O}{det(O)}$;
          \end{itemize}

\end{itemize}

Recovering all the internal parameters using {\tt MMVII} convention  is now direct :

\begin{itemize}
    \item  $F=\mathcal{C}_{al}(1,1)$ , $P^p_x=\mathcal{C}_{al}(2,0)$,  $P^p_y=\mathcal{C}_{al}(2,1)$;
    \item  $p_1=\frac{\mathcal{C}_{al}(0,0)}{F}-1$ ;
    \item  $p_2=\frac{\mathcal{C}_{al}(1,0)}{F}$ ;
\end{itemize}





% (I)    (PPx + F (u  p1 u + p2 v))     (F(1+p1)   p2F  PPx) (u)     (a b c) (u)      (u) [EqCal]
% (J) ~  (PPy + F v               )  =  (0         F    PPy) (v)  =  (0 e f) (v) =  C (v)
% (1)    (                       1)     (0         0     1)  (1)     (0 0 1) (1)      (1)


%-----------------------------------------------------
%-----------------------------------------------------
%-----------------------------------------------------

\section{Ortographic case with Tomasi-Kanabe}

%-----------------------------------------------------
%-----------------------------------------------------
%-----------------------------------------------------

\section{$2$-images orientation with essentiel matrix}

%-----------------------------------------------------
%-----------------------------------------------------
%-----------------------------------------------------

\section{$2$-images, case planary scenes}


    %-----------------------------------------------------

\subsection{Notes for TD}


\begin{verbatim}
TD1 : SPACE RESECTION

*  Begin with a calibrated camera w/o distortion,
*  From $3$ point compute 2 lamba with MMVII , then generate the rotations, solve ambiquity (with $4$ points)
*  From data with outlier (50\% on 50 point) use ransac like to make a robust init.

*  Do the same thing with distorsion => inverse disorsion (iterative method ? Lesqt Sq ? Majical optimzer of pytthon ?)

*  eventually => interface


TD2 :  CMP CALIB

TD3 : CALIB CONV
\end{verbatim}






\chapter{Description de la m\'ethode Aim\'e}

\section{Introduction}

Aim\'e vise à \^etre une m\'ethode de calcul de points homologues qui sera  int\'egr\'ee dans MMVII. 
Son architecture est fortement inspir\'ee par SIFT qui a fait ses preuves depuis de longues
ann\'ees en tant que m\'ethode \emph{analytique}\footnote{"analytique" par opposition à m\'ethode d'apprentissage,
les ayatolah du deep dirait handcrafted} de r\'ef\'erence. Elle vise aussi à tirer parti de $10$ ann\'ees
d'exp\'erience d'utilisation de SIFT en photogramm\'etrie pour corriger les principaux point faibles
de SIFT dans ce contexte.  Notemment les principaux points  sur lesquels on souhaite am\'eliorer sont :


\begin{itemize}
   \item SIFT passe difficilement à l'\'echelle lorsque  l'on l'utilise sur des tr\`es grands jeux de donn\'ees,
         notamment il ne permet pas de d\'etecter rapidement les paires potentiellement homologues;

   \item lorsque l'on en dispose, SIFT n'utilise pas d'information  de spatialisation approch\'ee qui permetrait
         de faciliter l'appariement (plus robuste et plus rapide), voir de se passer dans ce contexte d'invariance
         inutile (i.e ne pas normaliser à l'\'echelle ou à la rotation si on a des information permettant de
         connaitres ces infos);

   \item tirer parti des m\'ethodes d'apprentissage pour avoir un appariement meilleurs (quoique veuille dire
         ceci); une contrainte pour tirer parti des m\'ethodes d'apprentissage est de disposer de  jeux de
         donn\'es de v\'erit\'e suffisemment denses; or cela est possible, au moins pour la phase d'appariement 
         en utilisant des jeux de donn\'ees trait\'es par des cha\^ines photogramm\'etriques automatiques.

   \item prendre en compte que les probl\`emes d'appariement rencontr\'es en photogramm\'etrie sont
         suffisement vari\'es pour qu'il soit n\'ecessaire de viser \'a une m\'ethode fortement param\'etrable
         pour pouvoir s'adapter \`a cette vari\'et\'e de probl\`emes.


\end{itemize}

    % ==================================================================================
    % ==================================================================================
    % ==================================================================================

\section{Architecture g\'en\'erale d'Aim\'e}

L'architecture est la suivante :

\begin{itemize}
   \item calcul de points caract\'eristiques, cette partie est celle qui resssemble le plus \'a SIFT;

   \item calcul de descripteurs, le descripteur est fait d'un (ou plusieurs) descripteurs binaires
         permettant de faire une pr\'e-s\'election rapide des couples potentiellement appariables et
         d'un descripteurs plus complets , ces descripteurs sont bas\'es sur un r\'e\'echantillonage
         log-polaire de du voisinage de chaque point;

   \item appariement en plusieurs \'etapes, on commence par des filtres relativement peu s\'electifs
        mais rapides et l'on compl\`ete par des calcul plus s\'electif sur les descripteurs complets;

   \item filtrage spatial imposant une coh\'erence spatiale sur le principe "les homologues de mes voisins
         sont les voisins de mes homologues"; \'eventuellement la phase d'appariement a pu \^etre volontairement
         ambigu\"e (chaque point a un ensemble d'homologues potentiels) , parce que  certaines ambigu\"it\'es ne
         peuvent  pas \^etres r\'esolues au niveau individuel, et la phase de filtrage cherchera \`a r\'esoudre
         ces ambigu\"it\'es par une approche de type relaxtion.

   \item \'eventuellement, am\'elioration de la pr\'ecision de localisation par des m\'ethodes basiques 
         de corr\'elation.

\end{itemize}

Le degr\'e d'impl\'ementation actuel est tr\`es in\'egal et diminue lorsque l'on avance dans le processus.
Sch\'ematiquement ;

\begin{itemize}
   \item pour les points caract\'eristiques, il existe un prototype complet que l'on peut consid\'erer comme 
         pr\'e-op\'erationnel; il est surement perfectible \'a terme, mais ce n'est pas une priorit\'e; 

   \item le descripteur en log-polaire est implant\'es, des descripteurs binaires ont \'et\'e prototyp\'es de
         mani\`ers handcrafted (par une ACP),  mais gagneraient problement \'a \^etre con\c{c}u avec une 
         approche bas\'ee sur de l'apprentissage;

   \item l'appariement a \'et\'e impl\'emnt\'e de mani\`ere basique, essentiellement pour valider les deux
         \'etapes pr\'c\'dentes;  une voie d'am\'elioration majeure (en tout cas esp\'er\'ee comme telle)
         serait de faire de l'apprentissage sur les paires de descripteur complets;

   \item rien n'a \'et\'e fait sur le filtrage spatial, il aussi esp\'er\'e  que l'approche 
         "appariement ambigue/filtrage spatial avec relaxation" soit une source d'am\'elioration importante
         dans certain cas , notamment le cas de sc\`enes avec des structures partiellement r\'ep\'etitives;

   \item enfin le raffinement par corr\'elation, least square matching n'a pas \'et\'e test\'e, il s'agit
         d'une option a priori facile, qui donnerait une plus value op\'erationnelle sans avoir de valorisation
         en recherche (c'est assez classique).

\end{itemize}


    % ==================================================================================
    % ==================================================================================
    % ==================================================================================

\section{Calcul des points caract\'eristiques}


\subsection{Multi-\'echelle}

\subsubsection{Pr\'esentation g\'en\'erale}

Comme indiqu\'e pr\'ec\'edemment, cette partie est celle qui s'inspire le plus directement de SIFT.
Notamment, comme dans SIFT :

\begin{itemize}
    \item on calcule une pyramide d'image d'echelle d\'ecroissante suivant une loi exponentielle, 
          on note $I_0$ l'image initiale, et $I_k$ la $k^{ieme}$ image, entre $I_k$ et $I_{k+1}$
          il y a un rapport d'\`echelle $\sigma$, donc par it\'erations successive entre $I_0$
          et $I_k$ il y a un rapport d'\'echelle $\sigma^k$;

    \item comme dans SIFT, $\sigma$ est choisi tel que $\sigma^n=2$ ou $n$ est un entier, et
          tous les $n$ l'image est d\'ecim\'e d'un facteur $2$ pour gagner un facteur important de temps
          de calcul, sachant que quand l'image est devenu suffisement floue, l'auto-corr\'élation entre un
          pixel et son voisin fait que l'on perd peu d'information avec cette d\'ecimation;

    \item  la pyramide d'\'echelle permet d'avoir plus de point quand cela est n\'ecessaire, et surtout
           permet d'obtenir l'invariance \`a l'\'echelle (pour deux images $I$ et $J$ prises \`a deux
           \'echelles diff\'erentes de rapport $R$, si un point $p$ est d\'etect\'e sur $I_k$, il pourra aussi
           \^etre d\'etect\'e sur $J_{k'}$ avec $R = \sigma^{k-k'}$ )

    \item  la convolution par une gaussienne est isotrope ce qui  est n\'ecessaire pour l'invariance par rotation;


\end{itemize}


\subsubsection{Imp\'ementation dans MMVII de la pyramide}

\emph{Il n'est pas prouv\'e que les choix d'impl\'ementation d\'ecrit ici soit judicieux, cette description
vise à comprendre le code. Des tests de performance \'a une impl\'ementation basique reste \`a effectuer.}

L'impl\'ementation repose sur deux remarques :

\begin{itemize}
    \item en vertu du th\'eor\`eme central limite , si on convolue avec suffisement de fois avec des fonctions r\'eguli\`eres,
          on converge vers la convolution avec une gaussienne;
    \item convoluer avec deux gaussiennes sucessive d'\'ecart type $a$ et $b$, a exactement le même effet que
          de convoluer avec une seule gaussienne d'\'ecart type $\sqrt{a^2+b^2}$;
  
\end{itemize}

Le calcul se fait en iterant une convolution par $e^{-a|x|}$ qui est une fonction 
"smooth" et dont le produit de convolution peut \^etre calcul\'e rapidement (algorithme r\'ecursif classique).
La fonction est impl\'ement\'ee dans {\tt ExpFilterOfStdDev}.  Cette fonction prend
en param\`etre un $\sigma$ et un nombre d'it\'eration $N$, elle calcule la valeur
$a$ qui donnera un \'ecart type de $\frac{\sigma}{\sqrt{N}}$,

A chaque octave  :

\begin{itemize}
   \item  pour $I_0$ le calcule se fait avec un nombre relativement important d'it\'eration pour
          bien approximer la gaussienne

   \item  ensuite le calcul de $I_{k+1}$ se fait \`a partir $I_k$ (en tenant compte du flou d\'ej\'a pr\'esent
          en $I_k$, on convolue par $\sigma^k \sqrt{\sigma^2-1}$), comme $I_k$ est d\'ej\'a le r\'esultat de plusieurs
          convolution, on peut faire moins d'it\'ération (MMVII en fait $4$ pour $I_0$, $3$ pour $I_1$, et $2$ ensuite);
\end{itemize}

Il y a un cas particulier pour la premi\`ere it\'eration, si on veut que la pyramide soit r\'eguli\`ere en \'echelle.
Typiquement si l'image initiale est tr\`e floue, il faudrait la d\'econvoluer, inversement si elle est tr\`es piqu\'ee,
il faudra la flouter.  Je ne vois pas d'autre solution que de faire une hypothèse sur la largeur de la t\^ache image.





\begin{equation}
     I_k = I_0  \circledast G(\sigma^k)
\end{equation}


\section{VRAC}

\begin{itemize}
   \item indexe binaire
    \item crit\`ere rapide de d\'etection de paires
\end{itemize}




\chapter{The "Aime" methods for Tie Points computation}


% Conclusion, on peut sans doute limiter le nombre de point avec ScaleStab
% pour filtrage a priori => genre les 500 les plus stable

%---------------------------------------------
%---------------------------------------------
%---------------------------------------------

\section{Fast recognition}

%---------------------------------------------

\subsection{Motivation}
For each image, we have computed tie points. A tie points is made
of a vector $V \in \RR^n)$ . Typically $V$ is invariant
to the main geometric deformation .  Le  $V_1$ and $V_2$
be two tie points, we note :

\begin{itemize}
   \item  $H_{om}(V_1,V_2) $ the fact that two tie points are homologous;
\end{itemize}

Given $V_1,V_2$, there is of course no no oracle that can indicate if  $H_{om}(V_1,V_2)$,
and classically we want to compute  a fast mathematical function $\Psi $ that indicate if two vector $V_1$ and $V_2$
correspond to the same tie points .  The ideal function would be such :

\begin{itemize}
   \item  $\Psi(V_1,V_2)  \iff H_{om}(V_1,V_2)$
\end{itemize}


Of course this impossible, and we introduce the miss rate  and fall out:

\begin{itemize}
   \item   miss rate , probability of $\Psi=0$ knowing $H_{om}=1$ , we note $p_m$;
   \item   fall out , probability of $\Psi=1$ knowing $H_{om}=0$, we note $p_f$;
\end{itemize}

As we cannot have the ideal function $\Psi $ such as $p_m=0$ and $p_f=0$,
we have to compromise, and depending on the circunstances, the price of the
two error, will not be the same. Typically, in indexation step,  we are especially
interested to have a low $p_f$; converselly in recognition step we are
especially interested to have a low  $p_m$.


\subsection{Bits vector}


\section{Gaussian pyramid}

\subsection{Computing $\sigma_0$}

\label{GP:SIGMA0}

The gaussian pyramid is made from a succession of image at different scale that
result from a gaussian filter. How can we justify it :

\begin {itemize} 
   \item  image $I_k$ must be at resolution  $R_k =  R_0 s ^k$,
   \item if we assimmilat  $I_0$ to a (sum of) gaussian of std dev $\sigma_0$  , $I_k$
         must be (sum of)  gaussian  $\sigma_0 * R_k$ 
   \item so we can write $I_k = I_0 \ast G(\sigma_k)$ with $\sigma_k^2 + \sigma_0^2 = (\sigma_0  R_k)^2$;
\end {itemize} 

To compute the pyramid, we need an estimation of $\sigma_0$. Which is quite natural, if the initial
image is very blured, it a high $R_0$, and the value $R_1-R_0 = (s-1)R_0$ is also high, which require
a high value for gaussian filter.

We need a way to estimate the initial value $\sigma_0$. Also it's quite arbirtrary, the way it is
done in MMVII is :


\begin {itemize} 
   \item  assimilate $I_0$ to a gaussian of  std dev $\sigma_0$;
   \item  suppose $I_0$ is well sampled (nor blurred nor aliased);
   \item we traduce it mathematically by  :
\end {itemize} 

\begin{equation}
    \int_{-\infty}^{+\infty} I_0 |x| = \frac{1}{2}
\end{equation}

Due to symetry , we can replace by integral on $[0,+\infty]$ and supresse absolute value :

\begin{equation}
    \int_{0}^{+\infty} \frac{x}{\sigma_0 \sqrt{2\pi}} e^{-\frac{x^2}{2\sigma_0^2}}  = \frac{1}{4}
\end{equation}

We integrate :

\begin{equation}
      \lbrack  \frac{-\sigma_0}{\sqrt{2\pi}} e^{-\frac{x^2}{2\sigma_0^2}}\rbrack  _{0}^{+\infty}  = \frac{1}{4}
\end{equation}

So :

\begin{equation}
      \sigma_0 = \sqrt{\frac{\pi}{8}} \simeq 0.626
\end{equation}


%----------------------------------------
%  A conserver : equations TIPE Lolo
%-----------------------------------------
\COM{
\begin{equation}
   \delta R = \frac{1}{\sigma} \frac{L}{S} = \frac{1}{\sigma}  \frac{2 \pi a }{ e \; dz} 
\end{equation}

\begin{equation}
   \phi  = \iint \overrightarrow{B} \overrightarrow{dS} 
         = B_M \cos(\omega t) \pi a^2
\end{equation}

\begin{equation}
   e = -\frac{d\phi}{dt}= B_M \omega  \sin(\omega t)  \pi a^2
\end{equation}


\begin{equation}
   d P_J = \frac{e^2}{\delta R}  
         = \frac{ (B_M \omega \pi a^2  \sin(\omega t))^2 \sigma_e dz }{2 \pi a}
\end{equation}


\begin{equation}
   <d P_J> = \frac{B_M^2 \; \omega ^2 \; \pi \; a^3 \; e \; \sigma \; dz}{4}
\end{equation}

\begin{equation}
  <P_J> =  \int_{z=0}^H <d P_J> = \frac{B_M^2 \; \omega ^2  \; \pi \;  a^3  \; e  \; \sigma  \; H}{4}
\end{equation}

\begin{equation}
  dU = \delta ^2 Q_{creee} +  \delta ^2 Q_{recue de l'air}
\end{equation}

\begin{equation}
  C dT = <P_J> dt - g 2\pi aH(T-T_a)
\end{equation}

\begin{equation}
  C = \mu c 2 \pi a e H
\end{equation}


\begin{equation}
  \mu c 2 \pi a e H \frac{dT}{dt} = \frac{B_M \omega^2 \pi a^3 e \sigma H}{4} - g 2 \pi a H (T-T_a)
\end{equation}


\begin{equation}
  \tau  \frac{dT}{dt} + T = T_{\infty}
\end{equation}


\begin{equation}
  \tau  = \frac{\mu  c e}{g}
\end{equation}

\begin{equation}
  T_{\infty} - T_a = \frac{(B_M \omega a)^2 e \sigma}{8g}
\end{equation}
}


%---------------------------------------------
%---------------------------------------------
%---------------------------------------------

\section{Matching by relaxation}

\subsection{Notations}

We have two images $I$ and $J$, and two set of characteristic points
in $I$ and $J$ :

\begin{itemize}
   \item $A_i \in I$ for $i \in [1 \cdots M]$;
   \item $B_j \in J$ for $j \in [1 \cdots N]$;
\end{itemize}

We will note $D_{Im}$ the size of the images, typically the length of the diagonal. It will be useful
when we need to set some geometric thresholds.

For each characteristic point, we have both a vector descriptor and a position in the image.
We note :

\begin{itemize}
   \item $A^v_i , B^v_i$  the vector descriptor,  
         $A^v_i, B^v_j \in \mathbb{R}^K $  were the dimension $K$
         is typically some hundreds; however here we are only interested by the fact  there exist 
         some distance $D^v$ between
         vector such that the lower the distance, the more likely is that points are homologous;
         we note $D^v(A^v_i , B^v_i) = D^v(A_i , B_i) $
   \item  $A^p_i , B^p_i$  the position ("pixels")   in images of $A$ and $B$ , 
         $A^p_i , B^p_i \in \mathbb{R}^2$
\end{itemize}


We suppose that we can convert the distance $D^v$ in the likelihood $L$ expressing  that $A$ and $B$
are homologous. For example we  use a threshold $\sigma_v$ and set :

\begin{equation}
    L(A_i,B_j) =  e^{-\frac{D^v(A_i,B_i)}{\sigma_v}} \label{EQ:LikL}
\end{equation}

The transformation from  $D^v$ to $L$  , or even the computation of $\sigma_v$
admitting of formula like \ref{EQ:LikL}, may be an issue. By the way we will admit from
now that we have the function $ L$ , probably a way to rationalize this fact
would be to "learn" it from statistic of "real" homologous points.


\subsection{Formulation}

We want to compute a matching between $A_i$ and $B_j$ that use both the "radiometric context" informations 
and the "spatial consistency" information :

\begin{itemize}
   \item radiometric context : $A_i , B_j$  are matched if, as much as possible, $L(A_i,B_j)$ is high;
   \item spatial consistency : if $A_i$ and $B_j$ are matched, then for each $A'_i$ and $B'_j$ such that ,
         $A'^p_i$ is close to $A^p_i$ and $B'^p_i$ is close to $B^p_i$ , then the likelihood
         to match $A'_i$ and $B'_j$ is higher (i.e. neighbors of my homologous are homologous of
         my neighbors);
\end{itemize}

The principle of relaxation is to compute a non-unique weighted matching and to use
the two pieces of information to iteratively re-compute new value of the weighting. Generally, at
the beginning, the "brute" information (=radiometric context) has a higher importance, and as we go
further we give more importance to relational information (=spatial consistency).

To formalize the problem we consider that we want to compute simultaneously two functions :


\begin{itemize}
   \item a discrete matching function $\psi$  such that $\psi (A_i)$ is the homologous
         of $A_i$, with possibly $\psi (A_i) = 0$  when no match is found for $A_i$,
         and we set  $L(A_i,0) = L_0$ where $L_0$ is a constant;
   \item a geometric \emph{smooth} function $\phi$ .
\end{itemize}

And the criterion on $\psi$ and $\phi$ are to  maximize global
likelihood $\mathcal L(\psi)$ (\ref{Crit:Psi})  while minimizing the
spatial incoherence $ \mathcal S(\psi,\phi)$  (\ref{Crit:PhiPsi}).


\begin{equation}
    \mathcal L(\psi) = \sum  L(A_i,\psi(A_i)) \label{Crit:Psi}
\end{equation}


\begin{equation}
    \mathcal S(\psi,\phi) = \sum  | \psi(A_i)^p - \phi(A^p_i) | \label{Crit:PhiPsi}
\end{equation}

Of course, there must be some strong regularity constraint on $\phi$ so that the
equation ~\ref{Crit:PhiPsi} is not trivial. Generally it belongs to a parametric
space of low dimension (such as similitude, homography ...).


\subsection{Simultaneous $\psi$ and $\phi$ ?}

In most general case, it is quite complicated to compute simultaneously $\psi$
and $\phi$.

\subsubsection{Classical approach}
In the most current case in photogrammetry, a first estimation $\psi_0$ is computed with a simple
strategy as :

\begin{equation}
    \psi_0(A_i) =  \underset{B_j \in J}{\operatorname{argmax}}  L(A_i,B_j) \label{ArgMax:Psi}
\end{equation}

And then $\phi$
is computed to remove false match detected as "outliers" of tested $\phi$. 
There is many tricks to have better result (like symmetric matching), but basically
this is the idea.

A particular and very common case of this simultaneous computation is the case
where $\phi$ represents epipolar geometry and is computed by Ransac. By the way in this
case, the  use of Ransac supposes that we know $\psi$ and that it contains a reasonable
proportion of false matches (typically no more than $50\%$). This case is quite common
for example with images acquired the same time (no diachronism) and tie-points resulting
from SIFT.

\subsubsection{Approach by computing $\phi$ first}

In the problem we want to tackle here, the proportion of false match we would get with
equations like~\ref{ArgMax:Psi} is much too high . So we suppose that we can first
have an \emph{approximate} value $\phi_0$ of $\phi$. We will discuss just after how we can
estimate $\phi_0$.

We have to specify that what we mean by $\phi_0$ is an approximation of $\phi$.
To understand equation~\ref{Phi0:ApproxG}  and~\label{Phi0:Appro} we must
imagine a possible scenario for $\phi$ and $\phi_0$. Typicaly $\phi$, the real
correspondence, will the combination of epipolar geometry and $3D$ scene, so according
to the scene, the $\phi$ can be  continuous or only piece-wise continuous. For $\phi_0$,
in our context it will be a function with few parameters, typically a similitude
plane computed from few points (at least $2$). 

We then formalize "$\phi_0$ approximating  $\phi$" by (see also Fig.~\ref{fig:spatial-consist}) :


\begin{equation}
     | \phi_0(A)  - \phi(A) |  < D_A  \label{Phi0:Appro}
\end{equation}

\begin{equation}
     | (\phi_0(A) - \phi_0(A')) -(\phi(A) - \phi(A')) |  < D_g +  \alpha |A-A'|  \label{Phi0:ApproxG}
\end{equation}

By extension, using equation \ref{Crit:PhiPsi} and criterion $\mathcal S$ we want to minimize,
we will set :

\begin{equation}
     | \phi_0(A)  - \psi(A) |  < D_A  \label{PsiPhi0:Appro}
\end{equation}

\begin{equation}
     | (\phi_0(A) - \phi_0(A')) -(\psi(A) - \psi(A')) |  < D_g +  \alpha |A-A'|  \label{PsiPhi0:ApproxG}
\end{equation}

\begin{figure}
\centering
\includegraphics[width=12cm]{Methods/Images/spatial_consist.JPG}\caption{Illustration of equation~\ref{Phi0:ApproxG}.}\label{fig:spatial-consist}
\end{figure}

Let comment these equations  :

\begin{itemize}
   \item in equations~\ref{Phi0:Appro},~\ref{PsiPhi0:Appro},~\ref{PsiPhi0Im:Appro} the value 
         $D_A$ can be relatively high as $\phi0$
         is extracted with few points and the model (similitude) is a rough approximation
         of the "real" model; it seems natural to have $D_A$ proportional to  $D_{Im}$
        (see~\ref{PsiPhi0Im:Appro});

   \item equation~\ref{Phi0:ApproxG},~\ref{PsiPhi0:ApproxG} model the fact that once 
         we "know" that  $ \psi(A)=B$ then $\phi_0$ is a relatively accurate approximation
         of $\psi$ around $A$;  when  $\phi$  is continuous, we don't need $D_g$, however
         when the $3D$ scene is discontinuous, it create discontinuities ; 
         it seems also natural to have $D_g$ proportional to  $D_{Im}$, and obviously
         $D_g$ significantly smaller than $D_A$ (see~\ref{Phi0Im:ApproxG} )
\end{itemize}

So we can write :


\begin{equation}
     | \phi_0(A)  - \psi(A) |  < \beta_A D_{Im}  \label{PsiPhi0Im:Appro}
\end{equation}

\begin{equation}
     | (\phi_0(A) - \phi_0(A')) -(\psi(A) - \psi(A')) |  < \beta_g D_{Im} +  \alpha |A-A'|  \label{Phi0Im:ApproxG}
\end{equation}

Of course, practically, an important question is the values of thresholds $\beta_A, \beta_g, \alpha $.
As a rule of thumb, values of these parameter can be fixed in the following interval :

\begin{itemize}
   \item  $\beta_A \in [0.05,0.2]$, less than $0.05$ would be very optimistic , and by the way this low
          value is probably already sufficient for basic approach like in~\ref{Basic:ArgMax};
          more than $0.2$ would be very pessimistic and by the way difficult to use;

   \item  similarly $\beta_g \in [0.01,0.05]$ and $\alpha \in [0.05,0.3]$
\end{itemize}


And in a first try I woud use $\beta_A=0.1 \;  \beta_g=0.025 \; \alpha=0.15$.


\subsection{Estimation of $\phi_0$}

Obviously, if we cannot estimate $\phi_0$, the previous stuff is useless.
Basically, we can imagine $3$ way to estimate $\phi_0$ :

\begin{itemize}
   \item  the most trivial way is to have an "external" estimation, this can come from
          meta data ("tableau d'assemblage") or from operator measurements (operator select
          manualy two real homologous point and it is sufficient to estimate a similitude);

   \item  another easy way is to use tradional Ransac on a solution like ~\ref{ArgMax:Psi}; 
          with D2Net executed at full resolution, it will probably not work as the proportion
          of false match can be very high; however it is possible to run D2Net at a very low
          resolution, where~\ref{ArgMax:Psi} works not so bad, the accuracy is not so good, but
          it is probably not a problem;

   \item  a third, more sophisticated way, consist to use a non-unique matching at high
          or medium resolution; it is described bellow .
\end{itemize}

Non-unique matching :
\begin{itemize}
   \item  for each point in one image select a number $p$ of its potential matches in the other images, in other words for each point in  $A_i$, select 
            $\Psi (A_i) $ which contains the  $B_j$ corresponding to the $p$
          best scores of $L(A_i,B_j)$; to compute $\Psi$ it is also possible to compute more globally the
          $p * M$ best pairs of all matches $(A_i,B_j)$ where $M$ is a predefined value; here, however, there might be a case where certain points are not assigned a match; to assure that all points have matches a mix of
          both approaches can be adopted (computing a global set of pairs, and also assure to have a minimal
          number of matches per point);

   \item generate solutions, by random selection, if we decide to use similitude,
         it is sufficient to use $2$ pairs of $(A_i,B_j)$  ;  it is also possible to make a biased
         random selection that favors the set of pair having higher likelihood
         (a way to do that is, for each iteration, to generate a few random pairs, 
          and finally select the the pair corresponding to the best likelihood);

   \item for each $2$ pairs, we compute a similitude $S$  and estimate it cost $C(S)$
         by formula below;

   \item after many iteration, select finaly the $S$ minizing $C(S)$.
         by formula above;

\end{itemize}

\begin{equation}
     C(S) = \sum_i (\underset{B_j \in \Psi(A_i)}{\operatorname{Min}}  c(A_i,B_j,S))
\end{equation}

For $c(A_i,B_j,S)$, different formula can be tested, a basic formula proportional to distances, 
one using a threshold to limit the influence of outliers  : 

\begin{equation}
      c(A_i,B_j,S) = Min(|S(A_i)-B_j|,D_A)
\end{equation}

Or a smoother version :

\begin{equation}
      c(A_i,B_j,S) = \frac{|S(A_i)-B_j|}{|S(A_i)-B_j|+D_A)}
\end{equation}

It is also possible to use the likelihood $L(A_i,B_j)$ and merge it with the geometric term , 
however it is always complicated to mix valures of different kind.


\subsection{Basic usage of $\phi_0$}

The easiest way to use $\phi_0$ is to re-use the basic strategy of 
equation~\ref{ArgMax:Psi} but filtering the result with  equation~\ref{Phi0:Appro}  :


\begin{equation}
   \psi_1(A_i) =  \underset{B_j \in \mathcal D(\phi_0(A_i),D_A)}{\operatorname{argmax}}  L(A_i,B_j) 
\end{equation}

Where  $\mathcal D(\phi_0(A_i),D_A)$ is  the disc of center $\phi_0(A_i)$ and radius $D_A$.

\label{Basic:ArgMax}

\subsection{Relaxation}








\chapter{Several stuff unfinished}


%---------------------------------------------
%---------------------------------------------
%---------------------------------------------

%\section{Co-variance propagation}

%---------------------------------------------
%---------------------------------------------
%---------------------------------------------

\section{Bloc fusion}

Hypothesis :

\begin{itemize}
    \item Two bloc $B_1$ and $B_2$  (the method should work with $N$ blocs),
          each has been oriented and we have an unknown similitude to compute between $B_1$ and $B_2$

    \item $N\geq 2$ triangle between $B_1$ and $B_2$
\end{itemize}

We suppose that the rotation has already been solved, which is "easy" because for each triangle ,
the $2$ edges between the $2$ different bloc give an estimation.  For each triangle we have an estimation
and quality check ($3$ measure of rotation when $1$ is enough). 

We need to estimate the scale $\Lambda$ and translation,
let $P^1$  be a point measured in $B_1$ and $P^2$ the same point in $B_2$, 
we have :

\begin{equation}
	P_2 = \Lambda P_1 + T \label{Eq:Bloc12}
\end{equation}

For each triangle we compute the rotation going to $B_1,B_2$ , and we still
have an unknown scale translation for aligning it on $B_2$,
let $\lambda_i$ and $t_i$  the unknown scale and offset of the triangle .
Names $a,b,c$ the tree point of the triangle,  and $P^a_2, P^b_2 $ their homologues
in  $B_2$ , we have for example:

\begin{equation}
	P^a_2    = a \lambda_i + t_i     \label{Eq:Bloc:Tri2}
\end{equation}

Now we have $2$  possibility for each point , is it belongs to bloc $2$ we know $P^x_2$ the equation
\ref{Eq:Bloc:Tri2} holds and we can use it directly. Else we combine~\ref{Eq:Bloc12} and~\ref{Eq:Bloc:Tri2}
to obtain  :

\begin{equation}
	 \Lambda P^a_1 + T = a \lambda_i + t_i \label{Eq:Bloc:Tri1}
\end{equation}

So finally, for  $n$ triangles we solve a \emph{linear} system with $4*(n+1)$ unkwonw  :
$(\Lambda,T,\lambda_i,t_i)$ and we have $9*n$ equation  (by the way, the system
is unslovable for $n=1$ even if we have $9$ equation for $8$ unknowns, explain why ).

Proposed method :

\begin{itemize}
    \item if there is $2$ triangle use ~\ref{Eq:Bloc:Tri2} and ~\ref{Eq:Bloc:Tri1}
          to estimate $\Lambda,T$;

    \item if there is $N>2$ triangle use some ransac strategy by selecting pair of triangles
          and use previous case to solve it;

     \item if there is $1$ triangle, give up for now;  maybe later, use homologous point between
	     blocs , and outside the triangle to fixe the scale ????

\end{itemize}










\part{Programmer's doc}

% -------------------
% -------------------
% -------------------

\chapter{Programming organisation, style ...}

\section{Naming convention}

\section{Never use {\tt std::cout, printf \dots}}

\section{Encapsulation of boost, stl ..}
\section{Error handling}
\section{Memory check}
\section{Serialization}
\section{Shared pointer}

\section{Random number}

\PPP use some pseudo random generator. As every such generator
they must be initialized with a seed.
By default , the seed is always the same to facilitate debuging.
When user wants initialization from time this must be specified 
with a negative value of parameter {\tt  SeedRand}.



\section{Enum to string}

The enum/string  conversion is a recurent problem of \CPP which
as far as I know is still an issue. A possible solution
would be to use some code genration which from easy to read
an write text file woul generate it. But I tried to limitate
the code generation in \PPP.

In file {\tt Serial/uti\_e2string.cpp} is implemanted 
the used solution , it consist  to create data for each enum
for which we want to do the conversion {\tt Serial/uti\_e2string.cpp}.







\chapter{Use of non linear optimization in MMVII}

\label{Chap:NLO}

This chapter is targeted for programmer that will increase MMVII, it mainly describe classes
interfaces in headers and a detailled example of using this classes in a cpp file.
 It's not useful for advanced users; on the other hand, for programmer who will maintain the core of MMVII, 
more detailled documentation will have to be written.

At this step, the chapter has no theoretical presentation, it assumes that the reader is familliar
with least square, linearization, Gauss-Newton, Schur complement\dots


%---------------------------------------------
%---------------------------------------------
%---------------------------------------------

\section{Introduction}

%---------------------------------------------

\subsection{Quick formalization}

This chapter presents the classes used in MMVII for non linear optimisation using some Gauss-Newton methods. 
By non linear optimisation, we mean minimize $F(X)$ with :

\begin{equation}
      F(X) = \sum_{j=1,M} w_j f_j(X)^2  \;  X \in \RR^n  \label{EqNLOInit}
\end{equation}

The assumpution on the problem are classical :

\begin{itemize}
   \item we have an initial guess $X_0$ on $X_{min}$  , this guess is sufficiently 
         good for the local minimal we will (hopefully) find to be a global minimum;

   \item the function $f_j$ are differentiable (some weaker hypothesi are also possible)

\end{itemize}

The service offered by MMVII are typically those a Gauss-Newton iterations : 
use current estimation $\tilde X$ of $X_{min}$ for linearizing the $f_j$ and 
solve equation $\ref{EqNLOInit}$ by least square methods.

%---------------------------------------------

\subsection{2d-Triangularton problem in general}

We will illustrate the presentation by a detailled example that is used for testing correctness of the
implementation. Also relatively basic, this example should be sufficient for a first 
understanding and utilisation of the MMVII classes. The $2d$-triangulation problem we treat
here is the following :

\begin{itemize}
    \item we have a set of points $P_1 \dots P_n$ , each $P_k$ belongs to $\RR^2$;

    \item we do not know the exact values of $P_k$ but we have an initial estimation  $P^0_k$
          which is not too bad, whatever it means;

    \item for a certain number of pair $P_i,P_j$ we have a measurement of the distance $d_{ij}$ between
          $P_i$ and $P_j$,
\end{itemize}

The triangulation problem  consist to use the measurement of distances for recovering the unknown
value on $P_i$.  Let $M$ be the  number of pair for which we have measurement 
Let $X \in \RR^{2n} = \{P_1 \dots P_n\}$ be the unkown vector,
we typically want to find $X$ that minmize $D$ defined by :

\begin{equation}
      D(X) = \sum_{j=1,M} (d_{ij} -d(P_i-P_j))^2  \label{EqConsDist}
\end{equation}


Of course as the fonction $D$ is invariant to any isometry of the plane, the minimization
of $D$ would be an ill-posed problem, as for any set of points, its image by a global rotation
will give the same value. To ensure uniqueness of the solution  will force the value of
a certain number of coordinates.

%---------------------------------------------

\subsection{2d-Triangularion in this example}

As we want the example to be as simple as possible, and also we want to use it for checking
the correctness, the data will be different from a real one , the figure~\ref{fig:NetFull} illustrates 
this network:

\begin{itemize}
    \item the "real" value of points will be positionned on a regular grid, typically we
           will have $(2*N+1)^2$ points, each point having integers corrdinate 
           $(x,y) \in [-N,N]^2$

    \item we will have measure between pair of point that  are $8$-neighoor i.e
          $max(|x-x'|,|y-y'|)\leq 1 $ , this is sufficient for the solution to be unique
          up a rotation ;

    \item in real life  $d_{ij}$ would be noisy, but here in this simulation example we
          use the exact value because, to check correctness of the library, we want to check 
          that we are able to recover the exact values of coordinates.
\end{itemize}

Regarding the arbirtrary constraint we will keep it as simple as possible, so 
considering the two points of the grid  $A=(0,0)$ and $B=(0,1)$ , we will add the following
constraint to the minimization :

\begin{equation}
      x_A=0  \;  y_A=0  \;  x_B=0 \label{Eq:FixVarAB}
\end{equation}

The two first constraint freeze the solution in translation, while the last one freeze the rotation.


\begin{figure}
\centering
\includegraphics[width=12cm]{Methods/Images/T90-NetFull.JPG}\caption{Network, case w/o schur}\label{fig:NetFull}
\end{figure}



%---------------------------------------------

\subsection{Use of schur complement}

Schur complement is an efficient way to treat the case where there is a subset of variable that 
occurs in \ref{EqNLOInit}, but for which we are not specially interested in exact values ,
 however but we want to take them into account rigourously in 
the computation of the others. Once we know
\emph{all} the equations where such  subset of variable is involved, schur
complement offer a way to eliminate them without altering the value of the minimum.

A typical case is in bundle adjsutment, when we are interested only to the value
of camera parameters and not by the value of $3d$ points involved by the projection
equation. In this case, suppose we will typically  have $1000$ camera and $1000000$ points,
doing schur elimination can reduce the system from $3000000$ unkwnon to $6000$.



In our toy example, the schur complement will be purely artificial. When the
{\tt WithSchur}  option is activated, the network is modified in the following
way, the figure~\ref{fig:NetSchur} illustrates 

\begin{itemize}
    \item all the points on the line $x=1$ are considered as temporay variable,
          i.e variable that we want to eliminate, we will not  know their value at the
          end of the computation; 

    \item there will be no conexion between points of line $x=1$ (for example $(1,2)$
          and $(1,3)$ that were connected in previous case are non longer
\end{itemize}

\begin{figure}
\centering
\includegraphics[width=12cm]{Methods/Images/T90-NetSchur.JPG}\caption{Network, case with schur}\label{fig:NetSchur}
\end{figure}

%---------------------------------------------

\subsection{Files of MMVII involved}

We make a quick enumeration of files of MMVII that are to access for understanding :

\begin{itemize}
    \item {\tt include/MMVII\_SysSurR.h} contains the declaration of the class relatives to
          non linear optimisation, from the user point of view the main class
          is {\tt  cResolSysNonLinear};

    \item files on folder {\tt src/TutoBenchTrianguRSNL/} contain the code for the $2d$-triangulation
          example;

    \item {\tt src/SymbDerGen/Formulas\_Geom2D.h } and {\tt src/SymbDerGen/GenerateCodes.cpp } 
          contains the code for generating the automatic differentiation 

    \item {\tt src/GeneratedCodes/CodeGen\_cDist2DConsVal.cpp } contains the code for 
          generating the automatic differentiation , also it is theoretically not necessary 
          to see it for using it, it's no harm to be curious and not use it as a black box;
          by the way in real case use, it will not be necessary to look at the code that you
          will generate.
\end{itemize}

%---------------------------------------------
%---------------------------------------------
%---------------------------------------------

\section{Computing functions and their derivatives}
\label{Compute:Deriv:SysNL}

%---------------------------------------------
\subsection{Motivation}

For Gauss Newton iteration, in equation \ref{EqNLOInit}, we will need to compute not only
the $f_j$ but also all its partial derivative $\frac{\partial f_j}{\partial x_i}$
relatively to all variable $x_i$ used in $f_j$.  There are many way to do it,
and to be honest  all have  pro and cons:

\begin{itemize}
    \item numerical derivative are the most simple, but they are slow, and can be unacurrate
          when the "small" value are not correctly choosen;

    \item hand crafted derivative are accurate and can be fast, however they can be complicated
          to write for real case formula (like arrise in bundled adjustement) and can be
          a nightmare to maintain when a "small" change occurs in the formulas;

    \item jet derivative, like used it Ceres, are almost as simple to use as numericall, once
          you have the library,  and much faster and accurate as numerical ones; they are much
          easier to use and maintain as  hand crafted ones, but can be also slower than
          handcrafted;

    \item symbolic derivative and code generation, can be a bit more complicated to use
          than jet, especially the first time; however after  this quick learning step
          they are as easy as jets to use,  easy to maintain and, in some case, much
          faster.
\end{itemize}

The solution implemenented in MMVII in based on symbolic derivative with code generation.
This choice was made because it is believed that programmer that will goes into the
core of MMVII have rather high requirement in code efficiency and are ready to pay a
small learning steps.

%---------------------------------------------
%\section{Generating the code}

          %  - - - - - - - - - - - - - - - - - - - 

\subsection{class for specifying the formulas}

In this  example, we have only one formula we need to compute and derivate, it's
the formula of equation~\ref{EqConsDist}.  First introduce a bit of MMVII's jargon  :

\begin{itemize}
   \item in this formula we have two different kind of variables the $P_i$ and $P_j$
         and the $d_{ij}$;
         
   \item for the $P_i$ themselves we have no direct observation , we only have initial value,
         and our target is precisely to compute their values; they will be named {\tt unkowns};

   \item for the $d_{ij}$ we know their values (it could be with some uncertaincy, but that'
         another story) and dont try to compute it, they will be name {\tt observations}.
\end{itemize}

The class {\tt cDist2DConservation} in file  {\tt src/SymbDerGen/Formulas\_Geom2D.h } contains
all the information that are required for generating code. It must define $5$ members :

\begin{itemize}
   \item a contructor, that does nothing for such a basic example

   \item a method {\tt VNamesUnknowns} that return the vector of names of unkown ,
         here we have $2$ points and $2$ coordinates $x,y$ for each points so the
         vector has a length of $4$;  the names will be used for code generation;

   \item a method {\tt VNamesObs} that return the vector of names of observations ,
         here we have a single observation which is the targeted distance between 
         $P_1$ and $P_2$;  the names will be used for code generation;

   \item a method {\tt FormulaName}  that return the name of the formula itself,
         the method will be used for the name of class and files containing the generated
          code;

   \item a method {\tt formula}  this is the core of the class, this method
         take as input a vector of unknown and a vector of observation and return
         a vector that is the result of the  computation  (typically a residual
         when used in context of gauss newton);  the type of {\tt tUk} of the
         i/o vectors and of temporary variable, is not very important at this step,
         it's suffice to know that it is a type on which most current mathematicall
         operation are defined (and can be completed when missing), they represent
         mathematical formula as symbolic tree; see \RefFantome for detailled
         explanation on the type {\tt cFormula<Type>} defined in 

\end{itemize}

Regarding the $3$ methods that  return names, as they are used as \CPP identifier
its important that they contains only valid character (alpha numeric and {\tt\_} ,
no {\tt + - ...}),  also it's important that  inside a vector they are unique.
Apart of that, their exact name is unimportant, but giving names semantically
meaningful is useful if a human want to read the generated code (something we
generally dont do, but something we will do here).

\begin{equation}
      D(X) = \sum_{j=1,M} (\frac{d(P_i-P_j)}{d_{ij}} - 1)^2  \label{EqConsDistHom}
\end{equation}

Regarding the method  {\tt formula}, it's here a direct implementation 
of residual used in  equation~\ref{EqConsDistHom}, which
is non dimensionnal variant of equation~\ref{EqConsDist}. Some comment :

\begin{itemize}
   \item we make an intensive use of the {\tt auto} type specification which
         is convenient here;

   \item the less intuitive part is probably {\tt  CreateCste(1.0,x1);},  in fact
         when we need to create a constant  we must indicate the type of symbolic
         formula to the constant is create with adequate type, the {\tt x1} parameter
         is just used to indicate the type in the template function {\tt CreateCste},
         putting {\tt x2}, {\tt y1} or any other would have exactly the same effect;

   \item the function return a vector of formula, which here is of size $1$, and not
         single value, because in general, when we want to compute several values (like
         $x$ and $y$ residual in BA)  its more efficient and convenient to group 
         a multiple residual in a single formula that generating different formulas;

\end{itemize}

As a slightly more complex example, the reader can investigate example {\tt cRatioDist2DConservation}
in the same file. This example correspond to the case what we want to preserve is not the distances
but the angles or ratio of distance. For each triangle $P1,P2,P3$, we have the initial 
distance $D_{i,j}$ as observation and  we write :

\begin{equation}
      \frac{d(P_i,P_j)}{D_{ij}} - \frac{d(P_i,P_k)}{D_{ik}} = 0 \label{EqRatioDist}
\end{equation}

In this cas we have $3$ equation and the function has $3$  values corresponding to the
$3$ possible combination of  equation~\ref{EqRatioDist}.

          %  - - - - - - - - - - - - - - - - - - - 

\subsection{Generating the code}

Once the class {\tt cDist2DConservation} has been created, the things to do for generating the
code is quite simple :

\begin{itemize}
    \item   add in file  {\tt GenerateCodes.cpp } a call to   the method {\tt GenCodesFormula}
            with an object of type {\tt cDist2DConservation}, note that we call it twice,
            the second boolean parameter indicating if we want to  compute the derivate;
            in fact sometime we will be interested only by the function and will not want
            to pay the price for the derivate;


    \item   compile with {\tt make}, execute a call to {\tt MMVII} command {\tt GenCodeSymDer},
            and compile again.
\end{itemize}


Also it's generally not necessary, we invite here the curious reader to give a look
at generated code.  All the such codes are located in {\tt src/GeneratedCodes/} ,
and here in the files  {\tt CodeGen\_cDist2DConsVal.cpp}, w/o derivative, and 
{\tt CodeGen\_cDist2DConsVDer.cpp}, with derivatives, the declaration of the classes
are in corresponding header file ({\tt CodeGen\_cDist2DConsVal.h} \dots). 
Note that the name used for file ans classes come from the result of {\tt FormulaName}.

Lets make a quick comment on {\tt CodeGen\_cDist2DConsVDer.cpp} :

\begin{itemize}
  \item  the computation is made in a loop {\tt for (size\_t aK=0;....)}, this 
         allow a paralelization of the code when high performance computation is
         required;

  \item  the name used for unkowns and observation can be recoginzed as local
         variable at the begining of the loop;

  \item  after the code is a very monotonous code with basic instructions as
         {\tt Fx = Fy op Fz;}  or {\tt Fx=Op(Fy);}  ,  this is not the way
         human would write code ...  however it is not made to be read and you may
         think of it as "high level assembleur code";

  \item  by the way the code is relatively optmized to avoid multiple redundant computation,
         for example you can see that formula {\tt F8}  that represent {\tt y1-y2} is computed
         once and used three time;

  \item  the result of one iteration contain $5$ value here, because we have the value of the function
         itself and the value of the derivative relative to the $4$ variable ($x$ and $y$ of
         $P_1$ and $P_2$) ;  

  \item  there is a high level interface for extracting independantly value and derivatives,
         but we will not need it here, as everything will be encapsulated in the main 
         class {\tt cResolSysNonLinear};

  \item  give a look at file {\tt CodeGen\_cRatioDist2DConsVDer.cpp}, in this case we have $21$
         values ; $21 = 3 * (1 + 3*2)$ , the ratio return a vector of $3$ values, and
         for each value we compute the value itself and its derivatives relative to the $6$
         variable ($x,y$ for $3$ points), again there is a high level interface for extracting
         these values.
\end{itemize}

          %  - - - - - - - - - - - - - - - - - - - 

\subsection{Creating the object}

\label{CreateCalc}

For computig the value and derivative of {\tt cDist2DConservation} in our \CPP program,
 a possible and classical way is to include the file declaring the class
(i.e. {\tt CodeGen\_cDist2DConsVal.h}) and explicitely create a {\tt cDist2DConsVDer}.

However the optimizer is done to work with any object deriving from the abstract mother
class of all object resulting from code generation :  {\tt cCompiledCalculator<double>};
so the optimizer only manipulate pointers on such class and the exact class is not used.
There is a function for creating an object from its name, and we create an allocator
from this name with the function {\tt  EqConsDist(bool WithDerive,int aSzBuf)}.

The definition in {\tt GenerateCodes.cpp }  is pretty basic :

\begin{lstlisting}
cCalculator<double> * EqConsDist(bool WithDerive,int aSzBuf)
{
    return cName2Calc<double>::CalcFromName(NameFormula(cDist2DConservation(),WithDerive),aSzBuf);
}
\end{lstlisting}


And the declaration in a header file (here {\tt MMVII\_PhgrDist.h} ) :


\begin{lstlisting}
NS_SymbolicDerivative::cCalculator<double> * EqConsDist(bool WithDerive,int aSzBuf);
\end{lstlisting}

Now from the user side, the only thing we need to do is calling  {\tt EqConsDist}. This
correspond in file {\tt BenchResolSysNonLinear.cpp}  to line:


\begin{lstlisting}
     mCalcD =  EqConsDist(true,1);
\end{lstlisting}


%---------------------------------------------
%---------------------------------------------
%---------------------------------------------

\section{Non linear system}

This section was written assuming that the user explicitely controls the numbering of all
the variable because this was the way it was done at the begining. Meanwhile, a mecanism
described in~\ref{SecAutoUkAlloc} has been added to facilitate this automatic numbering.
By the way the explicit numbering is still accessible, because it is believed that the
two mecanisms are useful.

\subsection{Introduction}

The class for solving  non linear system is the template class 
{\tt  cResolSysNonLinear<Type>}. This class provides an easy interface
for computing fonctions and their derivatives (via object resulting from code generation)
and weighted least squares in the aim of solving non linear systems.

The template of the class refers to the way linearized equation are store and solved.
Probably {\tt tREAL8=double}  would be a good default value, while {\tt tREAL16}
could be used when high accuracy is required.

The main operations that can be done with such solver are:

\UNCLEAR
\begin{itemize}
   \item create a solver with initial values of the unknowns and parameters for
          specifying the adjoint least square solver; 

   \item add directly  an equation on subset of unkowns \emph{w/o} temporary unknowns;

   \item add  an equation with a subset of unkowns \emph{with} temporary unknowns  to
         a structure that accumulate them;  then  add this structure to the solver;

   \item add equation fixing a given variable;

   \item acces to current solution;

   \item compute the next current solution and reinitialize the solver.

\end{itemize}


          %  - - - - - - - - - - - - - - - - - - - 

\subsection{Least square solving and constructor}

In {\tt src/TutoBenchTrianguRSNL/cMainNetwork.cpp} the call to these constructors can be found
after tags {\tt  BASIC:CONSTRUCTOR} and {\tt LEASTSQ:CONSTRUCTOR}.
The dense vector is created from a standard vector.\UNCLEAR

In {\tt MMVII} the core calculation of matrix algebra is realized
by eigen library. The services offered by {\tt MMVII} is essentially
an interface for storing data and for a more homegeneous integration
in {\tt MMVII} "philosophy".

In {\tt MMVII} there is several ways to handle least squares systems, each
one correspond to a value of enumeration {\bf \tt eModeSSR} :

\begin{enumerate}

    \item{\bf \tt eSSR\_LsqDense :}
          uses dense matrix for storing the normal matrix; for system not so spare,
          or for small systems, this is the most efficient way; typically if you have
          (many) thousands of limited size systems, this is the mode you must use; solving
          is made by calling \emph{"ldlt"} method of eigen (i.e. Robust Cholesky 
          decomposition of a matrix with pivoting); this classes use schur-complement
          for handling temporay unknowns;

    \item {\bf \tt eSSR\_LsqNormSparse :}
          uses sparse matrix for storing the normal matrix; this is probably the most
          general way and the most efficient in time and storage for most cases in 
          photogrammetry involving many pose estimations. Its
          drawback is that using normal equations increase the conditionning of the
          system, the potential problem increasing with number of unknowns; 
          solving is made calling  {\tt SimplicialCholesky} decomposition of eigen;

    \item {\bf \tt eSSR\_LsqSparseGC :}
          uses sparse representation that memorize all the individual obsertions;
          \emph{do not} use normal equations, the solving is made using
          {\tt LeastSquaresConjugateGradient} decomposition of eigen;
          \emph{do not} use Schur complement for temporary unknowns, they
          are processed like the other unknowns; according to eigen documentation
          this method is the more robust for poorly conditionned systems 
          compared to {\bf \tt eSSR\_LsqNormSparse}, for big
          sparse system with a high proportion of temporary unknowns, 
          it's certainly less memory  efficient and it's probably
          less CPU-efficient (but intensive test remain to be done);

\end{enumerate}

The basic constructor takes an enum value as parameter to specify
the least square solver and an initial value:

\begin{lstlisting}
       cResolSysNonLinear(eModeSSR,const tDVect & aInitSol);
\end{lstlisting}


It is also possible to create a solver with an explicit least square solver.
This is usefull especially with {\bf \tt eSSR\_LsqNormSparse} because the memory
allocation (still in construction) is more complex and may require more parametrisation
from the user.  \UNCLEAR



          %  - - - - - - - - - - - - - - - - - - - 

\subsection{Adding a basic equation}

\label{AddBasicEq}

The tag  {\tt  BASIC:CALC} in {\tt BenchResolSysNonLinear.cpp} contain an example of such use.
The method for adding an observation is named {\tt CalcAndAddObs} :

\begin{lstlisting}
    void   CalcAndAddObs(tCalc * aCalc,const tVectInd & aVI,const tStdVect& aVObs,const tResidualW & aWeighter= tResidualW());
\end{lstlisting}

The $4$ parameters are :

\begin{itemize}
   \item {\tt aCalc} is a calculator as described in~\ref{CreateCalc};

   \item {\tt aVI} is a std::vector of  int that contains the index of unknowns used;

   \item {\tt aVObs} is a std::vector of  {\tt double} that contains the observation;

   \item {\tt aWeighter} is an object used for computing the weight of the observation, 
         the default value  associate a constant weight $1$, we will not discuss more 
         this parameter at this step.

\end{itemize}

What is done is what can be expected :

\begin{enumerate}
   \item  use  vector {\tt aVI} and {\tt aVObs}  to fill the parameters
          of the functor {\tt aCalc}; for {\tt VI} the index are used read the 
          values of the current unknown

   \item  execute the computation of {\tt aCalc} , that must  have been created
          with the {\tt WithDer} option at {\tt true};

   \item  use the result of differentiation and the weighting computed by {\tt aWeighter}
          to add a linearized equation in the least square system.
\end{enumerate}


          %  - - - - - - - - - - - - - - - - - - - 

\subsection{Adding with schur complement}

The tag  {\tt  SCHUR:CALC} in {\tt BenchResolSysNonLinear.cpp} contain an example of such use.
Adding equation with temporary variables, is slightly more complex, as the elimination
can only be done once we have all the equations involving a given subset of unknowns.
So the computation is done in $2$ steps : (1) create a structure, give at this initialisation
step  the current values of unkonws (2) accumulation in a structure 
with {\tt AddEq2Subst}, (3) using this structure
for adding the equation on unknowns after having done the elimination of temporary unknowns with
{\tt  AddObsWithTmpUK}

The structure is {\tt cSetIORSNL\_SameTmp<Type>}, the method for accumulating 
equation has the signature :

\begin{lstlisting}
    void  AddEq2Subst (tSetIO_ST & aSetIO,tCalc *,const tVectInd &,
                       const tStdVect& aVObs,const tResidualW & aWeighter= tResidualW());
\end{lstlisting}

In this method {\tt aVObs} and {\tt aWeighter} are identic to the equivalent in
{\tt CalcAndAddObs}. For the others :

\begin{enumerate}
   \item the first one {\tt cSetIORSNL\_SameTmp<Type>}, is the accumulating structure, it
	   constructor takes current values of temporaries;

   \item  the vector {\tt aVI} contains the  unknown and temporary unkown,  
	   conventionnaly the numbering of unknown is made with negative numbers starting from
           $-1$ to distinguish them from standard unknown, in this example they are $-1$ and $-2$,
           standard unknown are processed as before; 
         s  ee {\tt aVIndMixt} after tag {\tt SCHUR:CALC}

   \item  the vector {\tt aVTmp} contains the  values of temporary unknowns;
\end{enumerate}

Once the equation have been accumulated, it is sufficient to call {\tt AddObsWithTmpUK}
with the structure as parameter.

          %  - - - - - - - - - - - - - - - - - - - 
          %  - - - - - - - - - - - - - - - - - - - 
          %  - - - - - - - - - - - - - - - - - - - 

\subsection{Equation fixing a variable}

          %  - - - - - - - - - - - - - - - - - - - 
\subsubsection{Weigthed version for standard variables}

\label{WeightedFixVar}
The tag  {\tt  EQ:FIXVAR} in {\tt BenchResolSysNonLinear.cpp} contain an example .

It currently happen that the solution we compute is undetermined
up to  certain transformation.  If we do nothing, the least square
system will be not inversible and this will create problems.
A current way  to overcome this difficulty is to fix a set
of arbitrary variables.  In our case, as said in~\ref{Eq:FixVarAB}, 
we fix arbirtrarily $x_A,y_A,x_B$.  

The method {\tt AddEqFixVar} can be used for that, it takes $3$ parameters :
the number $k$ of the var, the value $V$ we want to assign, and the weight $w$
of the equation.  It simply add the following term  to the minimization :

\begin{equation}
      w (X_k -V)^2   
\end{equation}

The variant {\tt AddEqFixCurVar} fix the value of $x_k$ to its current value.

          %  - - - - - - - - - - - - - - - - - - - 

\subsubsection{Rigid fixing of a variable}

Sometime, we want to fix rigidly a variable to a given value.  Doing it with
a very high weight (almost "infinite") using method of~\ref{WeightedFixVar} would not be very 
good idea in general because the notion of very high weight is dependant of the context
and of the "dynamic" of the variable, and setting it to a universally  high value would lead to numerical 
instability.

In this case, the {\tt SetFrozenVar} familly must be used. The way it works is  totally
different of {\tt AddEqFixVar} and is the following :

\begin{itemize}
    \item {\tt MMVII} keep memory of all variable that have been frozen ;
    \item each time a linearized equation is added $\sum a_k X_k = B $ , for
             each $k$ where the variable $X_k$ is frozen to $F_k$ , $a_k$ is set to $0$ and $B$
             is set to $B - a_k F_k$;

    \item   also at the end (before solving) we had the equation $X_k=F_k$ with a weight of $1$
            for each frozen variable.
\end{itemize}

To have a correct behaviour of this "interception" mecanism it is required that all the frozen variable
are known before any equation is added. This constraint is tested by {\tt MMVII} and a dynamic
error will occur if this is not respected (see {\tt AssertNotInEquation}).

There is two method using explicit numbering :
\begin{itemize}
    \item {\tt void  SetFrozenVar(int aK,const  Type \&);} froze to an explicit value;
    \item {\tt void  SetFrozenVarCurVal(int aK);} froze to current value.
\end{itemize}

There exist also a familly of method for using with automatic numerotation, see~\ref{SecAutoUkAlloc}.


          %  - - - - - - - - - - - - - - - - - - - 

\subsubsection{Weighted/Froze, equation fixing a temporary}

\label{FrozeSetIORSNL}

In some case, it may be usefull to add a weighted fix observation  on temporary variable.
For example, if we have some ground observation on $3-d$ points associated to a tie-points.

This can be done, inside the {\tt cSetIORSNL\_SameTmp} with the method {\tt AddFixVarTmp}.

Also it may be useful to freeze the temporay if we want to reuse an existing calculator
in a context where some value are completely known \footnote{of course we could rewrite the
equation and set the unknown as observation, but this may be less convenient}.

This can be done at the creation of {\tt cSetIORSNL\_SameTmp} by adding an optional parameter :
the list of index that are frozen (it is also possible to change the value, but i really do see
why it should be useful !!!).

          %  - - - - - - - - - - - - - - - - - - - 

\subsection{Access to current solution}

There is two methods for accessing to the current solution of the system :

\begin{itemize}
       \item {\tt CurGlobSol()} : return globally the current solution (as a dense vector);

       \item {\tt CurSol(int k)} : return  the current value of $x_k$;

\end{itemize}

          %  - - - - - - - - - - - - - - - - - - - 

\subsection{Compute next iteration}

Once we have accumulated all the observation at a given step, what we classically
want to do is to :

\begin{itemize}
       \item use these observation  to have a better estimation of the solution by solving
             the least square system;

       \item supress  these observations of the system (reset it) because, due to linearisation,
             they were approximation, and we expect now to have a better approximation;

       \item use the computed solution in the next iteration as estimation for the linearization.
\end{itemize}

This is done by the method {\tt SolveUpdateReset() ;}.

%---------------------------------------------
%---------------------------------------------
%---------------------------------------------

\section{Image, optmization and differenciation in MMVII}

          %  - - - - - - - - - - - - - - - - - - - 
\subsection{Introduction}

In this section we present the facilities offered in MMVII when we want to mix non
linear optimisation with images. Typically example of such use are :

\begin{itemize}
     \item computing a deformation that transform a model of form  to an image;
           this is (will be) used  in MMVII for refining the  geometry of coded target detection;
           a very similar example, occurs in optical caracter recognition, if we know the font,
           the matching between a potential detectected character and the model ("Ab\'ec\'edaire" in french);

      \item this will/could be used (maybe) in matching method that used deformable mesh for modeling
            the geometry between different images (a well known variant being least-square matching);

      \item this could be used for deformable contour (a topics that used to be very active in 
            pattern recognition).
\end{itemize}

          %  - - - - - - - - - - - - - - - - - - - 
\subsection{Mathematical modelization}

The basic mathematical modelization is simply to consider images as functions.
As we deal with  continuous optimization, we need to have an interpolation scheme that
allow to consider them as function from $\RR^2 \rightarrow \RR$.
Ideally (to be mathematically consistant with derivation) we should use an interpolation 
model that is continuously derivable as bicubic interpolation. Also for efficiency reason,
we will generally prefer a non rigourous model as bilinear;  however the redundancy generally make
neglectable the difference with bicubic (probably, later, the bicubic option will be added for finer
tests).

A typical example is:

\begin{itemize}
      \item we have an image $I$, we note $I[i,k]$ its value for integer points;
      \item we extend  $I$ to a function $\RR^2 \rightarrow \RR$ via the interpolation;
      \item we have a mapping of  $\RR^2$,  $\phi : \RR^2 \rightarrow \RR^2$.
      \item we want to use the composed function $I \circ \phi$;
\end{itemize}

Let $p$ be a point and $q=(\tilde{x}_q,\tilde{y}_q)=\phi(p)$ its transformation by $\phi$. Let $x_0$ and $y_0$
be the lower bound of $q$, and $x_1=x_0+1, y_1 = y_0+1$:

\begin{equation}
	x_0 \leq \tilde{x}_q < x_1   \; and \;  y_0 \leq \tilde{y}_q < y_1
\end{equation}

Almost everywhere \footnote{except for integer values of $\tilde{x}_q,\tilde{y}_q$ }, 
the value of $I \circ \phi$ in a neighourhood of $p$ with  bilinear
interpolation is given by :

\begin{equation}
	I(\phi(p)) =   (x_1-\tilde{x}_q)(y_1-\tilde{y}_q) I[x_0,y_0]  
	             + (\tilde{x}_q-x_0)(y_1-\tilde{y}_q)I[x_1,y_0] \dots 
		     \label{Eq:Bilinear}
\end{equation}

We see with equation~\ref{Eq:Bilinear} that we don't need to add new primitive to compute $I \circ \phi$,
as it just a polynomial combination of existing primitives. By the way, reprogramming it
each time we need it would be fastidious and MMVII offers facility functions to make the use of 
formula~\ref{Eq:Bilinear}  easier.

          %  - - - - - - - - - - - - - - - - - - - 
\subsection{Facility functions}

\subsubsection{For code generation}

The facility functions are defined in the file {\tt "include/MMVII\_TplSymbImage.h"}. Note that this file 
must be explicitely included as it is not included by default in the library.

The template function {\tt FormalBilinIm2D\_Formula} is a direct implementation of equation~\ref{Eq:Bilinear}.
It will be used in the function generating code when we need things like $I \circ \phi$.
It takes two kind of parameters :

\begin{itemize}
   \item the parameters {\tt FX} and {\tt FY} correspond to function $\phi$ :  
	  {\tt FX $\sim \tilde{x}_q$} and {\tt FY $\sim \tilde{Y}_q$}, 
         they will be of type formulas; 

   \item the parameter {\tt aVObs}  corresponds to formula for the observations of equation~\ref{Eq:Bilinear},
         it contains $6$ values $x_0,y_0, I[x_0,y_0] \dots$ .
\end{itemize}

Fonction {\tt FormalBilinIm2D\_NameObs} just generate a vector of $6$ names, it use is recommande for standardizing generated code.
Note that it takes a prefix added to  each name, it will be useful in case we need to use several image in the same
formula to avoid name clash in generated code.
	

\subsubsection{For using generated code}

When using code generated, we need to fill the value of observation with values    $x_0,y_0, I[x_0,y_0] \dots$
in a given context.  The function {\tt FormalBilinIm2D\_SetObs} can be used to facilitate this filling .
More important it must be used to warantee the coherence of ordering with  {\tt FormalBilinIm2D\_Formula}.
The parameter are :

\begin{itemize}
    \item {\tt aVObs } is the vector to fill, {\tt K0} is the first index where filling begin;
    \item {\tt aPtIm}  is the point where want to evaluate $I$;
    \item {\tt aDIm}  is the image;
\end{itemize}


%---------------------------------------------
%---------------------------------------------
%---------------------------------------------

\section{Test example for image differenciation in MMVII}

In this section we describe a test example, that has been implemented in MicMac, this example 
has two objective : (1) make a didactic illustration of the fonctionnalities and (2) 
be usable in the automatic test of MMVII ("proof" of correctness and no regression).

\subsection{Mathemical problem}

We have :

\begin {itemize}
    \item a model $M$  function, this model function can be given by an analytic  formula or by a model image;
          in our case it will be a gaussian function;

    \item a parametric transformation of the model, here it is both geometric and radiometric;

    \item a ground truth for the parameters (used for generation and testing, ignored in computation);

    \item an image that contain the transformation of the model with the ground truth parameter;

    \item an initial guess of the parametric transformation, this guess being sufficiantly close to the
	    truth (whatever means "sufficiantly close");

     \item given the model, the image and the initial guess we want to recover the "exact" parameters of the 
           transformation.

\end {itemize}

For the model, we use  a smooth function to have an easy convergence (we dont want to 
make a theoretical analysis of robustness, just illustrate and test implementation).
More precisely we use a gaussian fonction :

\begin{equation}
	M(x,y) =  e^{-\frac{|p-\mu|^2}{\sigma^2}}
\end{equation}

The  parameter of transformation has 5 value :  $P=\{A,B,S,T_x,T_y\}$, where $\{A,B\}$ parametrises
the radiometric homotethy,  and $\{S,T_x,T_y\}$ parametrises the geometric homotethy.
So that :

\begin{equation}
	I[i,j] =  B + A * M(T_x + S*i,T_y+S*j) , (i,j) \in \NN^2  \label{EqDefIm}
\end{equation}

Knowing approximate guess  $\{A',B',S',T'_x,T'_y\}$, the model, the image and equation
\ref{EqDefIm} we want to recover "exact" value of $P$.

Note that we will use equation ~\ref{EqDefIm} in the invert sense of geometry, noting
the parameter of invert homothety $\{\tilde{S},\tilde{T}_x,\tilde{T}_y\}$:

\begin{equation}
	I(\tilde{T}_x + \tilde{S}*x,\tilde{T}_y + \tilde{S}*y) =  B + A * M(x,y) \label{EqDefImInv}
\end{equation}

So what we will estimate is $\{A,B,\tilde{S},\tilde{T}_x,\tilde{T}_y\}$,


\subsection{Generating the code}

The code for generating the automatic derivation applied to the example are located in two files:

\begin{itemize}
    \item {\tt src/SymbDerGen/FormulasImagesDeform.h } contains the definition of  class 
              {\tt cDeformImHomotethy} that will be used for generating the code, it is the homologous
              of class  {\tt cDist2DConservation} seen bellow;

    \item {\tt src/SymbDerGen/GenerateCodes.cpp} , as bellow, contains the code generate the code
          creating calculator;

\end{itemize}

There is no much commentary to do on the code of {\tt cDeformImHomotethy} who
should be explicit given previous section of this chapter.
Just focus on the point specific to bilinear interpolation :

\begin{itemize}
	\item for {\tt VNamesObs} we use the standard names of observation on bilinear 
              and add $3$ name specificto the model;

      \item for computing bilinear interopation of $I \circ \phi$ we use {\tt FormalBilinIm2D\_Formula};

      \item else the computation done in {\tt cDeformImHomotethy::formula()} is a direct "traduction"
	      of equations~\ref{EqDefImInv} and~\ref{Eq:Bilinear}.

\end{itemize}


\subsection{Using the generated code}

The file using the generated code for doing optimization can be found in 
{\tt src/Bench/BenchTutoImageDef.cpp}.  The main class is {\tt cTestDeformIm},
it main action are done in the constructor and the medthod {\tt OneIterationFitModele}.

The main action of constructor {\tt cTestDeformIm} are :

\begin{itemize}
    \item  inialize the ground truth homotethy and its inverse ({\tt mGT\_I2Mod} and {\tt mGT\_Mod2Im};
    \item  inialize the gaussian law in image an model geometry ( {\tt mGaussIm}  and {\tt mGaussModel});
    \item  initialize the image  in {\tt mIm}; 
    \item  put in vectors a set of point $P$ in model space and their associated value $M(P)$
           ({\tt mVPtsMod} and {\tt mValueMod});
\end{itemize}


In {\tt OneIterationFitModele} we parse all the point of the model, and for each point we
add an obsevation corresponding to equation~\ref{EqDefImInv}.

%---------------------------------------------
%---------------------------------------------
%---------------------------------------------

\section{Mechanism for automatic unknown allocation}

\label{SecAutoUkAlloc}

%---------------------------------------------

\subsection{General principle}

This section presents a mecanism to automate variable numbering in 
equation solver; this mecanism can be especially interesting when the same objects 
appears in different kinds of problems.

The general priniciple is:

\begin{itemize}
     \item all the objects that have unknowns must inherit from the class {\tt cObjWithUnkowns};

     \item this object must describe its unknowns as interval of {\tt double*};

     \item  all the objects of a same solver must be accumulated in an object of the class
            {\tt cSetInterUK\_MultipeObj};

      \item once this is done, the class offers many simplifications to communicate with a solver.
     
\end{itemize}

%---------------------------------------------

\subsection{Class {\tt cSetInterUK\_MultipeObj}}

This class organizes the "coordination" between all objects that will participate to the same solver.
Its main methods are:

\label{cSetIUK}

\begin{itemize}
    \item  {\tt  AddOneObj(cObjWithUnkowns<Type> * Obj);} , that must be called once with each object \emph{Obj},
           that has unknowns involved in the solver, the \emph{Set} will memorize the \emph{Obj},
           will make some initialization on  \emph{Obj} and then will call    the
           {\tt PutUknowsInSetInterval} on  \emph{Obj} 

    \item  {\tt  AddOneInterv} where an \emph{Obj}  will indicate its unknowns interval,
          this method will be  called by  \emph{Obj} inside its {\tt PutUknowsInSetInterval};
          the base methods take an adress {\tt double *} and a number, the other are
          just facility that recall this base method;
          
    \item  {\tt GetVUnKnowns}  return a vector of all the unknons and can be used to create a solver
           by giving the initial values;

    \item  {\tt SetVUnKnowns}  modify all the unknown with a vector (typically resulting from
           next iteration of the solver), will also call the method {\tt OnUpdate()} on all its \emph{Obj}.
\end{itemize}



%---------------------------------------------

\subsection{Class {\tt cObjWithUnkowns}}

\label{ClassOWU}

This is the base class for all object that will benefit from automatic ordering. It must override
the pure virtual method {\tt PutUknowsInSetInterval}, when this method is called by a set :

\begin{itemize}
   \item the protected field {\tt mSetInterv} will have been initialized 
   \item the object has just to call {\tt mSetInterv->AddOneInterv} with its unkonwn;
\end{itemize}

The other useful methods defined in the classe are :

\begin{itemize}
   \item    {\tt void PushIndexes(std::vector<int> \& VI);}  add the indexe of unknowns in  {\tt VI}
            to be used in equations in the solver (as with {\tt CalcAndAddObs}));

   \item    {\tt OnUpdate()}, this virtual callback will be used after the object has been
            modified by   {\tt SetVUnKnowns} .

\end{itemize}

There is different methods to access to the underlying numerotation of variables
({\tt IndOfVal, IndUk0} \dots) but these methods are to be used by the solver and their
direct use by the "end programmer" is not recommanded.

Instead of accessing directly to the numerotaion, the class {\tt cResolSysNonLinear}
 offers different facility freezing variable, they  take an object and the address to freeze.
See {\tt void SetFrozenVar(tObjWUk \& anObj, const Type \& aVal)} and after.

%---------------------------------------------
%---------------------------------------------
%---------------------------------------------

\section{An Example  with bundle adjusment}

\subsection{Global presentation}

We take the example of two classes where this mecanism is used ,
{\tt cPerspCamIntrCalib} for representing internal calibration of central pesrpective
camera, and {\tt cSensorCamPC}  for representing the pose of camera (contains
the pose itself + a refererence to the calibration).
These two classes are derivates classes of {\tt cObjWithUnkowns<tREAL8>}.

The code corresponding can be found in :

\begin{itemize}
   \item {\tt MMVII\_PCSens.h} for declaration of classes;

   \item {\tt cSensorCamPC.cpp}  and {\tt cCentralPerspCam.cpp} for definition of classes;

   \item {\tt cConvCalib.cpp}  that contain  class {\tt cCentralPerspConversion}.
\end{itemize}


We will study with more detail the class {\tt cCentralPerspConversion}, the purpose
of this class is to create a perspective camera from  as set of $3d-2d$ correspondance.
This is done by a bundle adjusment using colinearity equation where internal and
external parameter are optimized to fit the observations.
Note that this class is used in two different context :

\begin{itemize}
   \item first to do real conversion,  it is used for example by the command {\tt OriConvV1V2}
   \item second  it is use ti make test/bench; 
\end{itemize}

In second case, we generate artificially difficult condition : noisy initialization, but
also we dont give to the system all the information we have, but a sufficient subset, to
check that it still converge to the good solution. This is controled by two boolean
variable that are set to true in real conversion: 

\begin{itemize}
   \item {\tt  mFGC}   indicate if $3d$ are frozen, as it could be with $100\%$ reliable ground control point
                       
   \item {\tt mCFix} indicate is center of camera is frozen;
\end{itemize}


%---------------------------------------------
\subsection{Camera classes viewpoint}


\subsubsection{Class {\tt cSensorCamPC}}

The unknowns that are specific to a pose are the center  and  the rotation (coded as an axiator $\omega$ ).
The  formalisation is that the unknon rotation is the product of the currant rotation
and the a small rotation corresponding to the axiator ($X \rightarrow X + \omega\; \hat{}\; X$).
So, as a    {\tt cObjWithUnkowns<tREAL8>} the cSensorCamPC overrides {\tt PutUknowsInSetInterval} :

\begin{lstlisting}
void cSensorCamPC::PutUknowsInSetInterval()
{
    mSetInterv->AddOneInterv(mPose.Tr());
    mSetInterv->AddOneInterv(mOmega);
}
\end{lstlisting}

Be aware that the order in which we add these unknown has to be coherent with their use in calculator.
Also once the system has evolved, we still to udate $\omega$  and the current rotation
so that the unkwon rotation coded by $\omega$  remains "tiny". This is done  by
overriding {\tt OnUpdate} :


\begin{lstlisting}
void cSensorCamPC::OnUpdate()
{
     mPose.SetRotation(mPose.Rot() * cRotation3D<tREAL8>::RotFromAxiator(-mOmega));
     mOmega = cPt3dr(0,0,0);
}
\end{lstlisting}

\subsubsection{Class {\tt cPerspCamIntrCalib}}

The unknown specific to internal calibration of a central perspective camera are focal, principal 
point and parameters of distorsion. 


\begin{lstlisting}
void cPerspCamIntrCalib::PutUknowsInSetInterval()
{
    mSetInterv->AddOneInterv(mCSPerfect.F());
    mSetInterv->AddOneInterv(mCSPerfect.PP());
    mSetInterv->AddOneInterv(VParamDist());
}
\end{lstlisting}

If the camera is modified, the pseudo inverse is no longer valid, so we have to update it :

\begin{lstlisting}
void cPerspCamIntrCalib::OnUpdate()
{
    mInv_CSP       = mCSPerfect.MapInverse();
    if (mInvApproxLSQ_Dist!=nullptr)
       UpdateLSQDistInv();
}
\end{lstlisting}

\subsection{Detailled comment}

The code of {\tt cConvCalib.cpp} has been tagged with several commentary begining by {\tt \#DOC}.
We now describe the link between these tags and several sections of this chapter :

\begin{itemize}
    \item  {\tt  \#DOC-AddOneObj} describe the insertion of object with unknown in a coordinator
           as described in~\ref{cSetIUK};

    \item  {\tt  \#DOC-GetVUnKnowns} describe how we can extract a vector of unknowns
           from the coordinator, as described in~\ref{cSetIUK}, and use it to create a solver;

    \item {\tt   \#DOC-FixVar} describes freezing of some variable of an object as described in~\ref{ClassOWU};

    \item {\tt  \#DOC-PushIndex} describes how to complete the vector of indexes as describes in~\ref{ClassOWU};

    \item {\tt \#DOC-FrozTmp } describes how the value of temporary variable can be frozen using the
          constructor of {\tt cSetIORSNL\_SameTmp} as described in~\ref{FrozeSetIORSNL};
          in this example we freeze the $3$ coordinate of the point, this allow to use the same
          collineraity equation with tie points and ground control points.

    \item {\tt  \#DOC-SetUnknown} describes how we use the result of one iteration of the solver
          to modify all the object handled by the coordinator.

\end{itemize}

%---------------------------------------------
\subsection{Recommandation and limitation}

It is hope that the mecanism will significatively simplify the use of solver.

It is highly recommanded that inside the same solver, the developper does nor mix the two
mode (explicit and implicit numerotation);

A current limitation is that an object can belong only simultaneously to one coordinator.
If we need to change the coordinator the object, which can often happens, the coordinator 
must be destroyed (because at the destruction all object are reset from their link to the
coordinator). If the need arrise to have simultaneous  coordinator, there is possible solution,
then contact MPD.



%---------------------------------------------
%---------------------------------------------
%---------------------------------------------

\section{Topometric compensation (WIP)}

\subsection{Global presentation}

The premises of a topometric compensation system can be found in \texttt{MMVII/src/Topo/}.

Computation takes place in a 3D cartesian frame. For now, there is no georeferencing
and no Earth model.

All points must be given approximative coordinates as there is no automatic initialization.

For now it is only used in the Bench \texttt{TopoComp}.
This Bench will be used to illustrate the topometric classes and their usage.


\subsection{\texttt{TopoComp} Bench example}
\label{subsec:topoBench}

The example is created by the method \texttt{cTopoComp::createEx1()}.

\subsubsection{Step 1}

At first, there are 3 fixed points ($A$, $B$, $C$), forming an isosceles triangle
on an horizontal plane (Fig. \ref{fig:topoEx1}).

\begin{figure}[!h]
\centering
\includegraphics[width=12cm]{Programmer/benchtopo1.png}
\caption{The 3 fixed points}
\label{fig:topoEx1}
\end{figure}

The points are instances of the \texttt{cTopoPoint} class that
derives from \texttt{cObjWithUnkowns}, hence it has no copy constructor and
the points have to be added as pointers to \texttt{cTopoComp::allPts()}.

\begin{lstlisting}
  //create fixed points
  allPts.push_back(new cTopoPoint("ptA", cPt3dr(10,10,10), false));
  allPts.push_back(new cTopoPoint("ptB", cPt3dr(20,10,10), false));
  allPts.push_back(new cTopoPoint("ptC", cPt3dr(15,20,10), false));
  auto ptA = allPts[0];
  auto ptB = allPts[1];
  auto ptC = allPts[2];
\end{lstlisting}

At this point, there are no unknowns and no observations.


\subsubsection{Step 2}

A fourth point ($D$), this time not fixed, is initialized above the $ABC$ triangle.
Distances from $A$, $B$ and $C$ to $D$
are measured. For redondancy and error evaluation, the distance from $C$ to $D$ is measured twice
with different values (Fig. \ref{fig:topoEx2}).

\begin{figure}[!h]
\centering
\includegraphics[width=12cm]{Programmer/benchtopo2.png}
\caption{Point $D$ is determined by measured distances}
\label{fig:topoEx2}
\end{figure}

Observations must refer to an observation set (deriving from class \texttt{cTopoObsSet})
that is used to share parameters between observations.

In the case of measured distances there is no parameter, therefore the observation set will be 
a \texttt{cTopoObsSetSimple}. Observations sets must be created with \texttt{make\_TopoObsSet} template function.

All topometric observations are instances of \texttt{cTopoObs} class.
For distances observations, \texttt{TopoObsType::dist} is given to the constructor.

\begin{lstlisting}
  //add measured dist to point D
  allObsSets.push_back(make_TopoObsSet<cTopoObsSetSimple>());
  auto obsSet1 = allObsSets[0].get();
  allPts.push_back(new cTopoPoint("ptD", cPt3dr(14,14,14), true));
  auto ptD = allPts[3];
  cTopoObs(obsSet1, TopoObsType::dist, std::vector{ptA, ptD}, {10.0});
  cTopoObs(obsSet1, TopoObsType::dist, std::vector{ptB, ptD}, {10.0});
  cTopoObs(obsSet1, TopoObsType::dist, std::vector{ptC, ptD}, {10.0});
  cTopoObs(obsSet1, TopoObsType::dist, std::vector{ptC, ptD}, {11.0});
\end{lstlisting}

The system now contains 3 unknowns ($D$ coordinates) and 4 observations
(measured distances $AD$, $BD$ and $CD$ two times).

\subsubsection{Step 3}

A second non-fixed point, $E$, is initialized inside the $ABCD$ tetrahedron,
and observations expressing that
the distances from $E$ to $A$, $B$, $C$ and $D$ are equal and unknown ($d$ in Fig. \ref{fig:topoEx3})
are added.

\begin{figure}[!h]
\centering
\includegraphics[width=12cm]{Programmer/benchtopo3.png}
\caption{Point $E$ determined by equal distance $d$}
\label{fig:topoEx3}
\end{figure}

A new \texttt{cTopoObsSet} must be created. The \texttt{cTopoObsSetDistParam} class
is used to create the parameter corresponding to the unknown $d$.


\begin{lstlisting}
  //add point E to an unknown common dist
  allObsSets.push_back(make_TopoObsSet<cTopoObsSetDistParam>());
  auto obsSet2 = allObsSets[1].get();
  allPts.push_back(new cTopoPoint("ptE", cPt3dr(11,11,11), true));
  auto ptE = allPts[4];
  cTopoObs(obsSet2, TopoObsType::distParam, std::vector{ptE, ptA}, {});
  cTopoObs(obsSet2, TopoObsType::distParam, std::vector{ptE, ptB}, {});
  cTopoObs(obsSet2, TopoObsType::distParam, std::vector{ptE, ptC}, {});
  cTopoObs(obsSet2, TopoObsType::distParam, std::vector{ptE, ptD}, {});
\end{lstlisting}

This adds 4 unknowns ($E$ coordinates and the unknown distance $d$) and 4 observations
($EA = EB = EC = ED = d$).


\subsection{Topometric computation}

Topometric computation system is managed by the \texttt{cTopoComp} class.
This handles the list of points and observation sets, the non-linear system
(\texttt{cResolSysNonLinear}), and a mechanism to link the observations to
the automatic derivation system.

After \texttt{cTopoComp} object creation and filling (see \ref{subsec:topoBench}),
the method \texttt{OneIteration()} is called to improve parameters estimation.

%---------------------------------------------
%---------------------------------------------
%---------------------------------------------

\section{Handling constrained optimization}

%---------------------------------------------

\subsection{Theoretical aspect}

%  -  -  -  -  -  -  -  -  - -  -  -  -  -  -  -  -  - -  -  -  -  -  -  -  -  - -  -  -  -  -  -  -  -  -

\subsubsection{Introduction}

We consider the problem of minimizing $F(X)$ as in equation~\ref{EqNLOInit} under
the $n$ constraints:

\begin{equation}
	C_1(X) =0, \; C_2(X)=0 \dots \;  C_n(X)=0
\end{equation}

In  equation~\ref{EqNLOInit} we consider that the current solution is
close to the optimal solution and the equation can be linearized.
Similarly we consider that the current solution is close to the 
optimum under constraint, and consequently that the constraints are almost satisfied.
We consider then that the constraints can be linearized :

\begin{equation} 
C_k(X) \approx L_1 \cdot X  - c_k
\end{equation} 

So we can write :

\begin{equation} 
\begin{split}
  L_1 \cdot X = c_1 \\
  L_2 \cdot X = c_2 \\
    \dots \\
  L_n \cdot X = c_n 
\end{split}
\end{equation}

%  -  -  -  -  -  -  -  -  - -  -  -  -  -  -  -  -  - -  -  -  -  -  -  -  -  - -  -  -  -  -  -  -  -  -

\subsubsection{"Hard" vs "Soft" constraint}

What we call "soft" constraint is the method consiting in adding
the constraint as a penalization, with a certain weigthing $w$  in optimization:

\begin{equation}
    F_w(X) = F(X) + w  \sum_k (L_k-c_k)^2 \label{SoftConstraint}
\end{equation}

Soft constraints are just observations like others and do not require
to add any supplementary method. If what we want is hard constraints, an
easy way could be to use equation~\ref{SoftConstraint} with a "very high"
value for $w$.   Mathematically speaking, except for pathological cases,
generally when $w \rightarrow \infty$, the solution of ~\ref{SoftConstraint}
will converge to the solution of the hard constraint.

Although it can work, this cannot be a general method, because generally
it is difficult, if not impossible to define a \emph{"good very high $w$"}.
If it is not high enough, the solution will not be close enough to the hard
constraint.  It it is too high, it can create numerical inacuracy.

%  -  -  -  -  -  -  -  -  - -  -  -  -  -  -  -  -  - -  -  -  -  -  -  -  -  - -  -  -  -  -  -  -  -  -

\subsubsection{Lagrangian method}

To detail later : mention that it can be equivalent to solve the system with :

\begin{equation}
  \begin{pmatrix}
   A & L \\
   ^t L & 0 
  \end{pmatrix}
\end{equation}

But the system being no longer definite-positive, the resolution can be tricky.

%  -  -  -  -  -  -  -  -  -  -  -  -  -  -  -  -  -  - -  -  -  -  -  -  -  -  - -  -  -  -  -  -  -  -  -

\subsubsection{Substitution methods, case of $1$ constraint}
\label{CSTR:SUBST}

MMVII uses a substitution method. First, let's examine the case with $1$ variable, suppose:

\begin{itemize}
    \item we have a constraint $L \cdot X = c$;
    \item noting $L= (l_0,l_1 \dots)$ and suposing $l_0 \neq 0$,  note $L'=(l_1,l_2, \dots)$ and $X'=(x_1,x_2\dots)$;
\end{itemize}


The constraint can be written :
\begin{equation}
    x_0 = \frac{c-L' \cdot X'}{l_0} \label{LSQ:SUBST}
\end{equation}

Now  each time we add a new observation we will substitute $x_0$ :

\begin{itemize}
      \item let's note the observation  $Obs(X) = A \cdot X - C$
      \item with $A= (a_0,a_1 \dots)$, note  $A' = (a_1,a_2 \dots) $
       \item then   $Obs(x) = a_0 x_0 + A' X'-C= a_0\frac{c-L' \cdot X'}{l_0} + A'X' -C$
\end{itemize}

And finally:

\begin{equation}
    Obs(x) = (A'- \frac{a_0}{l_0} L') X'  - (C -c\frac{a_0}{l_0})
\end{equation}

So by doing the substitution for each observation we obtain a system with $N-1$ variables,
where $x_0$ has been eliminated, and without constraint. We can solve it, to find $x_1, x_2 \dots $,
and at the end, use equation~\ref{LSQ:SUBST} to find $x_0$.


%  -  -  -  -  -  -  -  -  - -  -  -  -  -  -  -  -  - -  -  -  -  -  -  -  -  - -  -  -  -  -  -  -  -  -

\subsubsection{Substitution methods, case of N constraints}
\label{CSTR:NSUBST}

We study the case with $2-3$ constraints, the generalization to $N$ constraints being straightforward.
Suppose we have two constraints $C_a$ and $C_b$ :

\begin{itemize}
    \item $C_a : L^a_0 x_0 +  L^a_1 x_1 \dots    = c^a$;
    \item $C_b : L^b_0 x_0 +  L^b_1 x_1 \dots    = c^b$;
\end{itemize}

For each observation, once we have made a substition with $C_a$ to eliminate $x_0$,
 we  want to use $C_b$ to eliminate $x_1$.
But for this we need to have $ L^b_0=0$, else it will "re inject" $x_0$.  To
force $L^b_0=0$, we remark  that for any $\lambda$ :

\begin{equation}
    C_a(X)=0  , C_b(X)=0   \Leftrightarrow  C_a(X)=0  , C_b(X)+\lambda C_a(X) =0
\end{equation}

So if we set $C'_b = C_b - \frac{L^b_0}{L^a_0} C_b$, we can now : use $C_a$ to eliminate $x_0$,
then use $C'_b$ to eliminate $x_1$. 

The idea is to make a "pre-processing" of constraints to allow a substitution in cascade.
If we have $3$ constraints $C_a, C_b, C_c$, we subract $C_a$ in $C_b$ and $C_c$, to have $C_a, C'_b, C'_c$
where $x_0$ is absent of $C'_b$ and $C'_c$. Then we substract  $C'_b$ to $C'_c$ to have a new constraint 
$C''_c$ where  $x_0$ and $x_1$ are eliminated.  For each observation, by substituting consecutively $C_a,C'_b,C''_c$, we can eliminate
$x_0$, $x_1$ and $x_2$.

This pre-processing of constraints is exactly the same than a gaussian elimination in linear resolution.
It requires some precaution to be stable, practically we make some "pivoting" to select the biggest coefficient
to eliminate.

%---------------------------------------------

\subsection{Using it in MMVII}

\label{Cstr:Use:MMVII}

%  -  -  -  -  -  -  -  -  -  -  -  -  -  -  -  -  -  - -  -  -  -  -  -  -  -  - -  -  -  -  -  -  -  -  -

\subsubsection{Restrictions/Potential Errors}

Although we think that this approach brings benefits, its use requires some precaution.
If not respected, it can rise several errors at execution of the program.

First and most important, the method requires that we know all the constraints
before any observation can be added.  This is handled by {\tt MMVII} using a boolean flag {\tt mInPhaseAddEq}.
This flag is set to {\tt false}  at the end of each iteration, and set to {\tt true} when
the first observation is added.  Once this flag is set to {\tt true}, no adding of constraint
will be allowed (see the method {\tt AssertNotInEquation}).

Another kind of error that can occur is to add too much constraints so that it becomes
impossible to comply with all of them. Of course this is not due to the way it is handled,
but is intrinsic to the notion of hard constraints. If this happens, an error
will be raised:

\begin{itemize}
     \item  see the method {\tt LinearMax} in class  {\tt cOneLinearConstraint};
     \item  an error with  message like {\tt "LinearMax probably bad formed constrained"} will be raised.
\end{itemize}

Note that this redundancy is detected from the structural point, not from the numerical point. For example
if we add $3$ constraints involving only the $2$ same unknowns an error will be detected; even if it is
$3$ times the same constraint (numerically it can be satisfied because if it's $3$ time the same, it is in
fact a constraint $1$ constraint, but "structurally" \emph{too many constraint for unknowns} will be detected).


Conversely, for example if $x$ and $y$ are unknown, if we add the $2$ constraint $x+5y=1$ and $\frac{x}{3}+ \frac{5y}{3}=0$,
in exact arthimetic it's is impossible to comply with these two constraint; but due to rounding error, it's likely
that no contradiction will be detected by the programm (but we will have a very unstable elimination).

%  -  -  -  -  -  -  -  -  -  -  -  -  -  -  -  -  -  - -  -  -  -  -  -  -  -  - -  -  -  -  -  -  -  -  -

\subsubsection{Handling "frozen" variables}

The most current use of constraint is the case where we want to impose that a given unknown
has a given value (its current value or a new value). The constraint is simply
$X_i=C$.   There are several methods to facilitate this manipulation in class {\tt cResolSysNonLinear}, for
the freezing part we have:

\begin{itemize}
     \item   {\tt SetFrozenVarCurVal(int aK)} freezes the value of an unknown to its current value,
              knowing the number of the unknown in the system;
     \item   {\tt SetFrozenVarCurVal(tObjWUk \& anObj,const  Type \& aVal)}, idem, but specify the object {\tt anObj}
	     and the adress of the value {\tt aVal} that must be \emph{"inside"} the object;
     \item  several variants freezing several unkowns of an object (using points or adresses + a number) or all the
            object;
      \item  {\tt SetFrozenVar(int aK,const  Type \&)}, to force an unknown to have a given value
             (i.e. \emph{a priori} different from its current value).

\end{itemize}

For the unfreezing part we have some similar methods:

\begin{itemize}
	\item   {\tt SetUnFrozen(\dots)} suppresses the constraint of an unknown (idem, with method knowing its number
		and method with an object and one address);
      \item  {\tt UnfrozeAll()} suppresses all the constraints on freezing any variable.
\end{itemize}

%  -  -  -  -  -  -  -  -  -  -  -  -  -  -  -  -  -  - -  -  -  -  -  -  -  -  - -  -  -  -  -  -  -  -  -

\subsubsection{Handling "shared" unknown}

Sometimes it occurs  that for a set of unknwons, theoretically different,  we want to enforce
temporarilly these unknowns to have the same value: that's what we call shared unknowns. 
A possible example is with several cameras where we want each camera to have its own focal length
but want them to have the same distorsion, and this distorsion
has to be adjusted.  \emph{unknown sharing} will allow to do that without requiring to add a specific model of camera.

From the theoretical point of view, it's quite direct to impose shared unknowns  with constrained optimization.  For forcing
$n$ unknows to have a shared value : $X0=X1=X2\dots=X_{n-1}$, we simply add $n-1$ equations selecting
an arbitrary reference variable $X_0$ : $X_0-X_1=0$, $X_0-X_2$=0, \dots , $X_0-X_{n-1}=0$.
The methods in class {\tt cResolSysNonLinear} for manipulating shared unknown are:

\begin{itemize}
    \item   {\tt SetShared(const std::vector<int> \&  aVUk)}  given a vector of unknowns numbers,
            make them a set of unkowns shared;
    \item   {\tt SetUnShared(const std::vector<int> \&  aVUk)} make the invert operation;
    \item   {\tt SetAllUnShared()} 
\end{itemize}


%  -  -  -  -  -  -  -  -  -  -  -  -  -  -  -  -  -  - -  -  -  -  -  -  -  -  - -  -  -  -  -  -  -  -  -

\subsubsection{Handling general case, linear and non linear}

Beside the special case of unknowns freezing or sharing, there are methods for handling general constraints.
For the linear case we have the method {\tt AddConstr} :

\begin{itemize}
    \item {\tt AddConstr(const tSVect \& V,const Type \& C,bool OnlyIfFirstIter=true);}
    \item  {\tt V} is the linear part , {\tt C} is the constant, it correspond to the equation $ V \cdot X = C$ ;
    \item  the parameter {\tt  OnlyIfFirstIter} indicate that the constraint must be added only if we are at the
           first iteration of the non linear system.
\end{itemize}

For non linear constraints we have the method {\tt AddNonLinearConstr}.  It's very similar to {\tt CalcAndAddObs}
(see \ref{AddBasicEq}), it takes a calculator corresponding to a funcion $F$ to compute the value 
and its derivatives and generate a linearized version of the constraint $F(X)=0$:

\begin{itemize}
    \item {\tt  void AddNonLinearConstr(tCalc * aCalcVal,const tVectInd \& aVInd,const tStdVect\& aVObs,bool  OnlyIfFirst=true);}

    \item  {\tt aCalcVal,aVInd, aVObs} play the same role than  in {\tt CalcAndAddObs};

    \item  the parameter {\tt  OnlyIfFirstIter} play the same role than in {\tt AddConstr}. 
\end{itemize}

The method {\tt SupressAllConstr()} suppresses all the constraints
added by {\tt AddConstr}  or {\tt AddNonLinearConstr}.

%  -  -  -  -  -  -  -  -  -  -  -  -  -  -  -  -  -  - -  -  -  -  -  -  -  -  - -  -  -  -  -  -  -  -  -

\subsubsection{Example in bench}

In the network bench the method {\tt AddGaugeConstraint} gives examples of using 
the constraints to fix the arbitrary rotation that cannot be fixed by distance conservation.
It also serves as an unit test of correctness.  The "natural" way, already described, is to fix
$3$ coordinates $X_{0,0},Y_{0,0},X_{1,0}$ with soft or hard constraints (using  {\tt AddEqFixVar} or  {\tt SetFrozenVar}).

The other tested methods are completely artificial and activated with the flag {\tt doMangleCstr},
this test is done using constraints involving the neighborhood of a point.
Let $X_k$ be the unknowns of the neighbouring points, and $R_k$  be the 
ground truth value of the $X_k$.

For testing the linear constraints, with {\tt AddConstr} we generate a random weighting $W_k$
and we add the linear constraint:

\begin{equation}
    \sum_k W_k  X_k = \sum_k  W_k  R_k  
\end{equation}

For the non linear constraint, we use the following equation:

\begin{equation}
    \sum_k  X_k^2 = \sum_k   R_k^2
\end{equation}

%---------------------------------------------

\subsection{Implementation details}

%  -  -  -  -  -  -  -  -  -  -  -  -  -  -  -  -  -  - -  -  -  -  -  -  -  -  - -  -  -  -  -  -  -  -  -

\subsubsection{Global presentation }

This part is for the programmer that will need to maintain and make evolve the core system.
For the programmer that "only" want to use it in optimisation in {\tt MMVII}
the interaction with {\tt cResolSysNonLinear}, described 
in \ref{Cstr:Use:MMVII}, is sufficient.

The code relative to constrained optimization is located in the files {\tt Matrix/LinearConstraint.h}
and  {\tt Matrix/cLinearConstraint.cpp}. The classes involved are:

\begin{itemize}
    \item  {\tt cOneLinearConstraint}  for representing a single constraint;
    \item  {\tt cSetLinearConstraint}  for representing a set of constraint  : initial value and value after
           pre-processing described in~\ref{CSTR:NSUBST};
    \item  {\tt cDSVec}   helper class allowing easy manipulation of sparse vector;
    \item  {\tt cBenchLinearConstr}   class for doing some unitary test on {\tt cSetLinearConstraint}.
\end{itemize}

%  -  -  -  -  -  -  -  -  -  -  -  -  -  -  -  -  -  - -  -  -  -  -  -  -  -  - -  -  -  -  -  -  -  -  -

\subsubsection{Class {\tt cDSVec} }

By itself the method of subsitution is relatively simple. But there is some complexity added 
for efficient implementation :

\begin{itemize}
    \item  in many context where non linear optimization is used, including photogrammetry,  each observation
           imply few unknowns (says $\approx 10-50$)  among many (say $\approx 500-100000$);

    \item  for efficient representation we use sparse vector for constraint and observation, 
           a sparse vector being a collection of pair index/value;

     \item let name $m$  the number of pair of a sparse vector and $N$ the number of unknowns;

    \item  many operation on sparse vector, like substitution described in~\ref{CSTR:NSUBST}, require some
           precaution if we want that the cost to be in $\mathcal{O}(m)$  rather than in $\mathcal{O}(N)$.
\end{itemize}


This is here that  {\tt cDSVec}  can help. It is a class for dense representaion of sparse vector. Typically a  {\tt cDSVec}:

\begin{itemize}
    \item  contains a dense real vector $V$ (initialy full of $0$), a list of used indexes $L$ (initially empty), a dense boolean
           vector indicating if the index is used $B$ (initially full of false) 
    \item  each time an element is added-suppressed on the {\tt cDSVec} at index $i$, the dense vector is updated 
           at $V[i]$ and if $B[i]$ is false (meaning it has not been used) then $B$ and $L$ are also updated;
    \item  a {\tt cDSVec} can be quickly reset to its initialvalue, by parsing $L$ to : set $V$ to $0.0$, set $B$ to false,
            and finally clearing $L$.
    
\end{itemize}

Typically, as the allocation of a dense vector can take some time (in  $\mathcal{O}(N)$) the idea is to allocate
a dense vector in class {\tt cSetLinearConstraint} and to reuse it many times and reset it at the end of each use
(that's a buffer).


%  -  -  -  -  -  -  -  -  -  -  -  -  -  -  -  -  -  - -  -  -  -  -  -  -  -  - -  -  -  -  -  -  -  -  -

\subsubsection{Class {\tt cOneLinearConstraint} }


The class {\tt cOneLinearConstraint} is used to represent one linear constraint (who would believe that !! \dots  :-;).
It contains a sparse vector {\tt mLP}$=V$ and a constant {\tt mCste}$=C$. At its creation it
represents directly the constraint  $V \cdot X = C$.

Once it has been "reduced" (i.e. the substitution described in  \ref{CSTR:SUBST} has been applied),
the {\tt mISubst}$=i$ contains the number of the unknowns to substitute, and {\tt mLP}$=V'$ no longer the
pair containing $I$, the constraint is then $V' \cdot X + X_i = C$.  The boolean {\tt mReduced}
indicate if the constraint was reduced.

%  -  -  -  -  -  -  -  -  -  -  -  -  -  -  -  -  -  - -  -  -  -  -  -  -  -  - -  -  -  -  -  -  -  -  -

\subsubsection{Class {\tt cSetLinearConstraint} }

The class {\tt cSetLinearConstraint} is the  main class, i.e the only class the
"rest of the word" needs to interact with. It contains essentially the following data:

\begin{itemize}
    \item  a copy of the initial version of the constraints;
    \item  a copy of reduced versions;
    \item  a bufffer of type {\tt cDSVec} to accelerate the computation;
\end{itemize}

A typical sequence of use will be:


\begin{itemize}
    \item   create the object, indicating the number of unknowns (to allocate the buffer);
    \item   add a number of $M$ constraints (using {\tt Add1Constr} or {\tt Add1ConstrFrozenVar});
    \item   "compile" the object, this means apply the processing of~\ref{CSTR:NSUBST} to create the
	    reduced constraint in {\tt mVCstrReduced};
    \item   each time an observation is added, use one of the $3$ following methods to eliminate the $M$ variables 
	    selected in {\tt Compile} :
	    \begin{itemize}
                 \item {\tt SubstituteInSparseLinearEquation}; 
                 \item {\tt SubstituteInDenseLinearEquation} 
                 \item {\tt SubstituteInOutRSNL}
	    \end{itemize}
    \item    use the method {\tt void AddConstraint2Sys(tLinearSysSR \&)}, this method simply add all the constraint
             as linear observation, this is necessary to fix the value of the substituted unknowns (because by 
             construction they have been eliminated).

\end{itemize}







\chapter{Symbolic Derivation}

We give a detailled description of how automatic differentiation is handled in {\tt MMVII}.
This description is for programmer who will have to maintain this part or possibly also for curious readers .
However we recommand to read before the short introduction given in~\ref{Compute:Deriv:SysNL} (or better all chapter~\ref{Chap:NLO}) to
have a better understanding of where we go.

For devlopper that will only be users of these  automatic differentiation facilities in {\tt MMVVII} , probably
~\ref{Compute:Deriv:SysNL} or~\ref{Chap:NLO} is sufficient.


%-----------------------------------------------------
%-----------------------------------------------------
%-----------------------------------------------------

\section{introduction}


%-----------------------------------------------------
\subsection{Motivations}

The computation of derivatives of a given function is a central point in metrology
where it is used in Gauss-Newton-like iteration.  There is basically $4$ methods
for computing derivatives :

\begin{itemize}
    \item  computing "by hand", i.e a human apply the rules of derivation as
           learned at school;  for small formula, this method is perfectly manageable 
           and also probably the fastest when the code is optimized;
           but with "complicated" function, as colinearity equation with many distorsion parameters,
           it is error prone, difficult to maintain, and potentially not so fast as there
           are simplifications that human may miss;

    \item  numericall derivative: for example with a $2$ variables function, we use equation as~\ref{NumDer}
           where $\epsilon_x$ is a "small" value relative to variable $x$;
           this method has the advantage of simplicity, if you know the function you know its derivatives;
           it has $2$ drawbacks :  first issue, it's not always easy to define a good small value, a too small value
           creates numericall problem, not enough small is unacurate, all the more with heterogeneous variable the "small"
            value must be re-defined for each variable;  second issue, it is relatively slow, with $N$  variables
           we must have $2N$ evaluations of $F$;

     \item jet method as used in CERES, see~\cite{CERES} for detailled description, it's not error
           prone as "hand crafted", there is no problem of accuracy, they are faster than 
           numerical derivatives; however they are relatively slow compared to formal methods;

   \item formal method, that "more or less" do the same thing than "human computation" (generate the
	   analyticall formula) but do
           it automatically; they are separated between automatic differenciation that analyse the code of a programm
           and symbolic differenciation that construct a tree representation, but to our mind this separation
           is a bit artificial : they are conceptually close and 
           their performance are similar;  {\tt MMVII} uses symbolic differenciation;
           the use is a bit more complicated than jet but can be significatively faster  on complicated
           formulas.
\end{itemize}

\begin{equation}
    \frac{\partial F(x_0,y_0)}{\partial x} \approx \frac{F(x_0 + \epsilon_x,y_0) - F(x_0-\epsilon_x,y_0)}{2* \epsilon_x}
     \label{NumDer}
\end{equation}

%-----------------------------------------------------
\subsection{Code localization}

The code on automatic differenciation has been organized in such a way that 
the code can be reused outside the {\tt MMVII} library. It's a header only code
that is located in {\tt MMVII/include/SymbDer/}.


The code that use it more specifically for task adressed in  {\tt MMVII},
as photogrammetry or compensation in general, are to be found in the
folder  {\tt MMVII/src/SymbDerGen}.


%-----------------------------------------------------
\subsection{Tree/DAG representation}

\begin{figure}
\centering
\includegraphics[width=12cm]{Programmer/ImagesProg/Tree.jpg}
\caption{Representation of formulas}
\label{fig:TreeFormula}
\end{figure}

A classic aproach in computer science is to represent mathematicall formulas 
as trees (or more precizely as we will see as "DAG"=  directly acyclic graph).
These trees can be described recursively :

\begin{itemize}
   \item  constant ($0,1,\pi \dots$)  and fonctions corresponding to variables ($x_1,x_2,\dots ,x_{144} \dots$)
          are  atomic formulas, their tree has one node labeled by the contant or the variable;

   \item if $Op$ is a unary operator, like $\cos, \sin, \exp $ for example,  and $F$ is a formula then $Op(F)$ is a 
         formula, its tree is a node labeled by $Op$ and has one son corresponding to the tree of $F$;

   \item if $Op$ is a binary operator, like $+,*,pow $ for example,  and $F_1$  and $F_2$ are 
	   formulas then $Op(F_1,F_2)$ (or $F_1 Op F_2$ in infix notation) is a 
         formula,  its trees is a node labeled by $Op$ and has two son corresponding to the trees of $F_1$ and $F_2$.
\end{itemize}


Figure~\ref{fig:TreeFormula} represent the tree for the formula $F(x,y) = (x+y)^2 +x - \cos(x+y)$. 
In the formula, we see that the sub-formula $x+y$ appears twice , then to optimize the computation,
a directed acyclic graph (DAG) is prefered to tree;  a DAG  is still a hierarchy, but the difference
if that a node can have several parents, which avoid duplication as the same formula can be shared.
The right part of figure~\ref{fig:TreeFormula} correspond to DAG for formula $F$.
When we will generate the code, we will set a variable on the shared node, for example $V_{577} = x+y$ , and then reuse
this variable when we need it instead of redoing computation.


The difference in efficiency between a tree and a DAG can be very significant 
in complicated multiple formula  (maybe up to $10$ in colinearity). By multiple formula we mean formula that return several values;
in differenciation we will compute $F$ but also all its derivatives and it often happen that
the same sub-formula appears in several partial derivatives.  In our example we
have $\frac{\partial F}{\partial x} = 2*(x+y) -1 + \sin(x+y)$ and $\frac{\partial F}{\partial y} = 2*(x+y) + \sin(x+y)$ ,
and figure~\ref{fig:DagMultiformula} represents the DAG for the multiple formula
$\{F,\frac{\partial F}{\partial x},\frac{\partial F}{\partial y}\}$.



\begin{figure}
\centering
\includegraphics[width=12cm]{Programmer/ImagesProg/DAG.jpg}
\caption{DAG for a multiple formula  $[F,\frac{\partial F}{\partial x}, \frac{\partial F}{\partial y}]$ with $F(x,y) = (x+y)^2 +x - \cos(x+y)$}
\label{fig:DagMultiformula}
\end{figure}


%-----------------------------------------------------
%-----------------------------------------------------
%-----------------------------------------------------

\section{General organization}

%-----------------------------------------------------

The code is located in  {\tt MMVII/include/SymbDer/}.

\subsection{formulas}

The key-file is {\tt SymbolicDerivatives.h}  where is defined   the key-classes   {\tt cFormula and cImplemF},
cFormula being simply pointer on cImplemF,  ({\tt cImplemF} is the real class containing the data, while
{\tt cFormula} are the classes manipulated externally as we want a remanent objet  semantic).
For now we will speak of formula to name undifferrently these two linked classes.


A formula is the {\tt C++} equivalent of the tree/DAG, basically formula work this way :

\begin{itemize}
       \item  a formula contains a set of sub-formula (currently between $0$ en $2$);

       \item  for each kind of formula there is a derived class (i.e one derived class for each operator)

       \item  mathematical operator ($+,*\dots ,\cos,\tan,\dots$) have been overloaded on formula, 
               for example consider a binary operator  $\otimes$ , each time we meet $F_1\otimes F_2$ 
		the rule is to create a new formula of the adequate derived class $c\otimes$ with  
		$\{F_1,F_2\}$ as set of sub-formula,
		there is two (very frequent) exception to this creation of new formula:

               \begin{itemize}
                    \item if the formulla has already be encontered inside a computation, this formula is returned
                          and nothing is created (that why it's a DAG);
		  \item if some simplication rule specific to the operator exist (like $F*1=F$) they are applied;
               \end{itemize}
	       
        \item  a formula contains several virtual methods indicating how to 
               \begin{itemize}
		       \item  compute its value (used in interpreted mode, not usefull for code generation)
		       \item  computes its derivative by return a new formula;
		       \item  computes the generated code;
               \end{itemize}

		the above listing extract these $3$ methods for the {\tt cMulF} correspondingto the multiplication
		of functions;
\end{itemize}


\begin{lstlisting}[language=c++]
// method for computing values
void ComputeBuf(int aK0,int aK1) override
{
    for (int aK=aK0 ; aK<aK1 ; aK++)
        mDataBuf[aK] =  mDataF1[aK] * mDataF2[aK];
}
// === Extract of code in class cMulF (multiplication of formulas )

// method for computing the derivative
cFormula<TypeElem> Derivate(int aK) const override
{
   return  mF2*mF1->Derivate(aK) + mF1*mF2->Derivate(aK);
}

// method for generating the code
std::string GenCodeDef() const override 
{
    return "(" + mF1->GenCodeRef() + " " + this->NameOperator() +  " " + mF2->GenCodeRef() + ")";
}

\end{lstlisting}

%-----------------------------------------------------

\subsection{Coordinator}

Generating the DAG, requires some coordination between the formulas created.
As we want to create a DAG with  minimal number of nodes, we must keep track
of all already existing formulas created for a specific task.
This is where the class {\tt cCoordinatorF} is usefull,  it creates a dictionnary ({\tt std::map})
that stores the association "name/Formula created".  So the coordinator "knows" all the 
formula that belongs to it while each formula "knows" its coordinator. Also the coordinator
associates to each new formula a unique number.

Let illustrate on an example how it works :

\begin{itemize}
	\item  suppose the code contains  $F_1\otimes F_2$  (or  $\otimes(F_1,F_2)$ );
	\item  when this code is executed, $F_1$ and $F_2$ have been created and are given as
		parameter to $\otimes $ (overloaded on formulas), the coordinator has
		attributed a unique number $N_1$ and $N_2$ to these formulas;
	\item  in the function {\tt cGenOperatorBinaire::Generate}, an identifiant is computed $Id="\otimes \; N_1 \; N_2" $,
	\item  if there is a value associated to $Id$, this value is returned, else a new formula is created 
	       the association "Name/formula" is memorized in its coordinator dictionnary, and this new value is returned;
\end{itemize}

The coordinator  is the access point for generating the code. The user has first to select the curent formulas :
it can be the formulas themselves or the formulas plus all their derivatives.
Note that we have the possibility to give several formulas (in a {\tt std::vector}) as curent formulas,
this is because, due to DAG reduction,  the  DAG of $N$ formulas will have less nodes
than the $N$ independant DAG and consequently the code will be more efficient.

Also why would a user generate code without derivatice while
obiously these classes are interesting because we want to compute derivatives? 
In fact  it can be interesting for a given formula to use it both for  computing funcion+derivative and
computing  only the function : for example in weighted least squares, we can whish to first compute the function
only and, if certain conditions are satisfied, compute the derivative.

Once the current formula has been settled, the code generation follows this pipeline :

\begin{itemize}
     \item make a topologicall sort of the formulas that can be reached from the current formulas
         (in topoligical sort a formula appears always after its sons);

     \item generate the code by parsing the sorted formula, the code is pretty monotom with 
           a succession of lines like $V_i = V_j \otimes V_k; $, due to topologicall sort, we are
           sure when this line is generated that $V_j$ and $V_k$ have been initialized;
\end{itemize}


%-----------------------------------------------------

\subsection{class calculator}

Once we have generated the code, we generally want to use it ;-)  that is where 
the class {\tt cCalculator} step in.  The class {\tt cCalculator} is an interface
class that gives access to calculation of generated code.

Basically,  a {\tt cCalculator} is an abstraction of real function,
if we have a formula with $n$ unkonwns that compute $m$ value and their partial derivatives,
the function will be $\RR^n \rightarrow \RR^{(n+1)*m}$. 
The function {\tt DoOneEval} take as parameter two vectors, one for variable and the other
for observations, and return the vector value (of size $(n+1)*m$).

The {\tt cCoordinatorF} inherits of class {\tt cCalculator}, it can  compute the
value of formulas, by the way it's an interpreted mode and is not very efficient.
In the code generated, the class generated also inherit from {\tt cCalculator},  they
use the generated code to compute efficiently the value in {\tt DoOneEval} ,
and they can be manipulated via the unique interface of a {\tt cCalculator}
(in fact in {\tt MMVII}, the "devlopper/user" of this service do not need to load
directly the file generated, he can just get an object from its name).

%-----------------------------------------------------
%-----------------------------------------------------
%-----------------------------------------------------

\section{Classes for  formula}

There exists $3$ main class that derive from the class {\tt cImplemF}, these $3$ classes
correspond to the cases where the formula has $0,1$ or $2$ sub-formula.

      %-----------------------------------------------------
\subsection{Atomic formula}

The atomic formula are the "final" (or "primitive" formula) .  The base
class for them is {\tt cAtomicF}, there is now $3$ atomic derived class, and
probably it will stay like that. The $3$ atomic formula are : 
class for unknowns $x_1,x_2\dots$ ({\tt cUnknownF}), class for constants ({\tt cConstantF}),
and class for observations ({\tt cObservationF}).

It should be pretty obvious what {\tt cUnknownF}  and {\tt cConstantF} represent.
Class {\tt cObservationF} is a less intuitive : consider the equation of 
$2d$ network  where we want to enforce the euclidean distance between $P_1=(x_1,y_1)$ and $P_2=(x_2,y_2)$
to be equal to $D_0$:

\begin{equation}
   d(P_1,P_2) = (x_1-x_2)^2 + (y_1-y_2)^2 = D_0
\end{equation}

In this equation $x_1,y_1,x_2,y_2$ are unknown and, in adjustment, we will need to compute 
the partial derivative relative to them. In the partial derivative $\frac{\partial d}{\partial x_1} = 2*(x_1-x_2)$
the $2$ is a contant we know it's value a time of code generation.

The status of $D_0$ is different, we don't know its value at time of code generation, but a run time
it has a value that will not change because it's not an unknown and for 
any variable $x_k$  we have $\frac{\partial D_0}{\partial x_k} =0$.
$D_0$ can be interpreted as constant whose value will only be fixed at runtime.



      %-----------------------------------------------------
\subsection{Unary formula}

      %  -  -  -  -  -  -  -  -  -  -  -  -  -  -  -  -  -  -  -
\subsubsection{Generatlities}

Unary operator are defined in file {\tt SymbDer/SymbDer\_UnaryOp.h}. Class for implementing unary operator
will derive from class  {\tt cUnaryF}, which itself derive from {\tt cImplemF}.


For definining a new unary operator we will take the example of the square operator   $square : x \rightarrow x^2$.
We must indicate several information, most are located in class {\tt cSquareF}:

\begin{itemize}
	\item   the static method {\tt Operation} simply describes the function itself $x \rightarrow x^2$, it will be  used
		in the reduction rule $ \alpha (C(a)) \rightarrow C(\alpha (a)) $ defined in~\ref{Reduc:Rule};
	
	\item  the static method {\tt StaticNameOperator} return the name of the  operator as a string {\tt "square"}, 
		it is used in code generation and to associate a unique identier to the formula (create an
		identifier like {\tt "square F70"}, where  {\tt "F70"} is the formula with number $70$);

	\item  the method {\tt ComputeBuf} does conceptually same thing as {\tt Operation} , but is used when
		applied to a buffer of data containing several values, (it is used interpretade mode, 
		not for code generation);

	\item  the method {\tt  Derivate}  return \emph{as a formula} the derivate $\frac{\partial F^2}{\partial x_k}$
		of the formula $F$ relatively to variable $x_k$;

        \item note also method {\tt NameOperator} ,it's  a virtual interface to method {\tt StaticNameOperator}
		(the code of this method which will be always the same).
\end{itemize}

Finaly we must also define a operator {\tt square} that work on formula and such that {\tt square(Formula F)} 
return a pointer on a object of type {\tt cSquareF}.  It's a bit more complicated than just creating
the pointer we must :

\begin{itemize}
    \item  check in the coordinator if there is already such formula, and if yes return it;
    \item  else check if  {\tt F} if a constant and if yes apply reduction;
    \item  else create the formula, and indicate to coordinator that it exist.
\end{itemize}

As all this action are systematcic, they can be done in  a generic way, this is done by the class 
template {\tt cGenOperatorUnaire}. So the definition of {\tt square(Formula F)} using this class
is pretty simple :

\begin{lstlisting}[language=c++]
template <class TypeElem> inline cFormula<TypeElem>  square(const cFormula<TypeElem> & aF)
{
    return cGenOperatorUnaire<cSquareF<TypeElem> >::Generate(aF);
}
\end{lstlisting}

      %  -  -  -  -  -  -  -  -  -  -  -  -  -  -  -  -  -  -  -
\subsubsection{Existing operators}

Also it is relatively easy to define new operators usign the scheme of {\tt  square}, a certain number have been
pre-defined :

\begin{itemize}
    \item integer power {\tt square}, {\tt cube},{\tt pow4}, {\tt pow5},{\tt pow6},{\tt pow7},
     \item unitary minus {\tt -};
     \item trigonometric {\tt cos} and  {\tt sin};
     \item {\tt exp} and  {\tt log};
     \item {\tt sqrt}.
\end{itemize}

There is also {\tt pow8} and {\tt pow9}, that for now use the binary {\tt pow};
and finally {\tt powI(Formula,int)} that return one of the following.

      %  -  -  -  -  -  -  -  -  -  -  -  -  -  -  -  -  -  -  -
\subsubsection{New operator in one line with a Macro}

If one takes a look at definition of class of operator, it's obvious that it's quite repetitive;
it should be possible to define new unary operator with minimal additionnal of code.

This is possible using the macro {\tt MACRO\_SD\_DEFINE\_STD\_UNARY\_FUNC\_OP\_DERIVABLE}
defined in {\tt SymbDer\_MACRO.h}.
Do have a differentiable operator on formula, we just need to indicate the name
of the function computing the values, the name of the function computing the derivative,
and the namespace where they are defined.

The easiest is to take the example of sinus-cardinal defined in  {\tt SymbDer\_MACRO.h}.
To define a new operator we need to have a functions on real values that compute $\sinc$ and a function
that compute $\frac{\partial \sinc}{\partial x}$  this is {\tt DerSinC}.  In {\tt MMVII} these function have been
defined in file  {\tt src/UtiMaths/uti\_fonc\_analytique.cpp}. 


\begin{lstlisting}[language=c++]
MACRO_SD_DEFINE_STD_UNARY_FUNC_OP_DERIVABLE(MMVII,sinC,DerSinC)
\end{lstlisting}

This macro does several things :

\begin{itemize}
	\item  declare the existence of operators {\tt sinC} and {\tt DerSinC} that works on formula;

	\item  define the classes  {\tt csinC} and {\tt cDerDinC} that inherits of {\tt cUnaryF};
		note that {\tt csinC}  will call the operator {\tt DerSinC} in method {\tt Derivate},
		while for {\tt cDerDinC}  we make the choice of having no meaningfull derivate
		(its {\tt Derivate} calls the method {\tt UndefinedOperUn};

	\item  declare the operators {\tt sinC} and {\tt DerSinC} , this definition must occurs  after the class
		definition.
\end{itemize}

The macro is pretty simple to use, the price it that it make choices that will not be always pertinent.
For example the choice that {\tt DerSinC} is not itself derivable may not be acceptable.
In this case it is possible to have a finer tuning usign piece by piece the $3$ macro

\begin{lstlisting}[language=c++]
MACRO_SD_DECLARE_STD_UNARY_FUNC_OP
MACRO_SD_DEFINE_STD_cUnaryF
MACRO_SD_DEFINE_STD_UNARY_FUNC_OP
\end{lstlisting}

An example is done in file {\tt src/SymbDerGen/Formulas\_CamStenope.h} with the function
{\tt cosh} and {\tt sinh}  (hyperbolic trigonometric functions).


      %-----------------------------------------------------
\subsection{Binary formula}

The implemantation of binary formula is pretty much the same than the implementation for unary formula,
the main difference being that binary formula contains 2 sub-formula $F_1$ and $F_2$ \dots

The code for is located in {\tt SymbDer\_BinaryOp.h}.  The base class of all binary formula is
{\tt cBinaryF}.  Binary formula have some method that are used exclusively in the reduction process 
as {\tt IsAssociatif},  {\tt IsDistribExt} and {\tt IsDistribExt}  (and probably usefull only
for the $4$ operator $+-*/$, so default value should work for all new binary operator).

There is $5$ binary operators defined in  {\tt SymbDer\_BinaryOp.h} : $+*/-$ and $pow$.

Like with unary operators, {\tt MACRO\_SD\_DEFINE\_STD\_BINARY\_FUNC\_OP\_DERIVABLE}
make possible to create a new binary operator with a single macro.
This macro take one more parameters than unary one, becasuse  we must indicate the two
derivative by each of the parameter to apply the rule :

\begin{equation}
        \frac{\partial (F_1 \otimes F_2)} {\partial x_k} 
     =   \frac{\partial F_1 } {\partial x_k} * \frac{\partial (F_1 \otimes F_2)} {\partial F_1} 
       + \frac{\partial F_2 } {\partial x_k} * \frac{\partial (F_1 \otimes F_2)} {\partial F_2} 
\end{equation}

Examples can be found in {\tt Formulas\_CamStenope.h}.


      %-----------------------------------------------------

\subsection{Reduction rules}

\label{Reduc:Rule}

To have a faster generated code, several reduction rules are used when 
constructing the formulas. Also this may seems anecdotic,
it has a non negligeable  impact on efficiency, as we can see with the equation~\ref{Form:Reduc}
that gives a basic example using, in a first step, blindly the 
rule of automatic derivation and then applying reduction rule .
We have two gains : first $x$ is obviously faster to compute than $0 * (y+2) + x * (1 + 0)$,
and second it increase the likeliness to have code sharing . Note the time to do
these reductions themselve is taken in a "compilation/generation" step and not at execution
(and by the way this compilation time it neglectible).

\begin{equation}
	\frac{\partial(x*(y+2))}{\partial y}
	=  0 * (y+2) + x * (1 + 0)
	\rightarrow x
	\label{Form:Reduc}.
\end{equation}

We give a, non exhaustive, list of these rules, :

\begin{itemize}
     \item  $0*F \rightarrow 0$ ,   $1*F  \rightarrow F$ , $-1*F \rightarrow -F$ (and idem $F*0\dots$);

     \item  $0/F \rightarrow 0$ ;

     \item  $0+F \rightarrow F$ ,   (and idem $F+0\dots$);

     \item  $-(-F) \rightarrow F$ ,  $F_1-(-F_2) \rightarrow F_1+F_2$;

     \item  $F_1*F_2 + F_1*F_3  \rightarrow F_1*(F_2+F_3)$ , $F_1*F_2 - F_1*F_3  \rightarrow F_1*(F_2-F_3)$ , 
     \item  $F_2/F_1 + F_3/F_1  \rightarrow (F_2+F_3)/F1$ ,  $F_2/F_1 - F_3/F_1  \rightarrow (F_2-F_3)/F1$ , 

     \item if  $F_1 > F_2$ then  $F_1 + F_2  \rightarrow F_2 + F_1$ , here the comparison $F_1 > F_2$ 
           is made on numbering, this rules is here favorize merging in DAG creation , any comparison as long
           as it anti-symetric would hold; we just want to avoid that in the same formula $F_1+F_2$ and $F_2+F_1$
	   cannot be merged;
     \item if  $F_1 > F_2$ then  $F_1 * F_2  \rightarrow F_2 * F_1$ 

     \item  $C(a) \otimes C(b) \rightarrow C(a \otimes b) $  where  $C(x)$ design the formula for constant $x$
	     and $\otimes$ is any bynary operator;
     \item  $ \alpha (C(a)) \rightarrow C(\alpha (a)) $  where  $\alpha$ design any unary operator;

\end{itemize}


%-----------------------------------------------------
%-----------------------------------------------------
%-----------------------------------------------------

\section{Code generation}







\chapter{The mapping object}

%---------------------------------------------
%---------------------------------------------
%---------------------------------------------

\section{Introduction}

%---------------------------------------------
%\subsection{Target of mapping objects}

The mapping  are targeted to offer service for object that represent "smooth" mapping
from $\RR^n  \rightarrow  \RR^p$. As an exemple of class naturally derived from mappings used
in photogrammetry we have :

\begin{itemize}
	\item projection $\pi : (x,y,z) \rightarrow (i,j)$ function of an image sensor, as mapping $\RR^3 \rightarrow \RR^2$;

	\item extended projection $\pi_d :  (x,y,z) \leftrightarrow (i,j,d)$ , where $d$ is  the depth,
		as \emph{bijective} mapping of $\RR^3$  ( $\RR^3 \rightarrow  \RR^3$);

	\item distorsion of central perpective camera as  \emph{bijective} mapping of $\RR^2$;

	\item any  transformation  between two geodetic coordinate systems as \emph{bijective}  of  $\RR^3$.

\end{itemize}

The minimal service that a mapping must offer is to define the method $F$ that computes its values.
The kind of services that offers the  mappingi package is :

\begin{itemize}
     \item offer a default method computing the derivative $\frac{\partial F}{\partial x_i}$  using a basic finite differrence ,
           the class can override this default method if has something better to offer;

   \item for $\RR^n \rightarrow \RR^n$ compute the inverse $F^{-1}(v)$ of a given  value using  an iterative method;
           the class can override this default method if has something better to offer;

   \item compute the approximate inverse mapping  of given mapping using some basis of function and a  least square approach;

   \item offer an interface to use generated code of symbolic derivative as a mapping.
\end{itemize}

%---------------------------------------------
%---------------------------------------------
%---------------------------------------------

\section{General organization}

\subsection{Localization}

The declaration of class for mapping are localized in file {\tt include/MMVII\_Mappings.h}.

The definition of these class are located in folder {\tt src/Mappings/}.
As the mapping class are template, there is an explicit instantiation  for
all expected use.



%---------------------------------------------
\subsection{class {\tt cDataMapping}}

         %  -  -  -  -  -  -  -  -  -  -  -  -  -  -
\subsubsection{Templatization}
The base  class of all mappings is {\tt cDataMapping}, its a template class defined by $3$ 
parameters :

\begin{itemize}
    \item {\tt class Type} which is the floatting number type on which all the computation will be made,
          it can be {\tt tREAL4, tREAL8} or {\tt tREAL16} ;  practically it is for now obly used
          with {\tt tREAL8}; by the way some precaution where made to assure that
          the class be intantiated with any complete numeric type in case higher precision woul be required;


    \item {\tt const int DimIn} the dimension of input space;

    \item {\tt const int DimOut} the dimension of output space.
\end{itemize}

         %  -  -  -  -  -  -  -  -  -  -  -  -  -  -
\subsubsection{Values}

The fundamental method that a  mappings must define is  {\tt Value(s)} and it computes the values of 
the function.  Note that there exist two methods :

\begin{itemize}
     \item {\tt Value} that make the computation of single value ;

     \item {\tt Values} that make the computation of vector of values, it can  be used
           if the class has some parallelism option to accelerate the computation.
\end{itemize}

Note that these two virtual methods  have a default implementation : {\tt Value}
is implemented calling {\tt Values},  while {\tt Values} is implemented calling
{\tt Value}.  So obviously, an infinite recursion will occur if none is defined
(BTW it is dynamically detected in debug mode).  The interest being obviously that
in the derived class, it's possible to overload only one to benefit of both.

For {\tt Value} there are two options : 

\begin{itemize}
     \item  one option where the user gives the vector for storing the result;

     \item  one option where the class furnish its own buffer by reference,
            btw the same vector is always returned, so if the user memorize
            the adress, at next call it will be overwritten; so if the vector
            is not used immediately, a copy must be made.
\end{itemize}



         %  -  -  -  -  -  -  -  -  -  -  -  -  -  -
\subsubsection{Jacobian}

The jacobian is computed by returning pair point/Matrix  where point is
the value  and matrix is the  jacobian. This is because generaly when the 
user needs the jacobian he also needs the value, and also when you 
compute the jacobian, you have also computed the value .

Be aware that even if the user make a copy a vector containing results,
due to {\tt MMVII} implementation of matrix (using shared pointer on data),
at next call the jacobian will overwrite the previous call.  In this rare
case, user should call the {\tt Dup} method.

Like {\tt Value} the  class propose default definition of {\tt Jacobian} that user can override.











\chapter{Sensor classes}


%---------------------------------------------
%---------------------------------------------
%---------------------------------------------

\section{Mother class {\tt cSensorImage}}

%---------------------------------------------

\subsection{bbb}

          %  - - - - - - - - - - - - - - - - - - - 
\subsubsection{CCC}

%---------------------------------------------
%---------------------------------------------
%---------------------------------------------




\COM
{
\chapter{The Fits method}

For now "bloc-note",

% Conclusion, on peut sans doute limiter le nombre de point avec ScaleStab
% pour filtrage a priori => genre les 500 les plus stable
}





%---------------------------------------------
%--------------- PART II ----------------------
%---------------------------------------------

\part{Reference documentation}

\chapter{Mesh related commands}


CNR  CERN 


%-----------------------------------------------------------------------
%-----------------------------------------------------------------------
%-----------------------------------------------------------------------

\section{Process of 3D meshes}

 {\tt MMVII} is not a point-cloud/mesh processing tool. However, during some devlopment, some
functionnalities appeared to be necessary and not easily accesible on open source.
As they may be usable in other context I describe them here.

%-----------------------------------------------------------------------
\subsection{MeshCheck}

\begin{verbatim}
 == Mandatory unnamed args : ==
  * string [FDP,Cloud] :: Name of input cloud/mesh
 == Optional named args : ==
  * [Name=Bin] bool :: Generate out in binary format ,[Default=false]
  * [Name=Out] string :: Name of output file if correction are done
  * [Name=Do2DC] bool :: check also as a 2D-triangulation (orientation) ,[Default=false]
  * [Name=Correct] bool :: Do correction, Defaut: Do It Out specified
\end{verbatim}

{\tt MicMac-V1} can generate mesh from a 3d point cloud with normal. For this it
uses the open source tool devlopped by M Kazhdan. Also this tool is robust,
it has sometime topologicall problem that can block further processing.
The command {\tt MeshCheck} has been developped to  detect and potentially
correct these problems.

The first problem detected   is the fact that the same point that is duplicate in
a single triangle  like $ABC$  with $P_A=P_B$. This point are detected and
if the option {\tt Correct} is activated, a graph-quotient algorithm is applied 
on point that are detected as multiple.

The second check is $3d$ surface orientatbility . A message indicating the detected
pair of adjacent triangle badly oriented is printed.  No correction is
done, because untill now no such problem was encounterred.

If the option {\tt Do2DC} is activated, the triangulation is considered
as a $2$ triangle (setting $z$ to $0$) and  the $2d$ orientation correctness is
tested (have all the triangle the same orientation).


%-----------------------------------------------------------------------
\subsection{MeshCloudClip}

\begin{verbatim}
 == Mandatory unnamed args : ==
  * string [FDP,Cloud] :: Name of input cloud/mesh
  * string [3DReg] :: Name of 3D masq

 == Optional named args : ==
  * [Name=Out] string :: Name of output file
  * [Name=Bin] bool :: Generate out in binary format ,[Default=false]
\end{verbatim}


The mesh generated by Poisson method generate a smooth extension of the surface
that can be problematic, for example in the surface devlopment process. This
commande allow to clip the mesh using a $3d$ region.  For now the region
must come in the {\tt MicMac-V1} format as seized by {\tt mm3d SaisieMasqQT}
as for now there is no well established format for $3d$ volumes.
Figure~\ref{fig:MeshClip} illustrate the result of {\tt MeshCloudClip}

\begin{figure}
\centering
\includegraphics[width=6cm]{CommandReferences/ImagesComRef/MeshWithSkirt.jpg}
\includegraphics[width=6cm]{CommandReferences/ImagesComRef/MeshCliped.jpg}
\caption{Letft mesh with undesirable extension, right mesh after clipping}
\label{fig:MeshClip}
\end{figure}



%-----------------------------------------------------------------------
%-----------------------------------------------------------------------
%-----------------------------------------------------------------------

\section{Mesh development}

This section describe the tools that were devloped for surface devlopment.

%-----------------------------------------------------------------------
\subsection{MeshDevGen}
This "tiny" tool was made to check the correctness of the surface devlopment algorithm. 
It generate a synthetic mesh, that is completely devlopable and for with we
know the 2 devlopment. This surface is a $3$ extrusion of a logarithmic spiral.

Figure~\ref{fig:SpirAndDev} represent the 3D surface and its developped 2d surface.

\begin{figure}
\centering
\includegraphics[width=6cm]{CommandReferences/ImagesComRef/Cyl3D01.jpg}
\includegraphics[width=6cm]{CommandReferences/ImagesComRef/CylDev00.jpg}
\caption{A synthetic 3D mesh, and its generated devloped surface}
\label{fig:SpirAndDev}
\end{figure}

%-----------------------------------------------------------------------
\subsection{MeshDev}


This tool is used to make the $2d$ development of a $3d$ surface. As the real surface
is generally not strictly devlopable, this is done by minimzing the energy of deformation.
More precisely :

\begin{itemize}
      \item let  $P_i$ be the $3d$ points of the triangulation, $P_i \in \RR^3$;
      \item let  $\phi(P_i)$ be unknown $2d$  position of the devlopment $\phi(P_i) \in \RR^2$;
      \item let  $T_k = \{P_{k_1},P_{k_2},P_{k_3}\}$ be the triangles;
\end{itemize}

For each triangle  $T_k$ \emph{independantly} we can easily compute a perfect development
$q^k_1,q^k_2,q^k_3$, such that $d(q^k_m,q^k_n)=d(P_{k_m},P_{k_n}) \; m,n \in (1,2,3)$. If the development
were perfect, there would exist for each triangle a rotation $R^k$ such that :

\begin{equation}
	\phi(P_{k_m}) =  R^k q^k_m  \;\; m \in (1,2,3) \label{Eq:MeshDev1T}
\end{equation}


The then try to compute the $\phi(P_i)$ and $R_k$ that minisze globaly the equation)~\ref{Eq:MeshDev1T} :

\begin{equation}
    \sum_k \sum_{m=1}^3  ||\phi(P_{k_m}) - R^k q^k_m|| ^2
\end{equation}


\COM{
MeshDev => Generate a planar devlopment minimizing deformations
MeshImageDevlp =>  Compute devlopped images from 3d-mesh, 2d-dev-mesh and ori
MeshProjImage => (internal) Project a mes on an image to prepare devlopment

}






%---------------------------------------------
%--------------- PART I ----------------------
%---------------------------------------------

\part{Annexes}

\appendix

\chapter{Bibliography}


\begin{thebibliography}{AAA}
   \bibitem[Tomasi Kanabe 98]{TomKan}   S. Roy, I.J. Cox , 1998, "Shape and Motion from Image 
            Streams under Orthography: a Factorization Method", International Journal of Computer Vision, 
            9:2, 137-154 (1992)


    \bibitem[CERES]{CERES} \emph{https://github.com/ceres-solver/ceres-solver/blob/master/include/ceres/jet.h}

   \bibitem[Cox-Roy 98]{CoxRoy}   S. Roy, I.J. Cox , 1998, "A Maximum-Flow
            formulation of the N-camera Stereo Correspondence
      Problem", \emph{Proc. IEEE Internation Conference on
      Computer Vision}, pp 492--499, Bombay.

   \bibitem[Fraser C. 97]{Fraser}  C. Fraser, 1997, "Digital camera self-calibration",
   \emph{ISPRS Journal of Photogrammetry and Remote Sensing}, vol. 52, issue 4, pp. 149-159,
   \bibitem[Penard L. 2006 ]{Penard}   L. Pénard, N. Paparoditis, M. Pierrot-Deseilligny.
           "Reconstruction 3D automatique de façades de bâtiments en multi-vues.",
            RFIA (Reconnaissance des Formes et Intelligence Artificielle),
            Tours, France, January 2006.
\end{thebibliography}


\printindex



\end{document}




