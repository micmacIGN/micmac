\chapter{Mesure de ressemblance et corr\'elation}

\label{CHAP:COREL}

    % - - - - - - - - - - - - - - - - - - - - - - - - - - - - - - - -

\section{G\'en\'eralit\'e}

MicMac utilise essentiellement le coefficient de corr\'elation
normalis\'e centr\'e comme crit\`ere d'attache aux donn\'ees.

Le cadre g\'en\'eral est la comparaison de vecteur de r\'eels
pond\'er\'ees \footnote{on pourrait g\'en\'eraliser
\`a des fonction r\'eelles sur un espace probabilis\'e}, typiquement
ce vecteur est constitu\'e de l'ensemble des valeur de la vignette
de corr\'elation et le poids vaut souvent uniform\'ement $1$.
Soit  $\Echant=\RR^N$ l'espace des vecteurs.
Pour  $\lambda \in \RR$, lorsqu'il n'y a pas d'ambiguit\'e,
  on notera  $\lambda$ le vecteur
constant de \Echant  tel que $\forall k, u_k=\lambda $.
On se donne une fonction de pond\'eration $p^{ds}_k ,k\in[1 \, n]$.
Pour chaque  vecteur $U$ de valeurs $U_k, k \in [1 \, n]$ on pose :

\begin{equation}
   E(U) =  \frac{\sum^n_{k=1} p^{ds}_k U_k}{\sum^n_{k=1}  p^{ds}_k}
\end{equation}

Etant donn\'e deux  vecteur $u_k$ et $v_k$, une d\'efinition classique du
coefficient normalis\'e centr\'e est :

\begin{equation}
   Cor(u,v) =  \frac{E(uv)-E(u)E(v)}{\sqrt{(E(u^2)-E(u)^2)*(E(v^2)-E(v)^2)}}
\end{equation}

Cette d\'efinition est proche de celle utilis\'ee pour le calcul
(notamment pour les algorithmes de calculs rapides), mais pas forc\'ement intuitive
\`a interpr\'eter. Pour b\^atir une interpr\'etation
g\'eom\'etrique, on remarque d'abord que $E(u,v)$ est un produit scalaire et
on munit $\Echant$  de la norme associ\'ee :

\begin{equation}
   \| X \|^2 =  E(X^2)
\end{equation}

On note $\PZero$ l'hyperplan des vecteur de moyenne nulle :

\begin{equation}
   \PZero = \{u\in \Echant / E(u)=0\}
\end{equation}

On note $\SUn$ la sph\`ere unit\'e de \Echant :

\begin{equation}
   \SUn = \{u\in \Echant / \|u\|=1 \}
\end{equation}

On note :

\begin{equation}
   \bar{u} = u -E(u)
\end{equation}

Il est imm\'ediat que $\bar{u} \in \PZero$, et $\bar{u}$ peut s'interpr\'eter
comme  la projection orthogonale de $u$ sur $\PZero$. On note :

\begin{equation}
   \tilde{u} = \frac{\bar{u}}{\| \bar{u} \|}
\end{equation}


Il est imm\'ediat que $\|\tilde{u}\|=1$ et $\tilde{u}$ peut s'interpr\'eter
comme la projection de $\bar{u}$ sur $\SUn$. Une propri\'et\'e interessante
de  l'application $u \rightarrow \tilde{u} $ est d'\^etre invariante  par
translation et homot\'etie :

\begin{equation}
  \forall  (\alpha,\beta) \in \RR^2 \; \tilde{}(\alpha+\beta*u) = \tilde{u}
\end{equation}

On v\'erifie  que le coefficient de correlation peut \^etre d\'efini comme :

\begin{equation}
   Corr(u,v) =  E(\tilde{u}\tilde{v})
   \label{Cor:Prod:Scal}
\end{equation}


Le coefficient de corr\'elation peut dont \^etre interpr\'et\'e
au choix comme :

\begin{itemize}
  \item  le produit scalaire entre $\tilde{u}$ et $\tilde{v}$;

  \item  le \emph{cosinus} de l'angle entre $\bar{u}$ et $\bar{v}$;

  \item  le \emph{cosinus} de l'angle entre les projection orthogonale
         de $u$ et $v$ sur \PZero.
\end{itemize}


Compte tenu de  $\|\tilde{u}\|=\|\tilde{v}\|=1 $,
l'\'equation~\ref{Cor:Prod:Scal}  peut se r\'e\'ecrire :


\begin{equation}
   Corr(u,v) =  1-\frac{\|\tilde{u}-\tilde{v}\|^2}{2}
\label{Corr:As:Norme}
\end{equation}

Ce qui fournit une interpr\'etation int\'eressante du coefficient
de corr\'elation: c'est, \`a un r\'e\'etalonement pr\`es, une mesure
de distance entre des \'echantillons normalis\'es de mani\`ere
 invariante par homot\'etie translation des radiom\'etries.

    % - - - - - - - - - - - - - - - - - - - - - - - - - - - - - - - -

\section{Utilisation pour la ressemblance de  deux vignettes}

\label{Vign:Correl}

Soit $I_k$ des images, $p=(i,j)$ un point de l'espace terrain
discr\'etis\'e, $p_{x_0}$ une parallaxe fix\'ee.
On consid\`ere $U^N_k(l,m)$ le vecteur  de $\RR^{(2N+1)^2}$ correspondant  \`a  la
"vignette" de taille $N$ centr\'ee en $p$ :


\begin{equation}
   U^N_k = (I_k( \breve\pi_k(i+l,j+m,p_{x_0}))) ,  (l,m) \in [-N \; N] ^2
\end{equation}

En g\'en\'eral on  utilise une pond\'eration uniforme $\forall k \;p^{ds}_k=1$ ,
le produit scalaire $U.V$ est alors d\'efini par :

\begin{equation}
   E(UV) =  \frac{\sum^n_{k=1} U_k V_k}n
\end{equation}


On d\'efinit le coefficient de corr\'elation de deux images $I_1$ et $I_2$,
au point $p$, avec la parallaxe $p_{x_0}$, sur une fen\^etre de taille $N$ par :


\begin{equation}
   Corr^N_{p_{x_0}}[I_1,I_2](p) =   Corr(U^N_1(p),U^N_2(p))
\end{equation}

    % - - - - - - - - - - - - - - - - - - - - - - - - - - - - - - - -

\section{Fen\^etre de "taille 1"}

\label{Vign:Taille:Un}

La formule~\ref{Corr:As:Norme} montre que  le coefficient de corr\'elation
peut \^etre exprim\'e \`a partir de  la distance euclidienne sur les vecteurs
normalis\'es en homoth\'etie-translation sur les radiom\'etries.

Dans le coefficient de corr\'elation standard, la vignette sur
laquelle est faite la normalisation est la m\^eme que celle sur
laquelle est calcul\'ee la norme. Cette contrainte n'a rien
d'automatique, et il peut \^etre \emph{a priori} coh\'erent de
faire la mesure d'\'ecart sur des fen\^etres plus petites que
sur celles sur laquelle est effectu\'ee la normalisation.

Soit $\Ind_M$ le vecteur de  $\RR^{(2N+1)^2}$ qui vaut $1$ ssi $(l\leq M , m \leq M)$
et $0$ sinon; on note $\| \|_M$ la norme de $\RR^{(2N+1)^2}$ d\'efinie
par :

\begin{equation}
 \|\tilde{U}^N(p)\|_M^2 = E(\tilde{U}^N(p)* \Ind_M ) * \frac{1}{E(\Ind_M)}
\end{equation}

A un terme de normalisation pr\`es, la norme $\| \|_M$ est simplement
l'\'ecart mesur\'e sur une vignette de taille $M$. On d\'efinit alors
le coefficient de corr\'elation sur une fen\^etre de taille $M/N$ par :


\begin{equation}
   Corr^{M/N}_{p_{x_0}}[I_1,I_2](p) =  1-
        \frac{ \| \tilde{U}^N_1(p)-\tilde{U}^N_2(p)\|_M^2}{2}
\end{equation}

On peut \'eventuellement utiliser $ Corr^{M/N}$ avec $M=1$,
d'o\`u le nom de la section. Le coefficient  $ Corr^{M/N}$
n'est clairement plus compris entre $-1$ et $1$.

% - - - - - - - - - - - - - - - - - - - - - - - - - - - - - - - -
% - - - - - - - - - - - - - - - - - - - - - - - - - - - - - - - -
% - - - - - - - - - - - - - - - - - - - - - - - - - - - - - - - -


    % - - - - - - - - - - - - - - - - - - - - - - - - - - - - - - - -

\section{"Fen\^etre exponentielle"}

    % - - - - - - - - - - - - - - - - - - - - - - - - - - - - - - - -

\subsection{Principe des fen\^etres \`a pond\'eration variable}

\label{MecaGen:FEN:EXPO}

L'utilisation de fen\^etre carr\'ee avec une pond\'eration uniforme
est souvent consid\'er\'e comme le choix naturel. Il n'a cependant
rien d'obligatoire et il peut m\^eme souvent \^etre plus judicieux
d'avoir des fen\^etre de taille plus grande et une pond\'eration
qui d\'ecroit en fonction de la distance au "pixel central".

On peut par exemple envisager une pond\'eration en "chapeau chinois",
soit par exemple avec des fen\^etres de taille $N$ :

\begin{equation}
   p^{ds}_{(x,y)} = |N-x||N-y|
\end{equation}

On peut aussi envisager des fen\^etres gaussiennes, de support
infini:

\begin{equation}
   p^{ds}_{(x,y)} = e^{-\frac{x^2+y^2}{\sigma ^2}}
\end{equation}
\begin{equation}
    E(UV) =\frac{\iint  UV e^{-\frac{x^2+y^2}{\sigma ^2}}}{C^{ste}}
\end{equation}

MicMac offre la possibilit\'e de choisir des fen\^etres exponentielles
qui ont l'avantage d'\^etre \`a support infini (pas de troncature
arbitraire) tout en pouvant \^etre  calcul\'ees rapidement :

\begin{equation}
    E(UV) =\frac{\iint  UV a^{-|x|} b^{-|y|}}{C^{ste}}
\end{equation}

    % - - - - - - - - - - - - - - - - - - - - - - - - - - - - - - - -

\subsection{Equivalence des tailles de fen\^etre}

En vrac pour l'instant.

Comment choisir un param\^etre d'att\'enuation exponentiel pour avoir des
fen\^etres, plus ou moins, \'equivalentes \`a des fen\^etres carr\'ees d'une
certaine taille.
L'id\'ee est de d\'efinir la taille "g\'en\'eralis\'ee"
comme l'esp\'erance de $|X|$ (ou l'\'ecart type $\sqrt{E(X^2})$).
On peut faire le calcul sur des s\'eries discr\`etes ou continues.

Avec des fen\^etres carr\'ees :

\begin{equation}
    C_2(N) = \frac{\sum_{k=-N}^{k=N} k^2}{\sum_{k=-N}^{k=N}1} = \frac{N*(N+1)}{3}
\end{equation}
\begin{equation}
    C'_2(N)
      = \frac{\int_{-N-\frac{1}{2}}^{N+\frac{1}{2}} X^2}{\int_{-N-\frac{1}{2}}^{N+\frac{1}{2} 1}}
      = \frac{(n+1)^2} {3}
\end{equation}

\begin{equation}
    E_2(a)
       = \frac{\sum_{k=-\infty}^{k=+\infty} a^{-|k|}k^2}{\sum_{k=-\infty}^{k=+\infty} a^{-|k|}}
       = \frac{2}{log(a)^2}
\end{equation}

Si on it\`ere $K$ le filtre exponentiel, on obtient un filtre \'equivalent dont l'\'ecart
type est $K$ fois plus grand. Donc finalement :



\begin{equation}
   a_k = e^{-\frac{\sqrt{6k}}{N+1}}
\end{equation}


Avec des fen\^etres exp :


% - - - - - - - - - - - - - - - - - - - - - - - - - - - - - - - -
% - - - - - - - - - - - - - - - - - - - - - - - - - - - - - - - -
% - - - - - - - - - - - - - - - - - - - - - - - - - - - - - - - -

\section{Multi-corr\'elation}


\label{MecaGen:Multi:Correl}
Lorsque l'on a plus de $2$ image, on souhaite d\'efinir un coefficient
de corr\'elation multi-images qui soit un prolongement naturel du
coefficient \`a $2$ images.

Soit $N$ le nombre d'images, on note $U_1\dots U_N$ les vecteurs
\`a comparer. Un premier cas o\`u il est facile de d\'efinir
un coefficient  naturel est celui o\`u il existe une image  jouant
un r\^ole privil\'egi\'e, dite image ma\^itresse. Supposons
que ce soit l'image $1$, on pose :

\begin{equation}
   Corr^M(U_1)(U_2,\dots,U_N) =
    \frac{1}{N-1} \sum_{k=2}^N Corr(U_1,U_k)
   \label{Cor:Im:Maitre}
\end{equation}

Un cas encore plus courant est celui o\`u toutes les images jouent
un r\^ole sym\'etrique. Une d\'efinition naturelle du coefficient de
corr\'elation est de faire une moyenne sur tous les couples possible:

\begin{equation}
   Corr^S(U_1,\dots,U_N) =
    \frac{2}{N*(N-1)} \sum_{1\leq i < j \leq N} Corr(U_i,U_j)
   \label{Cor:Im:Sym}
\end{equation}

L'inconv\'enient apparent de la formule~\ref{Cor:Im:Sym} est d'avoir
un co\^ut de calcul en $O(N^2)$. En fait, nous allons voir que
le calcul peut facilement se faire en $O(N)$. On remarque d'abord
que, d'apr\`es la formule~\ref{Corr:As:Norme}, il est \'equivalent
de calculer :

\begin{equation}
    \sum_{1\leq i < j \leq N} \|\tilde{U}_i-\tilde{U}_j\|^2
\end{equation}


En sym\'etrisant et rajoutant les termes nuls  $\|\tilde{U}_i-\tilde{U}_i\|$,
il est toujours \'equivalent de calculer :

\begin{equation}
    S=\sum_{i=1}^N \sum_{j=1}^N  \|\tilde{U}_i-\tilde{U}_j\|^2
\end{equation}

On d\'efinit le centre de gravit\'e $\Omega$ des $\tilde{U}_i$
(calculable en temps lin\'eaire) par :

\begin{equation}
    \Omega = \frac{\sum_{i=1}^N \tilde{U}_i}{N}
\end{equation}

On d\'eveloppe ensuite :

\begin{equation}
    S=\sum_{i=1}^N \sum_{j=1}^N  \|(\tilde{U}_i-\Omega) +(\Omega-\tilde{U}_j)\|^2
\end{equation}

\begin{equation}
    S= \sum_{i=1}^N \sum_{j=1}^N  (\|(\tilde{U}_i-\Omega)\|^2 +\|(\Omega-\tilde{U}_j)\|^2 )
       +2\sum_{i=1}^N  (\tilde{U}_i-\Omega). \sum_{j=1}^N  (\Omega-\tilde{U}_j)^2
\end{equation}

\begin{equation}
    S= 2N * \sum_{i=1}^N  \|(\tilde{U}_i-\Omega) \|^2
\end{equation}

    % - - - - - - - - - - - - - - - - - - - - - - - - - - - - - - - -

\section{Interpolation}

