\chapter{Some Realization Examples}


This chapter contains some 3D models realized with the tools presented
in this documentation. Its purpose is to give an idea of what can be achieved with these tools,
it has no didactic purpose.
For now it is just a gallery, some comments will be added later.

There are also several interesting sites corresponding to use
cases in cultural heritage and environmental application :

\begin{itemize}
   \item \url{http://www.tapenade.gamsau.archi.fr}, for architecture and
         archeology;

   \item \url{https://sites.google.com/site/geomicmac/home/documentation}, for geology and surveying;

   \item \href{http://combiencaporte.blogspot.fr/2013/10/micmac-tutoriel-de-photogrammetrie-sous.html}{micmac-tutoriel-de-photogrammetrie-sous},
         for architecture;

    \item \url{http://c3dc.fr/galerie/}, for cultural heritage
\end{itemize}

The {\tt DocMicMac} directory \footnote{where this file is originally located}
there are also several documents more or less related to these tools :

\begin{itemize}
   \item  the directory {\tt Paper-MPD} contains some papers describing {\tt Apero/MicMac}
          and protocols :

          \begin{itemize}
              \item {\tt MPD-Eurocow-17.docx} focus on acquisition protocols for orientation;
              \item {\tt Collection\_EDYTEM\_12-2011\_Images\_et\_modèles\_3D\_en\_milieux\_naturels.pdf}
                    focus on application to natural environments;
          \end{itemize}

   \item  the directory {\tt Paper-UseCase/} contains papers from colleagues that describe some experimentations with {\tt Apero/MicMac};

   \item  the directory {\tt Papers-Internship/} contains some reports of internships done
          by students of our school and using {\tt Apero/MicMac};

   \item  the directory {\tt Paper-Algo//} contains papers relative to some algorithmic aspects;
\end{itemize}

%-------------------------------------------------------------------
%-------------------------------------------------------------------
%-------------------------------------------------------------------


\section{3D Objects}

         %  - - - - - - - - - - - - - -
\subsection{Statues}
See~\ref{View:Eleph}

         %  - - - - - - - - - - - - - -
\subsection{Architectural Details}



%-------------------------------------------------------------------
%-------------------------------------------------------------------
%-------------------------------------------------------------------

\section{Indoors Global Modelization}
         %  - - - - - - - - - - - - - -
\subsection{Architecture}



%-------------------------------------------------------------------
%-------------------------------------------------------------------
%-------------------------------------------------------------------
\section{Globally Planar Objects}

         %  - - - - - - - - - - - - - -
\subsection{Elevation and ortho images}
         %  - - - - - - - - - - - - - -
\subsection{Painting and Fresco}
         %  - - - - - - - - - - - - - -
\subsection{Bas-relief}
         %  - - - - - - - - - - - - - -
\subsection{Macro-photo}


%-------------------------------------------------------------------
%-------------------------------------------------------------------
%-------------------------------------------------------------------
\section{Aerial Photos}

\subsection{Urban DEM}
\subsection{Satellite Images}
\subsection{UAV Missions}


%-------------------------------------------------------------------
%-------------------------------------------------------------------
%-------------------------------------------------------------------
\section{Miscellaneous}
\subsection{Industrial}


%-------------------------------------------------------------------
%-------------------------------------------------------------------
%-------------------------------------------------------------------
\section{Gallery of images}

\newpage




\begin{figure}
\begin{tabular}{||c|c||}
   \hline \hline
   \includegraphics[width=80mm]{FIGS/SAMPLES/DoocDOC-Eleph-Photo.jpg} &
   \includegraphics[width=80mm]{FIGS/SAMPLES/Doc-Elph-Shade.jpg}   \\ \hline  \hline
   \multicolumn{2}{||c||}{ \includegraphics[width=150mm]{FIGS/SAMPLES/Doc-Pl-Eleph.jpg}} \\ \hline  \hline
   \multicolumn{2}{||c||}{ \includegraphics[width=150mm]{FIGS/SAMPLES/Doc-Eleph-3D.jpg}} \\ \hline  \hline
\end{tabular}
\caption{{\bf Statues:} elephant in Chian long temple, one of the 60 images, a shaded mode, a global view of the 60 image,
the $3$D model and the camera position}
\label{View:Eleph}
\end{figure}

\begin{figure}
\begin{tabular}{||c|c|c|c||}
   \hline \hline
   \includegraphics[width=40mm]{FIGS/SAMPLES/P1-AB.JPG} &
   \includegraphics[width=40mm]{FIGS/SAMPLES/P1-Shade.jpg} &
   \includegraphics[width=40mm]{FIGS/SAMPLES/P2-A.JPG} &
   \includegraphics[width=40mm]{FIGS/SAMPLES/P2-Shade.jpg} \\
  \\ \hline  \hline
\end{tabular}
\caption{{\bf Statues:} Zhenjue temple}
\label{View:Zhenjue}
\end{figure}



\begin{figure}
\includegraphics[width=70mm]{FIGS/SAMPLES/Aj2.jpg}
\includegraphics[width=100mm]{FIGS/SAMPLES/Aj1.jpg}
\caption{{\bf Indoor architecture: } Chapelle imperiale Ajaccio, with 100 fish-eye images;
left position of camera, right 3D model}
\end{figure}



\begin{figure}
\includegraphics[width=160mm]{FIGS/SAMPLES/Pompei-Planche.jpg}
\includegraphics[width=160mm]{FIGS/SAMPLES/Pompei-Shade2.jpg}
\includegraphics[width=160mm]{FIGS/SAMPLES/Pompei-Ortho1.jpg}
\caption{The set of images acquired on a wall in Pompei, a global
view of the 3D model of the wall and a global view of orthophoto}
\end{figure}

\begin{figure}
\centering
\includegraphics[width=160mm]{FIGS/SAMPLES/Pompei-Shade1.jpg}
\includegraphics[width=160mm]{FIGS/SAMPLES/Pompei-Ortho2.jpg}
\caption{Detail on 3D model and ortho photo in Pompei}
\end{figure}


\begin{figure}
\includegraphics[width=150mm]{FIGS/SAMPLES/LambaleShade.jpg}
\includegraphics[width=150mm]{FIGS/SAMPLES/Lamballe-Ortho-Test-Redr.jpg}
\caption{3D model and ortho photo on "Coll\'egiale Notre Dame de la Garde (Lamballe)"}
\end{figure}




\begin{figure}
\includegraphics[width=84mm]{FIGS/SAMPLES/Louvre-Superp.jpg}
\includegraphics[width=75mm]{FIGS/SAMPLES/FresqVilAv.jpg}
\caption{{\bf Painting and Fresco: } Fine depth maps computation on painting and fresco,
images and in superposition level curves and hypsometry;
left image, photo C2RMF/Jean Marsac; right fresco in Villeneuve
les Avignon, photo CICRP/Odile Guillon}
\end{figure}



\begin{figure}
\begin{tabular}{||c|c|c|c||}
   \hline \hline
   \includegraphics[width=40mm]{FIGS/SAMPLES/FRISE-IM1.jpg} &
   \includegraphics[width=40mm]{FIGS/SAMPLES/FRISE-IM2.jpg} &
   \includegraphics[width=40mm]{FIGS/SAMPLES/FRISE-Det1-F8Bits.jpg}&
   \includegraphics[width=40mm]{FIGS/SAMPLES/FRISE-Detail-ScaledShade.jpg} \\ \hline  \hline
   \multicolumn{4}{|c|}{\vspace{2mm} \includegraphics[width=160mm]{FIGS/SAMPLES/FriseDepthBits.jpg}} \\ \hline  \hline
   \multicolumn{4}{|c|}{\vspace{2mm} \includegraphics[width=160mm]{FIGS/SAMPLES/FrisScaledShade.jpg}} \\ \hline  \hline
\end{tabular}
\caption{{\bf Bas Relief} Frise in Villeneuve-lès-Maguelone \dots}
\end{figure}


\begin{figure}
\begin{tabular}{||c|c|c||}
   \hline \hline
   \includegraphics[width=52mm]{FIGS/SAMPLES/Hiero1_2908.JPG} &
   \includegraphics[width=52mm]{FIGS/SAMPLES/Hiero2_2908.JPG} &
   \includegraphics[width=52mm]{FIGS/SAMPLES/Hiero3_2908.JPG} \\ \hline  \hline
\end{tabular}
\caption{{\bf Bas Relief} stone in Louvre, image and 3D model rendered in shading and depth map \dots}
\end{figure}



\begin{figure}
\begin{tabular}{||c|c||}
   \hline \hline
   \includegraphics[width=80mm]{FIGS/SAMPLES/ROUFF-IM.JPG}&
   \includegraphics[width=80mm]{FIGS/SAMPLES/ROUFF-SHADE.jpg} \\ \hline  \hline
\end{tabular}
\caption{{\tt Macro photography} in Rouffignac cave, at $\frac{1}{20}$ mm resolution: image and 3D model shading}
\end{figure}


\begin{figure}
\begin{tabular}{||c|c||}
   \hline \hline
   \includegraphics[width=72mm]{FIGS/SAMPLES/Doc-Pijo1.jpg}&
   \includegraphics[width=75mm]{FIGS/SAMPLES/Doc-Pijo2.jpg} \\ \hline  \hline
    \multicolumn{2}{|c|}{\includegraphics[width=150mm]{FIGS/SAMPLES/Doc-Pijo3.jpg}}  \\ \hline  \hline
\end{tabular}
\caption{{\tt Macro photography} booklet of Mayenne science cave \dots}
\end{figure}











\begin{figure}
\includegraphics[width=80mm]{FIGS/SAMPLES/DGPF_8Bits.jpg}
\includegraphics[width=80mm]{FIGS/SAMPLES/DGF-SHADE.jpg}
\caption{Digital elevation model on semi urban area, 8cm resolution, DGPF data set, for
Euro-SDR benchmarking on image matching}
\end{figure}

\begin{figure}
\includegraphics[width=160 mm]{FIGS/SAMPLES/DGPF-GLOB.jpg}
\caption{A more global view of DEM on DGPF data set}
\end{figure}




\begin{figure}
\begin{tabular}{||c|c||}
\hline \hline
\includegraphics[width=80 mm]{FIGS/SAMPLES/Salse2.jpg}&
\includegraphics[width=80 mm]{FIGS/SAMPLES/Ortho-Test-Redr.jpg} \\ \hline  \hline
\\ \multicolumn{2}{|c|}{\includegraphics[width=160 mm]{FIGS/SAMPLES/SlsMNE_1_25.jpg}} \\ \hline  \hline
 \end{tabular}
\caption{Forteresse de Salses, photo acquired by drone survey, in collaboration
with Map-CNRS; hypsometry and shading, ortho photography, oblique view of 3D model, Euro-SDR benchmarking on image matching}
\end{figure}



